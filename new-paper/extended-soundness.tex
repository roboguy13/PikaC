\section{Soundness Proof}
\label{sec:soundness-proof}

First, we introduce an operator that will make it more convenient to talk about the conjunction of two SSL propositions:
\[
  (p_1 ; s_1) \otimes (p_2 ; s_2) = (p_1 \land p_2 ; s_1 \sep s_2)
\]

\noindent
Next, we present a lemma regarding this operator that will be used in the soundness proof:
\begin{lemma}[$\otimes$ pairing]\label{thm:otimes-entail}
  If $(\sigma_1, h_1) \models p$ and $(\sigma_2, h_2) \models q$ and $\sigma_1 \subseteq \sigma_2$ and $h_1 \mathbin{\bot} h_2$,\\
  then $(\sigma_2, h_1 \circ h_2) \models p \otimes q$
\end{lemma}

\noindent
Now we proceed to soundness.

\soundnessThm*

\begin{proof}
  Proceeding by induction on $e$:
  \begin{itemize}
    \item $e = i$ for some $i \in \mathbb{Z}$.\\
      We have 
      \begin{itemize}
        \item $\Tsem{i}{V,r} = (v == i ; \emp)$
        \item $v \fresh V$
        \item $v = r$
        \item $\EndSigma = \StartSigma \cup \{ (r,i) \}$
        \item $\End{h'} = \Start{h} = \emptyheap$
      \end{itemize}
      Plugging these in, we find that we want to prove
      \[
        (\StartSigma \cup \{ (r,i) \}, \emptyheap) \models (r == i ; \emp)
      \]
      This immediately follows from the rules of separation logic.

    \item $e = b$ for some $b \in \mathbb{B}$. This case proceeds exactly as the previous case.

    \item $e = e_1 + e_2$.
      From the $\Tsem{\cdot}{}$ hypothesis, we have
      \begin{itemize}
        \item $(e_1, V_0) \tstep (p_1, s_2, V_1, v_1)$
        \item $(e_2, V_1) \tstep (p_2, s_2, V_2, v_2)$
        \item $v \fresh V_2$
        \item $v = r$
        \item $\Tsem{e_1 + e_2}{V, r} = (v == v_1 + v_2 \land p_1 \land p_2 ; s_1 \sep s_2)$
        \item $\Tsem{e_1}{V_0,v_1} = (p_1 ; s_1)$
        \item $\Tsem{e_2}{V_1,v_2} = (p_2 ; s_2)$
        % \item $ $
      \end{itemize}

      Plugging these in, we want to show
      \[
        \EndPair \models (v == v_1 + v_2 \land p_1 \land p_2 ; s_1 \sep s_2)
      \]

      Furthermore, we see that
      \[
        \Tsem{e}{V,r} = (v == v_1 + v_2 ; \emp) \otimes \Tsem{e_1}{V_0,v_1} \otimes \Tsem{e_2}{V_1,v_2}
      \]

      From the {\sc AM} relation hypothesis we have
      \begin{enumerate}
        \item $\EndSigma = \sigma_y \cup \{(r, z)\}$
        \item $z = x' + y'$
        \item $(x, \sigma, \mathcal{F}, h_1) \step (x', \sigma_x, h_1', \mathcal{F}, v_x)$
        \item $(y, \sigma_x, \mathcal{F}, h_2) \step (y', \sigma_y, h_2', \mathcal{F}, v_y)$
        \item $v_1 = v_x$
        \item $v_2 = v_y$
        \item $h = h_1 \circ h_2$
        \item $h' = h_1' \circ h_2'$
      \end{enumerate}

      From the first six items, we can derive
      \[
        (\EndSigma, \emptyheap) \models (v == v_1 + v_2 ; \emp)
      \]

      From the inductive hypotheses, we get
      \begin{itemize}
        \item $(\sigma_x, h_1') \models \Tsem{e_1}{\dom(\sigma_x), v_x}$
        \item $(\sigma_y, h_2') \models \Tsem{e_2}{\dom(\sigma_y), v_y}$
      \end{itemize}

      By Lemma~\ref{thm:otimes-entail}
      \[
        (\sigma_y, h_1' \circ h_2') \models \Tsem{e_1}{\dom(\sigma_x), v_x} \otimes \Tsem{e_2}{\dom(\sigma_y), v_y}
      \]

      and we know that $\sigma_x \subseteq \sigma_y$.

      By Lemma~\ref{thm:otimes-entail}, we conclude
      \[
        (\sigma_y \cup \{(r, z)\}, h') \models (v == v_1 + v_2 ; \emp) \otimes \Tsem{e_1}{\dom(\sigma_x),v_1} \otimes \Tsem{e_2}{\dom(\sigma_y),v_2}
      \]

    \item $e = v$ for some $v \in \Var$.
      Therefore the only two {\sc AM} rules that apply are {\sc AM-Base-Var} and {\sc AM-Loc-Var}. Both
      cases proceed in the same way.

      We know
      \begin{itemize}
        \item $h = h' = \emptyheap$
      \end{itemize}

      So we want to show
      \[
        (\EndSigma, \emptyheap) \models (\verb|true| ; \emp)
      \]
      This trivially holds.

    \item $e = \lowerS{A}{v}$ for some layout $A$ and $v \in \Var$.
      The applicable {\sc AM} rule is {\sc AM-Lower}.
      From the premises of {\sc AM-Lower}, we have
      \begin{itemize}
        \item $h' = \hat{h} \cdot H'$
      \end{itemize}
      where $H'$ comes from applying the layout $A$ to the constructor application expression
      obtained by reducing $e$. As a result, $H'$ exactly fits the specification of
      one of the branches of the $A$ layout.

      $v$ must be associated with some high-level value in $\mathcal{F}$ and $\hat{h}$
      is the part of the heap that is updated when this is reduced. Since all expressions
      are required to be well-typed, $\hat{h}$ must satisfy some collection of heaplets
      in the $A$ layout branch that is associated to $H'$.

      Note that $\Tsem{e}{V,r} = A(v)$

      From these facts, we can conclude
      \[
        \EndPair \models A(v)
      \]

    \item $e = \lowerS{A}{C\; e_1 \cdots e_n}$. This case is similar to the previous case,
      except that we do not need to lookup a variable in the store before dealing with the constructor
      application.

    \item $e = \instantiateS{A,B}{f}(v)$ for some $v \in \Var$.
        The rule {\sc AM-Instantiate} applies here.

        The {\sc S-Inst-Var} rule requires that we satisfy the inductive predicate
        associated to $f$, that is $\mathcal{I}_{A,B}(f)(v)$. Note that the SSL propositions in the definition of
        such an inductive predicate, given by {\sc FnDef}, will always not only explicitly
        describe the memory used in its result (given by the layout $B$) but also the memory used
        in its argument (given by the layout $A$).

        Now, consider that the {\sc AM-Instantiate} rule uses the layout $A$ to the function argument to
        update the heap and uses the $B$ layout to produce the function's result on the heap. This will match the
        inductive predicate associated to $f$ instantiated at the layouts $A$ and $B$, as required.

      %   Among our hypotheses, we have
      % \begin{itemize}
      %   \item $\End{h'} = \hat{h}_1 \circ \hat{h}_2 \circ \cdots \circ \hat{h}_{n+1}$
      %   \item $\EndSigma = \sigma_f$
      %   \item $(A[x]\; (C\; a_1 \cdots a_n) := H) \in \Sigma$
      %   \item $(f\; (C\; b_1 \cdots b_n) := e_f) \in \Sigma$
      %   \item $(\lowerS{B}{e_f[b_1 := r_1]\cdots[b_n := r_n]}, \sigma_{n+1}, \hat{h}_1) \step (e_f', \sigma_f, \hat{h}_2, r)$
      % \end{itemize}
      %
      % Let $\hat{h} = \hat{h}_1 \circ \hat{h}_2 \circ \cdots \circ \hat{h}_n$. Note that we stop at $n$, not $n+1$. This is the
      % difference between $\hat{h}$ and $\End{h'}$.
      %
      % We can recursively apply the present lemma to the last item, since it is a derivation of an abstract machine reduction which is
      % strictly smaller than the abstract machine reduction derivation we started with. From this, we obtain:
      %
      % \[
      %   (\EndSigma, \hat{h}) \models \lowerS{B}{e_f[b_1 := r_1]\cdots[b_n := r_n]}
      % \]


      % and from the {\sc AM} relation hypothesis we have
      % \begin{itemize}
      %   \item $(x, \sigma, \mathcal{F}, h) \step (x', \sigma_x, h, \mathcal{F}, v_x)$
      % \end{itemize}

      \item $e = \instantiateS{A,B}{f}(C\; e_1 \cdots e_n)$.
        This is much like the previous case. The only difference is that the {\sc S-Inst-Constr} rule
        unfolds the inductive predicate $\mathcal{I}_{A,B}(f)(C\; e_1 \cdots e_n)$. The resulting SSL proposition
        will still be satisfied by the model $\EndPair$ given by {\sc AM-Instantiate}.

      \item $e = \instantiateS{B,C}{f}(\instantiateS{A,B}{g}(e_0))$.
        This case combines to instantiates together. The main condition we need to check here is that the
        two instantiate applications use disjoint parts of the heap in $\EndPair$ in {\sc AM-Instantiate}.

        This can be seen to be true by the fact that the resulting heap is built up out of the subexpressions
        using $\circ$, ensuring that the parts of the heap are disjoint.
  \end{itemize}
  % {\color{red} ???}
\end{proof}
