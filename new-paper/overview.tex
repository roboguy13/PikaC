\section{Overview}
\label{sec:overview}

\subsection{Background}
The \Pika{} language is translated into
\SuSLik~\cite{polikarpova:2019:suslik}, which is a program synthesis
tool that uses Synthetic Separation Logic (SSL)-a variant of
Hoare-style~\cite{hoare:1969:axiomatic} Separation Logic
(SL)~\cite{ohearn:2001:seplogic}.

% SSL is a variant of separation
% logic that works particularly well for program
% synthesis.~\cite{polikarpova:2019:suslik}

A synthesis task specification in \SuSLik{} is given as a function
signature together with a pair of pre- and post-conditions, which are
both SL assertions~\cite{polikarpova:2019:suslik}.
%
The synthesiser generates code that satisfies the given specification,
along with the SL \emph{proof} of its correctnees by performing the
search in a space of proofs that can be derived by using the rules of
the underlying logic~\cite{WatanabeGPPS21}.
%
A distinguishing feature of \textit{Synthetic} Separation Logic is the
format of its assertions. An SSL assertion consists of two parts: a
\textit{pure part} and a \textit{spatial part}. The pure part is a
Boolean expression constraining the variables of the specification
using a few basic relations, like equality and the less-than relation
for integers. The spatial part is a \textit{symbolic heap}. This
consists of a list of heaplets. The list separator for the list of
heaplets is the $\sep$ symbol. Each heaplet takes on one of the
following forms~\cite{polikarpova:2019:suslik}:

\begin{itemize}
  \item \texttt{emp}: This represents the empty heap. It is also the left and right identity for $\sep$.
  \item $\ell \mapsto a$: This asserts that memory location $\ell$ points to the value $a$.
  \item $[\ell, \iota]$: At memory location $\ell$, there is a block of size $\iota$.
  \item $p(\overline{\phi})$: This is an application of the
    \textbf{\textit{inductive predicate}} $p$ to the arguments
    $\overline{\phi}$. An inductive predicate has a collection of
    branches, guarded by Boolean expressions (conditions) on the
    parameters. The body of each branch is an SSL assertion. The
    assertion associated with the first branch condition that is
    satisfied is used in place of the application. Note that inductive
    predicates can be (and often are) recursively defined.
\end{itemize}

\noindent
The general form of an SSL assertion is
$(p; h_1 \sep h_2 \sep \cdots \sep h_n)$, where $p$ is the pure part
and $h_1, h_2, \cdots, h_n$ are the heaplets which are the conjuncts
that make up the separating conjunction of the spatial part. A syntax
definition for SSL is given in \autoref{fig:SSL-syntax}, which is
adapted from \textit{Cyclic Program Synthesis} by Itzhaky
\etal~\cite{itzhaky:2021:cyclic-synth}.

\begin{figure}[t]
\setlength{\abovecaptionskip}{5pt}
%\setlength{\belowcaptionskip}{-15pt}
{\small{
\[
  \begin{array}[t]{ l l }
    \text{Variable} &x, y\;\; \text{Alpha-numeric identifiers $\in$ \textnormal{Var}}\\
    \text{Size, offset} &n, \iota\;\; \text{Non-negative integers}\\
    \text{Expression} &e ::= 0\; \mid \texttt{true} \mid x \mid e = e \mid e \land e \mid \neg e \mid d\\
    \text{$\mathcal{T}$-expr.} &d ::= n \mid x \mid d + d \mid n \cdot d \mid \{\} \mid {d} \mid \cdots\\
    \text{Command} &c ::= \begin{aligned}[t] &\texttt{let x = *(x + }\iota\texttt{)} \mid \texttt{*(x + }\iota\texttt{) = e} \mid
      \\
      &\texttt{let x = malloc(n)} \mid \texttt{free(x)} \mid \texttt{error}\newline
      \mid \texttt{f(}\overline{e_i}\texttt{)}
      \end{aligned}\\
    \text{Program} &\Pi ::= \overline{f(\overline{x_i})\; \{\; c\; \}\; ;}\; c\\
    \text{Logical variable} &\nu, \omega\\
    \text{Cardinality variable} &\alpha\\
    \text{$\mathcal{T}$-term} &\kappa ::= \nu \mid e \mid \cdots\\
    \text{Pure logic term} &\phi, \psi, \chi ::= \kappa \mid \phi = \phi \mid \phi \land \phi \mid \neg \phi\\
    \text{Symbolic heap} &P, Q, R ::= \texttt{emp} \mid \mbox{$\langle e, \iota \rangle \mapsto e \mid [e, \iota]$} \mid p^{\alpha}(\overline{\phi_i})
      \mid \mbox{$P * Q$}\\
    \text{Heap predicate} &\mathcal{D} ::= p^{\alpha}(\overline{x_i}) : \overline{e_j \Ra \exists \overline{y}.\{\chi_j;R_j\}}\\
    \text{Assertion} &\mathcal{P},\mathcal{Q} ::= \{\phi; P\}\\
    \text{Environment} &\Gamma ::= \forall\overline{x_i}.\exists\overline{y_j}.\\
    \text{Context} &\Sigma ::= \overline{\mathcal{D}}\\
    \text{Synthesis goal} &\mathcal{G} ::= P \leadsto Q
  \end{array}
\]
}}
\caption{Syntax of Synthetic Separation Logic}
  \label{fig:SSL-syntax}
\end{figure}

\begin{figure}
  \ruleSAdd
  \caption{{\sc S-Add} rule}
  \label{fig:ruleSAdd}
\end{figure}

As the specification language of \SuSLik, SSL serves as the
compilation target for the \Pika{} language. From there, executable
programs are generated through \SuSLik's program synthesis. Consider a
program that takes an integer $x$ and a result in location $r$ and
stores $x+1$ at location $r$. This can be written as the \SuSLik{}
specification:

\begin{lstlisting}[language=SynLang]
void add1Proc(int x, loc r)
  { r :-> 0 }
  { y == x+1 ; r :-> y }
{ ?? }
\end{lstlisting}

\noindent
In contrast, this example can be written as follows in \Pika:

\begin{lstlisting}[language=Pika]
add1Proc : Int -> Int;
add1Proc x := x + 1;
\end{lstlisting}

\noindent
In constrast with \SuSLik{} specs, the one below features no direct
manipulation with pointers.
%
This program is translated using the {\sc S-Add} rule shown in
\autoref{fig:ruleSAdd}. 

\subsection{The \Pika{} Language}
\label{sec:language}

While SSL provides a specification language that allows tools like
\SuSLik{} to synthesise code, it is only able to express specifications
as pointers. This is useful for some applications, such as embedded
systems, but it does not provide any high-level abstractions. As a
result, every part of a specification is tailored to a specific memory
representation of each data structure involved.
%
To addresss this shortcoming, we introduce a language with algebraic
data types that gets translated into SSL specifications. Additionally,
we introduce a language construct that allows the programmer to
specify a memory representation of an algebraic data type. This is
called a \textit{layout}. This distinction between algebraic data
types and layouts allows the separation of concerns between the low
level representation of a data structure and code that manipulates it
at a high level.

Syntactically, \Pika{} resembles a functional programming language
of the Miranda~\cite{turner:1986:miranda} and
Haskell~\cite{hudak:2007:haskell} lineage. It supports algebraic data
types, pattern matching at top-level function definitions (though not
inside expressions) and Boolean guards. The primary difference arises
due to the existence of layouts and the fact that the language is
compiled to an SSL specification rather than executable code.

Functions in \Pika{} are only defined by their operations on algebraic
data types. Thus, all function definitions are ``layout polymorphic''
over the particular choices of layouts for their arguments and result.
Giving a layout-polymorphic function, a particular choice of layouts
is called ``instantiation''. Specifying the layout of a non-function
value is called ``lowering.''

There is a subtlety here: The layout of a subexpression inside of a
function definition might not be determined by the layouts that the
function definition is instantiated at. In this case, we must
explicitly specify a layout. This is done using the \verb|instantiate|
and \verb|lower| constructs. The former is for function calls, while
the latter is for all other types of values. These operations are
performed with the \verb|instantiate| and \verb|lower| language
constructs, respectively. For example, \verb|instantiate [A] B f x|
calls $f$ with the argument $x$, where the layout $A$ is used for the
argument and the layout $B$ is used for the result of the function
application. This is also written $\instantiateS{A,B}{f}(x)$. The
latter is primarily used in \autoref{sec:semantics}.

The code generator is instructed to generate a \SuSLik{} specification
for a certain function at a certain instantiation by using a
\generate{} directive. For example, if there is a function definition
with the type signature \verb|mapAdd1 : List -> List|, a line reading
\verb|%generate mapAdd1 [Sll] Sll| would instruct the
\Pika{} compiler to generate the \SuSLik{} inductive predicate
corresponding to \verb|mapAdd1| instantiated to the \verb|Sll| layout
for both its argument and its result.

An example of an ADT definition and a corresponding layout definition
is given in \autoref{fig:List-def}. There is one unusual part of
the syntax in particular which requires further explanation: layout
type signatures. A layout definition consists a \textit{layout type
  signature} and a pattern match (much like a function definition),
with lists of SSL heaplets on the right-hand sides. A layout type
signature has a special form $A : \alpha \monic \layout[x]$. This says
that the layout $A$ is for the algebraic data type $\alpha$ and the
SSL output variable is named $x$.
% The full \Pika{} grammar is given in
% the Appendices. %x~\ref{sec:grammar}.

\begin{figure}[t]
\begin{lstlisting}[language=Pika]
data List := Nil | Cons Int List;

Sll : List >-> layout[x];
Sll (Nil) := emp;
Sll (Cons head tail) := x :-> head ** (x+1) :-> tail ** Sll tail;
\end{lstlisting}
  \caption{\lstinline{List} algebraic data type together with its singly-linked list layout \lstinline{Sll}}
  \label{fig:List-def}
\end{figure}

\subsection{\Pika{} by Example}

We demonstrate the characteristic usages of \Pika{} on a series of
examples.
%
In these examples, we will often make use of the \verb|List| algebraic
data type and its \verb|Sll| layout from~\autoref{fig:List-def}. A
simple example of \Pika{} code that illustrates algebraic data types and
layouts is a function which creates a singleton list out of the given
integer argument:

\begin{lstlisting}[language=Pika]
%generate singleton [Int] Sll

singleton : Int -> List;
singleton x := Cons x (Nil);
\end{lstlisting}

\noindent
This gets compiled to the following \SuSLik{} specification, with generated names simplified for greater clarity:

\begin{lstlisting}[language=SynLang]
predicate singleton(int p, loc r) {
| true => { r :-> p ** (r+1) :-> 0 ** [r,2] }
}
\end{lstlisting}

\noindent
A slightly more complicated example comes from trying to write a
functional-style \verb|map| function directly in \SuSLik. Consider a
function which adds 1 to each integer in a list of integers.
Considering the list implementation to be a singly-linked list with a
fixed layout, one way to express this in \SuSLik{} is shown in \autoref{fig:sllx}

\begin{figure}[t]
  \begin{lstlisting}[language=SynLang]
predicate sll(loc x) {
| x == 0 => { emp }
| not (x == 0) => { [x, 2] ** x :-> v ** (x+1) :-> nxt ** sll(nxt) }
}

predicate mapAdd1(loc x, loc r) {
| x == 0 => { emp }
| not (x == 0) => {
    [x, 2] ** x :-> v ** (x+1) :-> xNxt **
    [r, 2] ** r :-> (v+1) ** (r+1) :-> rNxt **
    mapAdd1(xNxt, rNxt)
  }
}

void mapAdd1_fn(loc x, loc y)
  { sll(x) ** y :-> 0 }
  { y :-> r ** mapAdd1(x, r) }
{ ?? }
\end{lstlisting}
\caption{Specifying a function that adds one to each element of a
  singly-linked list in \SuSLik.}
\label{fig:sllx}
\end{figure}

\noindent
Note that inductive predicates are used for \textit{two} different purposes: the \verb|sll| inductive
predicate describes a singly-linked list data structure, while the \verb|mapAdd1| inductive predicate 
describes how the input list relates to the output list. Both are used in the specification of
\verb|mapAdd1_fn|: \verb|sll| in the precondition and \verb|mapAdd1| in the postcondition.

Using the \verb|mapAdd1| inductive predicate gives us two advantages over attempting to
put the SSL propositions directly into the postcondition of \verb|mapAdd1_fn|:
%
\begin{enumerate}
  \item We are able to express a conditional on the shape of the list. This is much like pattern
    matching in a language with algebraic data types, but we are examining the pointer involved directly.
  \item We are able to express \textit{recursion} part of the postcondition: the \verb|mapAdd1| inductive
    predicate refers to itself in the \verb|not (x == 0)| branch.
\end{enumerate}
%
These two features are both reminiscent of features common in
functional programming languages: pattern matching and recursion.
However, there are still some significant differences:

\begin{itemize}
  \item In traditional pattern matching, the underlying memory representation of the data structure is not exposed.
  \item Compared to a functional programming language, the meaning of the specification is more obscured. It is
    necessary to think about the structure of the linked data structure to determine what the specification is saying. This
    is related to the first point: The memory representation is front-and-center.
  \item In many functional languages, mutation is either restricted or generally discouraged. In \SuSLik, mutation is commonplace.
\end{itemize}

Say we want to write the functional program that corresponds to this specification. One way to do this in a Haskell-like language
is the following, using the \verb|List| type from \autoref{fig:List-def}.

\begin{lstlisting}[language=Pika]
mapAdd1_fn : List -> List;
mapAdd1_fn (Nil) := 0;
mapAdd1_fn (Cons head tail) := Cons (head + 1) (mapAdd1_fn tail);
\end{lstlisting}

\noindent
The only missing information is the memory representation of the \verb|List| data structure. We do not want the
\verb|mapAdd1_fn| implementation to deal with this directly, however. We want to separate the more abstract notions
of pattern matching and constructors from the concrete memory layout that the data structure has.

To accomplish this, we now extend the code with the definition of
\verb|Sll| from \autoref{fig:List-def}. \verb|Sll| is a
\textit{layout} for the algebraic data type \verb|List|.
%
Now we have all of the information of the original specification but
rearranged so that the low-level memory layout is separated from the
rest of the code. This separation brings us to an important
observation about the language, manifested throughout these examples:
none of the function definitions need to \emph{directly} perform any
pointer manipulations. This is relegated entirely to the reusable
layout definitions for the ADTs. The examples are written entirely as
recursive functions that pattern match on, and construct, ADTs.

All that is left is to connect these two parts: the layouts and the
function definitions. We instruct a \SuSLik{} specification generator to
generate a \SuSLik{} specification from the \verb|mapAdd1_fn| function
using the \verb|Sll| layout:
%
\begin{lstlisting}[language=Pika]
%generate mapAdd1_fn [Sll] Sll
\end{lstlisting}
%
The \verb|[Sll]| part of the directive tells the generator which
layouts are used for the arguments. In this case, the function only
has one argument and the \verb|Sll| layout is used. The \verb|Sll| at
the end specifies the layout for the result.

\subsubsection{map}

We can generalise our \verb|mapAdd1| to map arbitrary \verb|Int|
functions over a list and then redefine \verb|mapAdd1| using the new
\verb|map|.

\begin{lstlisting}[language=Pika]
%generate mapAdd1 [Sll] Sll

data List := Nil | Cons Int List;

Sll : List >-> layout[x];
Sll (Nil) := emp;
Sll (Cons head tail) := x :-> head ** (x+1) :-> tail ** Sll tail;

map : (Int -> Int) -> List -> List;
map f (Nil) := Nil;
map f (Cons x xs) := Cons (instantiate [Int] Int f x) (map f xs);

add1 : Int -> Int;
add1 x := x + 1;

mapAdd1 : List -> List;
mapAdd1 xs :=
  instantiate [Int -> Int, Sll] Sll map add1 xs;
\end{lstlisting}

This example makes use of \lstinline{instantiate} in two places. In the first case where we have the call
\lstinline{instantiate [Int] Int f x}, the builtin \verb|Int| layout is used for both the input
and output. In this special case, the \verb|Int| layout shares a name with the \verb|Int| type
that it represents. This is necessary since \lstinline{instantiate} is used for all non-recursive calls that are not constructor applications.

In the second use, \lstinline{instantiate [Int -> Int, Sll] Sll map add1 xs}, we specify that
the second argument uses the \verb|Sll| layout for the \verb|List| type from \autoref{fig:List-def}. We also give \verb|Sll| as
the layout for the result of the call. Note that it is not necessary to use \verb|instantiate| for the recursive call to \verb|map|. This is because the appropriate layout
is inferred for recursive calls.

It might be surprising that \verb|instantiate| is required in the body of \verb|mapAdd1| since the type signature
of \verb|mapAdd1| suggests that it is layout polymorphic and yet we must pick a specific \verb|List| layout when
we use \verb|instantiate| to call \verb|map|. This is because, in general, a call inside the body of some function \verb|fn| might use
any layout, even layouts that have no relation to the layouts that \verb|fn| is instantiated to.

It is possible to do inference of some layouts, for example in
\verb|mapAdd1| we would usually want to use the same layout as the
argument layout, but we leave this for future work. Another approach
is to introduce type variables that correspond to layouts, as done in
the series of works on the \tname{Dargent}{}
tool~\cite{chen:2023:dargent}.
%
We leave this approach for future work as well.

\subsubsection{Guards}

While we have a pattern-matching construct at the top level of a function definition, we
have not seen a way to branch on a Boolean value so far. This is a feature that is
readily available at the level of \SuSLik, since the same conditional construct we
use to implement pattern matching can also use other Boolean expressions.

We can expose this in the functional language using a \textit{guard},
much like Haskell's guards. Say, we want to write a specialised
filter-like function. Specifically, we want a function that filters
out all elements of a list that are less than 9. This is a specific
example where the \SuSLik{} specification is noticeably more difficult
to read. For a \SuSLik{} specification of this example, see
\autoref{fig:SuSLik-filter}

\begin{figure}
  \begin{lstlisting}[language=SynLang]
predicate filterLt9(loc x, loc r) {
| (x == 0) => { r == 0 ; emp }
| not (x == 0) && head < 9 =>
    { x :-> head ** (x+1) :-> tail ** [x,2] ** filterLt9(tail, r) }
| not (x == 0) && not (head < 9) =>
    { x :-> head ** (x+1) :-> tail ** [x,2] ** filterLt9(tail, y)
      ** r :-> head ** (r+1) :-> y ** [r,2] }
}

void filterLt9(loc x1, loc r)
  { Sll(x1) ** r :-> 0 }
  { filterLt9(x1, r0) ** r :-> r0 }
{ ?? }
  \end{lstlisting}
  \caption{\SuSLik{} specification of \lstinline{filterLt9}, excluding \lstinline{Sll} which is given in \autoref{fig:List-def}}
  \label{fig:suslik-filter}
\end{figure}

On the other hand, an implementation of this in \Pika{} is:

\begin{lstlisting}[language=Pika]
%generate filterLt9 [Sll] Sll

filterLt9 : List -> List;
filterLt9 (Nil) := Nil;
filterLt9 (Cons head tail)
  | head < 9       := filterLt9 tail;
  | not (head < 9) := Cons head (filterLt9 tail);
\end{lstlisting}

\noindent
When translating a guarded function body, the translator takes the conjunction of the Boolean guard condition with
the condition for the pattern match.

\subsubsection{\texttt{let} bindings and \texttt{if-then-else}}

Two more features that are common in functional languages are \verb|let| bindings and
\verb|if-then-else| expressions. Both of these have straightforward translations into
\SuSLik. A \verb|let| binding corresponds to introducing a new variable with an equality
constraint corresponding to the equality in the \verb|let| binding in the pure part of a \SuSLik{} assertion.

The \verb|if-then-else| construct corresponds to \SuSLik's C-like ternary operator. We can use these features together to implement a \verb|maximum| function:

\begin{lstlisting}[language=Pika]
%generate maximum [Sll] Int

maximum : List -> Int;
maximum (Nil) := 0;
maximum (Cons x xs) :=
  let i := maximum xs
  in
  if i < x
    then x
    else i;
\end{lstlisting}

\noindent
The \verb|let| binding is required here. This is because the recursive predicate
application must use an additional variable, beyond the argument and output
parameters, to store the intermediate result \verb|i| and this is introduced in
\Pika{} using a \verb|let|. It is possible to extend the translator to
automatically introduce these \verb|let| bindings, but we relegate this to future work.

Furthermore, since the \verb|let| is required and the variable introduced by the \verb|let| is used
in the \verb|if| condition, we cannot replace the \verb|if| with a guard. Guards can only occur
at the top-level of a function definition, so it could not occur after a \verb|let|.

\subsubsection{Using multiple layouts}

To show the interaction between multiple algebraic data types, we write a
function that follows the left branches of a binary tree and collects the values
stored in those nodes into a list. This example demonstrates a binary tree
algebraic data type and a layout that corresponds to it.

\begin{lstlisting}[language=Pika]
%generate leftList [TreeLayout] Sll

data Tree := Leaf | Node Int Tree Tree;

TreeLayout : Tree >-> layout[x];
TreeLayout (Leaf) := emp;
TreeLayout (Node payload left right) :=
  x :-> payload **
  (x+1) :-> left **
  (x+2) :-> right **
  TreeLayout left **
  TreeLayout right;

leftList : Tree -> List;
leftList (Leaf) := Nil;
leftList (Node a b c) := Cons a (leftList b);
\end{lstlisting}


\subsubsection{fold}
\label{sec:examples-fold}

A fold is a common kind of operation on a data structure in functional programming, where a binary function
is applied the elements of a data structure to create a summary value. For example, if the binary function
is the addition function, it will give the sum of all the elements of the data structure. The
classic example of such a fold is a fold on a list. In this example, we will write a
right fold over a \verb|List|.

We also demonstrate \verb|Ptr| types and the \verb|addr| builtin operation. This is somewhat similar to an \verb|@| pattern in Haskell,
but it is a low-level construct, indictating how pointers should be used, rather than a high-level construct.
Given a base type $\alpha$ (such as \verb|Int|), \verb|Ptr Int| is a type. This type is
passed around differently in the generated code. In particular, it is passed by reference.
\verb|addr| is the corresponding value-level introduction operation.
\begin{lstlisting}[language=Pika]
%generate fold_List [Int, Sll] (Ptr Int)

fold_List : Int -> List -> Ptr Int;
fold_List z (Nil) := z;
fold_List z (Cons x xs) :=
  instantiate
    [Ptr Int, Ptr Int]
    (Ptr Int)
    f
    (addr x)
    (fold_List z xs);
\end{lstlisting}

\noindent
The compiler produces the following \SuSLik{} specification for \verb|fold_List|:

\begin{lstlisting}[language=SynLang]
predicate fold_List(int i, loc p_x ,loc __r) {
| (p_x == 0) => { __r :-> i }
| (not (p_x == 0)) => {
  p_x :-> x ** (p_x+1) :-> xs ** [p_x,2] **
  func f(p_x, __p_2, __r) **
  fold_List(i, xs, __p_2) **
  temploc __p_2 }
}
\end{lstlisting}
%
In the generated code, \verb|f| is applied to \verb|p_x| rather than
\verb|x| as a result of using \verb|Ptr| and \verb|addr|.
