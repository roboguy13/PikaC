\section{Related Work}
\label{sec:related-work}

\tool is built upon the \suslik synthesis framework. \suslik provides
a synthesis mechanism for heap-manipulating programs using a variant
of separation logic.~\cite{polikarpova:2019:suslik} However, it does
not have any high-level abstractions. In particular, writing \suslik
specifications directly involves a significant amount of pointer
manipulation. Further, it does not provide abstraction over specific
memory layouts. As described in \autoref{sec:language}, \tool
addresses these limitations.

The \tname{Dargent} language~\cite{chen:2023:dargent} also includes a
notion of layouts and layout polymorphism for a class of algebraic
data types, which differs from our treatment of layouts in two primary
ways:

\begin{enumerate}

\item In \tool, abstract memory locations (with offsets) are used. In
  contrast, \tname{Dargent} uses offsets that are all
  relative to a single ``beginning'' memory location. The \tool
  approach is more amenable to heap allocation, though
  this requires a separate memory manager of some kind. This is
  exposed in the generated language with \verb|malloc| and
  \verb|free|.
  On the other hand, the technique taken by \tname{Dargent} allows
  for greater control over memory management. This makes
  dealing with memory more complex for the programmer, but it is no
  longer necessary to have a separate memory manager.

  \item Algebraic data types in the present language include
    \emph{recursive} types and,
    as a result, \tool has recursive layouts for these ADTs. This
    feature is not currently available in \tname{Dargent}.
  \end{enumerate}

Furthermore, layout polymorphism also works differently. While
\tname{Dargent} tracks layout instantiations at a type-level with type
variables, in the present work we simply only check to see if a layout
is valid for a given type when type-checking. In particular, we cannot
write type signatures that \textit{require} the same layout in
multiple parts of the type (for instance, in a function type
\verb|List -> List| we have no way at the type-level of requiring that
the argument \verb|List| layout and the result \verb|List| layout are
the same). \Pika{} 2 makes progress towards eliminating this restriction but, as of now, it is layout monomorphic. This more rudimentary approach that \tool currently takes
could be extended in future work.
%
Overall, the examples in the \tname{Dargent} paper tend to focus on
the manipulation of integer values. In contrast, we have focused
largely on data structure manipulation, which follow the primary
motivation of \suslik. \autoref{fig:dargent-comparison} has a comparison of some of the major features of \Pika{} and \tname{Dargent}.

\begin{figure}
  \begin{tabular}{|c|c|c|}
    \hline
    Feature & \Pika & \tname{Dargent}\\
    \hline
    Bit-level layouts & \xmark & \cmark\\
    Recursive layouts & \cmark & \xmark\\
    Synthesis from type signatures & \cmark & \xmark\\
    Custom record accessors & \xmark & \cmark\\
    \hline
  \end{tabular}
  \caption{Comparison to \tname{Dargent}}
  \label{fig:dargent-comparison}
\end{figure}

\tname{Synquid} is another synthesis framework with a functional
surface language. While \tname{Synquid} allows an even higher-level
program specification than \tool through its liquid types, it does not
provide any access to low-level data structure
representation.~\cite{polikarpova:2016:synquid} In contrast, \tool's
level of abstraction is similar to that of a traditional functional
language but, similar to \tname{Dargent}, it also allows control over
the data structure representation in memory. The \synth{} feature
has similarities to \tname{Synquid}'s approach to program synthesis, but
\Pika{} 2's \synth{} allows lower level constraints to be expressed.


