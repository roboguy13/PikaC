\section{Examples}
\label{sec:examples}

\subsection{\Pika{} 1}

\begin{figure}[b]
\setlength{\abovecaptionskip}{5pt}
\setlength{\belowcaptionskip}{-15pt}
\begin{center}
  \begin{table}[H]
  \begin{tabular}{ c c c c }
    \hline
    Name & \suslik size & \tool 1 size & \tool 1 size / \suslik size\\
    \hline
    \verb|append| & 164 & 84 & 0.512 \\
    \verb|cons| & 82 & 52 & 0.634 \\
    \verb|filterLt9| & 136 & 66 & 0.485 \\
    \verb|singleton| & 70 & 47 & 0.671 \\
    \verb|map| & 147 & 93 & 0.633 \\
    \verb|mapSum| & 138 & 97 & 0.703 \\
    \verb|maximum| & 106 & 61 & 0.575 \\
    \verb|leftList| & 144 & 104 & 0.722 \\
    \verb|take| & 180 & 112 & 0.622
  \end{tabular}
  \end{table}
\end{center}
  \caption{\tool 1 spec size vs generated SSL spec size measured in number of AST nodes}
  \label{fig:size-comparison}
\end{figure}

\subsection{\Pika{} 2}

Version 2 of \Pika{} is a rewrite of version 1 and it is missing a couple of features of version 1. However, it has the \synth{} feature. It also has the groundwork for layout polymorphism to be implemented. The second aspect, in particular, is a substantial change from version 1 and motivated the rewrite.

\begin{figure}
  \begin{tabular}{ c c }
    \hline
    Name & Uses \synth\\
    \hline
    \verb|add1Head| & No\\
    \verb|anagram| & No\\
    \verb|fact| & No\\
    \verb|filterLt| & No\\
    \verb|heap| & Yes\\
    \verb|leftList| & No\\
    \verb|mapAdd| & No\\
    \verb|maximum| & No\\
    \verb|reverse| & No\\
    \verb|set| & Yes\\
    \verb|sum| & No\\
    \verb|take| & No\\
    \verb|treeSize| & No\\
  \end{tabular}
  \caption{\Pika{} 2 examples}
  \label{fig:pika-2-examples}
\end{figure}                   
                               
                               
                               
                               
                               
                               
                               
                               
                               
                               
                               
                               
                               
                               
