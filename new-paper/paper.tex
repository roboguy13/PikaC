\documentclass[acmsmall]{acmart}
\usepackage{graphicx}
\usepackage{proof}
\usepackage{stmaryrd}
\usepackage{amsmath}
%\usepackage{amssymb}
\usepackage{xcolor}
\usepackage{tcolorbox}
\usepackage{tikz}
\usepackage{tikz-cd}
\usepackage{syntax}
\usepackage{listings}
\usepackage{xspace}
\usepackage{mathtools}
\usepackage{mathpartir}
\usepackage{pifont}
\usepackage{thmtools}
\lstset{basicstyle=\ttfamily, mathescape=true, literate={~} {$\sim$}{1}}

% From https://tex.stackexchange.com/questions/235118/making-a-thicker-cdot-for-dot-product-that-is-thinner-than-bullet
\makeatletter
\newcommand*\bigcdot{\mathpalette\bigcdot@{.5}}
\newcommand*\bigcdot@[2]{\mathbin{\vcenter{\hbox{\scalebox{#2}{$\m@th#1\bullet$}}}}}
\makeatother

\newcommand {\Pika} {\textsf{Pika}}
\newcommand {\PikaCore} {\textsf{PikaCore}}

\newcommand {\instExpr} {\keyword{inst}}

\newcommand {\labinfer} [3] [] {\infer[{\textsc{#1}}]{#2}{#3}}

% \newcommand {\keyword} [1] {\textsf{#1}}
\newcommand {\keyword} [1] {\textbf{#1}}

\newcommand {\Int} {\keyword{Int}}
\newcommand {\Bool} {\keyword{Bool}}
\newcommand {\Type} {\keyword{Type}}
\newcommand {\layout} [1] {\keyword{layout}(#1)}

\newcommand {\data} {\keyword{data}}

\newcommand {\inL} {\keyword{inL}}
\newcommand {\inR} {\keyword{inR}}
\newcommand {\fold} {\keyword{fold}}
\newcommand {\unfold} {\keyword{unfold}}
\newcommand {\fst} {\keyword{fst}}
\newcommand {\snd} {\keyword{snd}}
\newcommand {\either} {\keyword{either}}

\newcommand {\outs} {\longrightarrow}

\newcommand {\Pair} {\otimes}
\newcommand {\Either} {\oplus}
\newcommand {\RecTy} {\mu}

\newcommand {\monic} {\rightarrowtail}

\newcommand {\type} {\textsf{type}}
\newcommand {\ctype} {\textsf{ctype}}
\newcommand {\adt} {\textsf{adt}}

\newcommand {\LayoutDefs} {\mathcal{L}}

\newcommand {\matchWith} [1] {\keyword{case}\,#1\,\keyword{of}\,}
\newcommand {\layoutmatch} [1] {\keyword{layoutcase}\,#1\,\keyword{of}\,}

\newcommand {\withIn} [1] {\keyword{with}\,#1\,\keyword{in}\,}
\newcommand {\letPure} [1] {\keyword{letpure}\,#1\,\keyword{in}\,}

% Based on https://tex.stackexchange.com/questions/451786/how-do-i-put-a-circle-around-a-symbol
\makeatletter
\newcommand {\osep}{\mathbin{\mathpalette\make@circled\ast}}
\newcommand{\make@circled}[2]{%
  \ooalign{$\m@th#1\smallbigcirc{#1}$\cr\hidewidth$\m@th#1#2$\hidewidth\cr}%
}
\newcommand{\smallbigcirc}[1]{%
  \vcenter{\hbox{\scalebox{0.77778}{$\m@th#1\bigcirc$}}}%
}
\makeatother

% \newcommand {\lowerExpr} {\keyword{lower}}
% \newcommand {\liftExpr} {\keyword{lift}}
\newcommand {\instantiate} {\keyword{instantiate}}
\newcommand {\apply} {\keyword{apply}}

\newcommand {\liftExpr} {\keyword{lift}}

\newcommand {\tyApply} {\mathbin{@}}

\newcommand {\isA} {\sim}

\newcommand {\ra} {\rightarrow}
\newcommand {\Ra} {\Rightarrow}

\newcommand {\Dargent} {{\sc Dargent}}

\newcommand {\highlight} [1] {\begin{tcolorbox}[hbox] #1 \end{tcolorbox}}

\newcommand {\tyLambda} {\mathrm{\Lambda}}

\newcommand {\sem} [1] {\llbracket #1 \rrbracket}

\newcommand {\argCount} [1] {N\sem{#1}}
\newcommand {\namesem} [1] {\Var^{\argCount{#1}}}

\newcommand {\pcsubscript} {\textsf{PC}}
\newcommand {\step} {\longrightarrow}
\newcommand {\steps} {\longrightarrow^{*}}
\newcommand {\pcstep} {\step_{\pcsubscript}}
\newcommand {\pcsteps} {\steps_{\pcsubscript}}
\newcommand {\pcNotionStep} {\rightsquigarrow_{\pcsubscript}}

\newcommand {\FV} {\textsf{FV}}

\newcommand {\bigstep} {\Downarrow}
\newcommand {\pcStep} {\bigstep_{\pcsubscript}}

\newcommand {\defStep} {\bigstep_{\pcsubscript}}

\newcommand {\val} {\textsf{value}}

\newcommand {\fresh} [1] {#1 \textrm{ fresh}}

\newcommand {\recspec} [1] {\langle #1 \rangle}

\newcommand {\Layout} [1] {\keyword{Layout}(#1)}

\newcommand {\todo} [1] {\text{{\color{red}TODO}: #1}}

\newcommand {\dom} {\textsf{dom}}
\newcommand {\range} {\textsf{range}}

% \newcommand {\ssl} [2] {\keyword{ssl}(#1) \{ #2 \}}
% \newcommand {\ssl} [1] {\keyword{ssl} \{ #1 \}}
\newcommand {\sslmath} [3] {\ssl\{ #1 \}\{ #2 \}\{ #3 \}}
\newcommand {\ssl} {\keyword{layout}}
\newcommand {\SSL} {\keyword{SSL}}

\newcommand {\SuSLik} {SuSLik}

\newcommand {\Var} {\textsf{Var}}

\newcommand {\Heap} {\textsf{Heap}}

\newcommand {\lifted} [1] {#1_{\bot}}

\newcommand {\eq} {\stackrel{\bigcdot}{=}}
\newcommand {\defeq} {:=}
\newcommand {\ok} [1] {\;\textsf{ok}_{#1}}

\newcommand {\typesem} [1] {\mathcal{A}_{#1}}

\newcommand {\True} {\keyword{True}}
\newcommand {\False} {\keyword{False}}

\newcommand {\generate} {\keyword{generate}}

\newcommand {\vdashpc} {\vdash_{\textsf{PC}}}

\newcommand {\fboxNewlines} {\\\\\\}

\newcommand {\layoutParams} {\textsf{params}}

% From https://tex.stackexchange.com/questions/502652/define-tcolorbox-in-math-mode
\newtcbox{\boxtext}{on line,colback=white,colframe=black,size=fbox,arc=3pt,boxrule=0.8pt}
\newcommand{\boxmath}[1]{\boxtext{$#1$}}

% https://tex.stackexchange.com/questions/24132/overline-outside-of-math-mode
\makeatletter
\newcommand*{\textoverline}[1]{$\overline{\hbox{#1}}\m@th$}
\makeatother




%%
%% \BibTeX command to typeset BibTeX logo in the docs
\AtBeginDocument{%
  \providecommand\BibTeX{{%
    Bib\TeX}}}

%% Rights management information.  This information is sent to you
%% when you complete the rights form.  These commands have SAMPLE
%% values in them; it is your responsibility as an author to replace
%% the commands and values with those provided to you when you
%% complete the rights form.
\setcopyright{acmcopyright}
\copyrightyear{2018}
\acmYear{2018}
\acmDOI{XXXXXXX.XXXXXXX}


%%
%% These commands are for a JOURNAL article.
\acmJournal{JACM}
\acmVolume{37}
\acmNumber{4}
\acmArticle{111}
\acmMonth{8}

%%
%% Submission ID.
%% Use this when submitting an article to a sponsored event. You'll
%% receive a unique submission ID from the organizers
%% of the event, and this ID should be used as the parameter to this command.
%%\acmSubmissionID{123-A56-BU3}

%%
%% For managing citations, it is recommended to use bibliography
%% files in BibTeX format.
%%
%% You can then either use BibTeX with the ACM-Reference-Format style,
%% or BibLaTeX with the acmnumeric or acmauthoryear sytles, that include
%% support for advanced citation of software artefact from the
%% biblatex-software package, also separately available on CTAN.
%%
%% Look at the sample-*-biblatex.tex files for templates showcasing
%% the biblatex styles.
%%

%%
%% The majority of ACM publications use numbered citations and
%% references.  The command \citestyle{authoryear} switches to the
%% "author year" style.
%%
%% If you are preparing content for an event
%% sponsored by ACM SIGGRAPH, you must use the "author year" style of
%% citations and references.
%% Uncommenting
%% the next command will enable that style.
%%\citestyle{acmauthoryear}

%%
%% end of the preamble, start of the body of the document source.
\begin{document}

%%
%% The "title" command has an optional parameter,
%% allowing the author to define a "short title" to be used in page headers.
\title{Compilation With Program Synthesis}

% %%
% %% The "author" command and its associated commands are used to define
% %% the authors and their affiliations.
% %% Of note is the shared affiliation of the first two authors, and the
% %% "authornote" and "authornotemark" commands
% %% used to denote shared contribution to the research.
% \author{Ben Trovato}
% \authornote{Both authors contributed equally to this research.}
% \email{trovato@corporation.com}
% \orcid{1234-5678-9012}
% \author{G.K.M. Tobin}
% \authornotemark[1]
% \email{webmaster@marysville-ohio.com}
% \affiliation{%
%   \institution{Institute for Clarity in Documentation}
%   \streetaddress{P.O. Box 1212}
%   \city{Dublin}
%   \state{Ohio}
%   \country{USA}
%   \postcode{43017-6221}
% }
%
% \author{Lars Th{\o}rv{\"a}ld}
% \affiliation{%
%   \institution{The Th{\o}rv{\"a}ld Group}
%   \streetaddress{1 Th{\o}rv{\"a}ld Circle}
%   \city{Hekla}
%   \country{Iceland}}
% \email{larst@affiliation.org}
%
% \author{Valerie B\'eranger}
% \affiliation{%
%   \institution{Inria Paris-Rocquencourt}
%   \city{Rocquencourt}
%   \country{France}
% }
%
% \author{Aparna Patel}
% \affiliation{%
%  \institution{Rajiv Gandhi University}
%  \streetaddress{Rono-Hills}
%  \city{Doimukh}
%  \state{Arunachal Pradesh}
%  \country{India}}
%
% \author{Huifen Chan}
% \affiliation{%
%   \institution{Tsinghua University}
%   \streetaddress{30 Shuangqing Rd}
%   \city{Haidian Qu}
%   \state{Beijing Shi}
%   \country{China}}
%
% \author{Charles Palmer}
% \affiliation{%
%   \institution{Palmer Research Laboratories}
%   \streetaddress{8600 Datapoint Drive}
%   \city{San Antonio}
%   \state{Texas}
%   \country{USA}
%   \postcode{78229}}
% \email{cpalmer@prl.com}
%
% \author{John Smith}
% \affiliation{%
%   \institution{The Th{\o}rv{\"a}ld Group}
%   \streetaddress{1 Th{\o}rv{\"a}ld Circle}
%   \city{Hekla}
%   \country{Iceland}}
% \email{jsmith@affiliation.org}
%
% \author{Julius P. Kumquat}
% \affiliation{%
%   \institution{The Kumquat Consortium}
%   \city{New York}
%   \country{USA}}
% \email{jpkumquat@consortium.net}
%
% %%
% %% By default, the full list of authors will be used in the page
% %% headers. Often, this list is too long, and will overlap
% %% other information printed in the page headers. This command allows
% %% the author to define a more concise list
% %% of authors' names for this purpose.
% \renewcommand{\shortauthors}{Trovato et al.}

%%
%% The abstract is a short summary of the work to be presented in the
%% article.
\begin{abstract}
  Synthetic Separation Logic (SSL) is a specification language used by the \SuSLik{} program synthesis tool.
  \SuSLik{} will take a Hoare-style specification expressed in SSL and synthesize a program that satisfies this
  specification in a C-like language. However, these specifications are very low-level and SSL does not allow for much abstraction. In this paper,
  we explore a connection between SSL specifications and functional programming. We give a functional
  programming language, \Pika, which has both a standard semantics and a translation into SSL that respects the standard semantics. This translation also makes interfacing between SSL and \Pika{} easy. We introduce ``layouts'' to bridge the gap between functional programming and SSL. This allows us to describe the memory layout of algebraic data types as SSL predicates inside \Pika.

  We then develop a second version of \Pika{} extended with an additional synthesis feature. This synthesis feature allows SSL propositions to be included in type signatures. Then, an implementation is synthesised from the type signature using \SuSLik. These two versions of \Pika{} are compared.

  % Along the way, we encounter limitations in the \SuSLik{} tool. In confronting these limitations, we develop a new feature
  % in \Pika{} that allows the programmer to express synthesis tasks at a higher-level.

  % A clear and well-documented \LaTeX\ document is presented as an
  % article formatted for publication by ACM in a conference proceedings
  % or journal publication. Based on the ``acmart'' document class, this
  % article presents and explains many of the common variations, as well
  % as many of the formatting elements an author may use in the
  % preparation of the documentation of their work.
\end{abstract}

%%
%% The code below is generated by the tool at http://dl.acm.org/ccs.cfm.
%% Please copy and paste the code instead of the example below.
%%
\begin{CCSXML}
% <ccs2012>
%  <concept>
%   <concept_id>00000000.0000000.0000000</concept_id>
%   <concept_desc>Do Not Use This Code, Generate the Correct Terms for Your Paper</concept_desc>
%   <concept_significance>500</concept_significance>
%  </concept>
%  <concept>
%   <concept_id>00000000.00000000.00000000</concept_id>
%   <concept_desc>Do Not Use This Code, Generate the Correct Terms for Your Paper</concept_desc>
%   <concept_significance>300</concept_significance>
%  </concept>
%  <concept>
%   <concept_id>00000000.00000000.00000000</concept_id>
%   <concept_desc>Do Not Use This Code, Generate the Correct Terms for Your Paper</concept_desc>
%   <concept_significance>100</concept_significance>
%  </concept>
%  <concept>
%   <concept_id>00000000.00000000.00000000</concept_id>
%   <concept_desc>Do Not Use This Code, Generate the Correct Terms for Your Paper</concept_desc>
%   <concept_significance>100</concept_significance>
%  </concept>
% </ccs2012>
\end{CCSXML}

% \ccsdesc[500]{Do Not Use This Code~Generate the Correct Terms for Your Paper}
% \ccsdesc[300]{Do Not Use This Code~Generate the Correct Terms for Your Paper}
% \ccsdesc{Do Not Use This Code~Generate the Correct Terms for Your Paper}
% \ccsdesc[100]{Do Not Use This Code~Generate the Correct Terms for Your Paper}

% %%
% %% Keywords. The author(s) should pick words that accurately describe
% %% the work being presented. Separate the keywords with commas.
% \keywords{Do, Not, Us, This, Code, Put, the, Correct, Terms, for,
%   Your, Paper}

% \received{20 February 2007}
% \received[revised]{12 March 2009}
% \received[accepted]{5 June 2009}

%%
%% This command processes the author and affiliation and title
%% information and builds the first part of the formatted document.
\maketitle

\newcommand{\ruleSAdd} {
  \labinfer[S-Add]{(e_1 + e_2, \VarSet_0) \tstep (v == v_1 + v_2 \land p_1 \land p_2, s_1 \sep s_2, \VarSet_2 \cup \{v\}, v) }
    {(e_1, \VarSet_0) \tstep (p_1, s_2, \VarSet_1, v_1) &
     (e_2, \VarSet_1) \tstep (p_2, s_2, \VarSet_2, v_2) &
     \freshVar{v}
    }}



\section{Introduction}

Synthetic separation logic (SSL) is the specification language for the \SuSLik{} program synthesiser. \SuSLik{} will take a precondition and postcondition pair together
with a function prototype and use these to synthesise a function that satisfies that specification in a C-like language called \SuSLang.

We have developed a  functional programming language, \Pika, which will compile to SSL specifications. \SuSLik{} can then take these SSL specifications and generate \SuSLang. Finally, the resulting \SuSLang{} code can be converted to C and then run. In addition to this translational semantics, we also provide a standard abstract machine semantics for \Pika{} and use this to show correctness of the translation into SSL.

Two versions of \Pika{} have been built. \Pika{} 1 has a focus on using \SuSLik{} to generate all of the low-level code, while \Pika{} 2 adds an additional synthesis feature inside the \Pika{} language itself and has a builtin testing framework.

%In the present work, we will start by talking about \Pika{} 1, describe the fundamental limitations of using \SuSLik{} to generate everything and then use this to motivate the creation of \Pika{} 2. The two versions are then compared.

\paragraph{Contributions} In this paper, we make the following contributions:
\begin{itemize}
  \item We explore the connection between SSL and functional programming using a pair of formal semantics for \Pika: A standard abstract machine semantics and a
    semantics that translates \Pika{} into SSL specifications. The second semantics is shown to be sound relative to the first semantics. (Section~\ref{sec:semantics})

  \item In order to provide seamless interoperability between SSL and \Pika, we provide a mechanism to related algebraic data types to their representation in memory. In particular, Pika has a language construct that allows the programmer to define a kind of function from values of an algebraic data type to SSL assertions. This is called a \textit{layout}. (Section~\ref{sec:layouts})

  \item We develop an extension of \Pika{} that allows \SuSLik's synthesis capabilities to be exposed more naturally inside the \Pika{} language. (Section~\ref{sec:synth})

  \item A comparison of examples that work on the first version of \Pika{} is compared with examples that are enabled by the new feature in the second version of \Pika. This new version of \Pika{} is not backwards compatible with the first version, so we also examine which examples in the old version no longer work in the new version. (Section~\ref{sec:examples})
\end{itemize}



\section{Overview}
\label{sec:overview}

\subsection{Background}
The \Pika{} language is translated into
\SuSLik~\cite{polikarpova:2019:suslik}, which is a program synthesis
tool that uses Synthetic Separation Logic (SSL)-a variant of
Hoare-style~\cite{hoare:1969:axiomatic} Separation Logic
(SL)~\cite{ohearn:2001:seplogic}.

% SSL is a variant of separation
% logic that works particularly well for program
% synthesis.~\cite{polikarpova:2019:suslik}

A synthesis task specification in \SuSLik{} is given as a function
signature together with a pair of pre- and post-conditions, which are
both SL assertions~\cite{polikarpova:2019:suslik}.
%
The synthesiser generates code that satisfies the given specification,
along with the SL \emph{proof} of its correctnees by performing the
search in a space of proofs that can be derived by using the rules of
the underlying logic~\cite{WatanabeGPPS21}.
%
A distinguishing feature of \textit{Synthetic} Separation Logic is the
format of its assertions. An SSL assertion consists of two parts: a
\textit{pure part} and a \textit{spatial part}. The pure part is a
Boolean expression constraining the variables of the specification
using a few basic relations, like equality and the less-than relation
for integers. The spatial part is a \textit{symbolic heap}. This
consists of a list of heaplets. The list separator for the list of
heaplets is the $\sep$ symbol. Each heaplet takes on one of the
following forms~\cite{polikarpova:2019:suslik}:

\begin{itemize}
  \item \texttt{emp}: This represents the empty heap. It is also the left and right identity for $\sep$.
  \item $\ell \mapsto a$: This asserts that memory location $\ell$ points to the value $a$.
  \item $[\ell, \iota]$: At memory location $\ell$, there is a block of size $\iota$.
  \item $p(\overline{\phi})$: This is an application of the
    \textbf{\textit{inductive predicate}} $p$ to the arguments
    $\overline{\phi}$. An inductive predicate has a collection of
    branches, guarded by Boolean expressions (conditions) on the
    parameters. The body of each branch is an SSL assertion. The
    assertion associated with the first branch condition that is
    satisfied is used in place of the application. Note that inductive
    predicates can be (and often are) recursively defined.
\end{itemize}

\noindent
The general form of an SSL assertion is
$(p; h_1 \sep h_2 \sep \cdots \sep h_n)$, where $p$ is the pure part
and $h_1, h_2, \cdots, h_n$ are the heaplets which are the conjuncts
that make up the separating conjunction of the spatial part. A syntax
definition for SSL is given in \autoref{fig:SSL-syntax}, which is
adapted from \textit{Cyclic Program Synthesis} by Itzhaky
\etal~\cite{itzhaky:2021:cyclic-synth}.

\begin{figure}[t]
\setlength{\abovecaptionskip}{5pt}
%\setlength{\belowcaptionskip}{-15pt}
{\small{
\[
  \begin{array}[t]{ l l }
    \text{Variable} &x, y\;\; \text{Alpha-numeric identifiers $\in$ \textnormal{Var}}\\
    \text{Size, offset} &n, \iota\;\; \text{Non-negative integers}\\
    \text{Expression} &e ::= 0\; \mid \texttt{true} \mid x \mid e = e \mid e \land e \mid \neg e \mid d\\
    \text{$\mathcal{T}$-expr.} &d ::= n \mid x \mid d + d \mid n \cdot d \mid \{\} \mid {d} \mid \cdots\\
    \text{Command} &c ::= \begin{aligned}[t] &\texttt{let x = *(x + }\iota\texttt{)} \mid \texttt{*(x + }\iota\texttt{) = e} \mid
      \\
      &\texttt{let x = malloc(n)} \mid \texttt{free(x)} \mid \texttt{error}\newline
      \mid \texttt{f(}\overline{e_i}\texttt{)}
      \end{aligned}\\
    \text{Program} &\Pi ::= \overline{f(\overline{x_i})\; \{\; c\; \}\; ;}\; c\\
    \text{Logical variable} &\nu, \omega\\
    \text{Cardinality variable} &\alpha\\
    \text{$\mathcal{T}$-term} &\kappa ::= \nu \mid e \mid \cdots\\
    \text{Pure logic term} &\phi, \psi, \chi ::= \kappa \mid \phi = \phi \mid \phi \land \phi \mid \neg \phi\\
    \text{Symbolic heap} &P, Q, R ::= \texttt{emp} \mid \mbox{$\langle e, \iota \rangle \mapsto e \mid [e, \iota]$} \mid p^{\alpha}(\overline{\phi_i})
      \mid \mbox{$P * Q$}\\
    \text{Heap predicate} &\mathcal{D} ::= p^{\alpha}(\overline{x_i}) : \overline{e_j \Ra \exists \overline{y}.\{\chi_j;R_j\}}\\
    \text{Assertion} &\mathcal{P},\mathcal{Q} ::= \{\phi; P\}\\
    \text{Environment} &\Gamma ::= \forall\overline{x_i}.\exists\overline{y_j}.\\
    \text{Context} &\Sigma ::= \overline{\mathcal{D}}\\
    \text{Synthesis goal} &\mathcal{G} ::= P \leadsto Q
  \end{array}
\]
}}
\caption{Syntax of Synthetic Separation Logic}
  \label{fig:SSL-syntax}
\end{figure}

\begin{figure}
  \ruleSAdd
  \caption{{\sc S-Add} rule}
  \label{fig:ruleSAdd}
\end{figure}

As the specification language of \SuSLik, SSL serves as the
compilation target for the \Pika{} language. From there, executable
programs are generated through \SuSLik's program synthesis. Consider a
program that takes an integer $x$ and a result in location $r$ and
stores $x+1$ at location $r$. This can be written as the \SuSLik{}
specification:

\begin{lstlisting}
void add1Proc(int x, loc r)
  { r :-> 0 }
  { y == x+1 ; r :-> y }
{ ?? }
\end{lstlisting}

\noindent
In contrast, this example can be written as follows in \Pika:

\begin{lstlisting}
add1Proc : Int -> Int;
add1Proc x := x + 1;
\end{lstlisting}

\noindent
In constrast with \SuSLik{} specs, the one below features no direct
manipulation with pointers.
%
This program is translated using the {\sc S-Add} rule shown in
\autoref{fig:ruleSAdd}. 

\subsection{The \Pika{} Language}
\label{sec:language}

While SSL provides a specification language that allows tools like
\SuSLik{} to synthesise code, it is only able to express specifications
as pointers. This is useful for some applications, such as embedded
systems, but it does not provide any high-level abstractions. As a
result, every part of a specification is tailored to a specific memory
representation of each data structure involved.
%
To addresss this shortcoming, we introduce a language with algebraic
data types that gets translated into SSL specifications. Additionally,
we introduce a language construct that allows the programmer to
specify a memory representation of an algebraic data type. This is
called a \textit{layout}. This distinction between algebraic data
types and layouts allows the separation of concerns between the low
level representation of a data structure and code that manipulates it
at a high level.

Syntactically, \Pika{} resembles a functional programming language
of the Miranda~\cite{turner:1986:miranda} and
Haskell~\cite{hudak:2007:haskell} lineage. It supports algebraic data
types, pattern matching at top-level function definitions (though not
inside expressions) and Boolean guards. The primary difference arises
due to the existence of layouts and the fact that the language is
compiled to an SSL specification rather than executable code.

Functions in \Pika{} are only defined by their operations on algebraic
data types. Thus, all function definitions are ``layout polymorphic''
over the particular choices of layouts for their arguments and result.
Giving a layout-polymorphic function, a particular choice of layouts
is called ``instantiation''. Specifying the layout of a non-function
value is called ``lowering.''

There is a subtlety here: The layout of a subexpression inside of a
function definition might not be determined by the layouts that the
function definition is instantiated at. In this case, we must
explicitly specify a layout. This is done using the \verb|instantiate|
and \verb|lower| constructs. The former is for function calls, while
the latter is for all other types of values. These operations are
performed with the \verb|instantiate| and \verb|lower| language
constructs, respectively. For example, \verb|instantiate [A] B f x|
calls $f$ with the argument $x$, where the layout $A$ is used for the
argument and the layout $B$ is used for the result of the function
application. This is also written $\instantiateS{A,B}{f}(x)$. The
latter is primarily used in \autoref{sec:semantics}.

The code generator is instructed to generate a \SuSLik{} specification
for a certain function at a certain instantiation by using a
\generate{} directive. For example, if there is a function definition
with the type signature \verb|mapAdd1 : List -> List|, a line reading
\verb|%generate mapAdd1 [Sll] Sll| would instruct the
\Pika{} compiler to generate the \SuSLik{} inductive predicate
corresponding to \verb|mapAdd1| instantiated to the \verb|Sll| layout
for both its argument and its result.

An example of an ADT definition and a corresponding layout definition
is given in \autoref{fig:List-def}. There is one unusual part of
the syntax in particular which requires further explanation: layout
type signatures. A layout definition consists a \textit{layout type
  signature} and a pattern match (much like a function definition),
with lists of SSL heaplets on the right-hand sides. A layout type
signature has a special form $A : \alpha \monic \layout[x]$. This says
that the layout $A$ is for the algebraic data type $\alpha$ and the
SSL output variable is named $x$.
% The full \Pika{} grammar is given in
% the Appendices. %x~\ref{sec:grammar}.

\begin{figure}[t]
\begin{lstlisting}
data List := Nil | Cons Int List;

Sll : List >-> layout[x];
Sll (Nil) := emp;
Sll (Cons head tail) := x :-> head, (x+1) :-> tail, Sll tail;
\end{lstlisting}
  \caption{\lstinline{List} algebraic data type together with its singly-linked list layout \lstinline{Sll}}
  \label{fig:List-def}
\end{figure}

\subsection{\Pika{} by Example}

We demonstrate the characteristic usages of \Pika{} on a series of
examples.
%
In these examples, we will often make use of the \verb|List| algebraic
data type and its \verb|Sll| layout from~\autoref{fig:List-def}. A
simple example of \Pika{} code that illustrates algebraic data types and
layouts is a function which creates a singleton list out of the given
integer argument:

\begin{lstlisting}
%generate singleton [Int] Sll

singleton : Int -> List;
singleton x := Cons x (Nil);
\end{lstlisting}

\noindent
This gets compiled to the following \SuSLik{} specification, with generated names simplified for greater clarity:

\begin{lstlisting}
predicate singleton(int p, loc r) {
| true => { r :-> p ** (r+1) :-> 0 ** [r,2] }
}
\end{lstlisting}

\noindent
A slightly more complicated example comes from trying to write a
functional-style \verb|map| function directly in \SuSLik. Consider a
function which adds 1 to each integer in a list of integers.
Considering the list implementation to be a singly-linked list with a
fixed layout, one way to express this in \SuSLik{} is shown in \autoref{fig:sllx}

\begin{figure}[t]
\begin{lstlisting}
predicate sll(loc x) {
| x == 0 => { emp }
| not (x == 0) => { [x, 2] ** x :-> v ** (x+1) :-> nxt ** sll(nxt) }
}

predicate mapAdd1(loc x, loc r) {
| x == 0 => { emp }
| not (x == 0) => {
    [x, 2] ** x :-> v ** (x+1) :-> xNxt **
    [r, 2] ** r :-> (v+1) ** (r+1) :-> rNxt **
    mapAdd1(xNxt, rNxt)
  }
}

void mapAdd1_fn(loc x, loc y)
  { sll(x) ** y :-> 0 }
  { y :-> r ** mapAdd1(x, r) }
{ ?? }
\end{lstlisting}
\caption{Specifying a function that adds one to each element of a
  singly-linked list in \SuSLik.}
\label{fig:sllx}
\end{figure}

\noindent
Note that inductive predicates are used for \textit{two} different purposes: the \verb|sll| inductive
predicate describes a singly-linked list data structure, while the \verb|mapAdd1| inductive predicate 
describes how the input list relates to the output list. Both are used in the specification of
\verb|mapAdd1_fn|: \verb|sll| in the precondition and \verb|mapAdd1| in the postcondition.

Using the \verb|mapAdd1| inductive predicate gives us two advantages over attempting to
put the SSL propositions directly into the postcondition of \verb|mapAdd1_fn|:
%
\begin{enumerate}
  \item We are able to express a conditional on the shape of the list. This is much like pattern
    matching in a language with algebraic data types, but we are examining the pointer involved directly.
  \item We are able to express \textit{recursion} part of the postcondition: the \verb|mapAdd1| inductive
    predicate refers to itself in the \verb|not (x == 0)| branch.
\end{enumerate}
%
These two features are both reminiscent of features common in
functional programming languages: pattern matching and recursion.
However, there are still some significant differences:

\begin{itemize}
  \item In traditional pattern matching, the underlying memory representation of the data structure is not exposed.
  \item Compared to a functional programming language, the meaning of the specification is more obscured. It is
    necessary to think about the structure of the linked data structure to determine what the specification is saying. This
    is related to the first point: The memory representation is front-and-center.
  \item In many functional languages, mutation is either restricted or generally discouraged. In \SuSLik, mutation is commonplace.
\end{itemize}

Say we want to write the functional program that corresponds to this specification. One way to do this in a Haskell-like language
is the following, using the \verb|List| type from \autoref{fig:List-def}.

\begin{lstlisting}
mapAdd1_fn : List -> List;
mapAdd1_fn (Nil) := 0;
mapAdd1_fn (Cons head tail) := Cons (head + 1) (mapAdd1_fn tail);
\end{lstlisting}

\noindent
The only missing information is the memory representation of the \verb|List| data structure. We do not want the
\verb|mapAdd1_fn| implementation to deal with this directly, however. We want to separate the more abstract notions
of pattern matching and constructors from the concrete memory layout that the data structure has.

To accomplish this, we now extend the code with the definition of
\verb|Sll| from \autoref{fig:List-def}. \verb|Sll| is a
\textit{layout} for the algebraic data type \verb|List|.
%
Now we have all of the information of the original specification but
rearranged so that the low-level memory layout is separated from the
rest of the code. This separation brings us to an important
observation about the language, manifested throughout these examples:
none of the function definitions need to \emph{directly} perform any
pointer manipulations. This is relegated entirely to the reusable
layout definitions for the ADTs. The examples are written entirely as
recursive functions that pattern match on, and construct, ADTs.

All that is left is to connect these two parts: the layouts and the
function definitions. We instruct a \SuSLik{} specification generator to
generate a \SuSLik{} specification from the \verb|mapAdd1_fn| function
using the \verb|Sll| layout:
%
\begin{lstlisting}
%generate mapAdd1_fn [Sll] Sll
\end{lstlisting}
%
The \verb|[Sll]| part of the directive tells the generator which
layouts are used for the arguments. In this case, the function only
has one argument and the \verb|Sll| layout is used. The \verb|Sll| at
the end specifies the layout for the result.

\subsubsection{map}

We can generalise our \verb|mapAdd1| to map arbitrary \verb|Int|
functions over a list and then redefine \verb|mapAdd1| using the new
\verb|map|.

\begin{lstlisting}
%generate mapAdd1 [Sll] Sll

data List := Nil | Cons Int List;

Sll : List >-> layout[x];
Sll (Nil) := emp;
Sll (Cons head tail) := x :-> head, (x+1) :-> tail, Sll tail;

map : (Int -> Int) -> List -> List;
map f (Nil) := Nil;
map f (Cons x xs) := Cons (instantiate [Int] Int f x) (map f xs);

add1 : Int -> Int;
add1 x := x + 1;

mapAdd1 : List -> List;
mapAdd1 xs :=
  instantiate [Int -> Int, Sll] Sll map add1 xs;
\end{lstlisting}

This example makes use of \lstinline{instantiate} in two places. In the first case where we have the call
\lstinline{instantiate [Int] Int f x}, the builtin \verb|Int| layout is used for both the input
and output. In this special case, the \verb|Int| layout shares a name with the \verb|Int| type
that it represents. This is necessary since \lstinline{instantiate} is used for all non-recursive calls that are not constructor applications.

In the second use, \lstinline{instantiate [Int -> Int, Sll] Sll map add1 xs}, we specify that
the second argument uses the \verb|Sll| layout for the \verb|List| type from \autoref{fig:List-def}. We also give \verb|Sll| as
the layout for the result of the call. Note that it is not necessary to use \verb|instantiate| for the recursive call to \verb|map|. This is because the appropriate layout
is inferred for recursive calls.

It might be surprising that \verb|instantiate| is required in the body of \verb|mapAdd1| since the type signature
of \verb|mapAdd1| suggests that it is layout polymorphic and yet we must pick a specific \verb|List| layout when
we use \verb|instantiate| to call \verb|map|. This is because, in general, a call inside the body of some function \verb|fn| might use
any layout, even layouts that have no relation to the layouts that \verb|fn| is instantiated to.

It is possible to do inference of some layouts, for example in
\verb|mapAdd1| we would usually want to use the same layout as the
argument layout, but we leave this for future work. Another approach
is to introduce type variables that correspond to layouts, as done in
the series of works on the \tname{Dargent}{}
tool~\cite{chen:2023:dargent}.
%
We leave this approach for future work as well.

\subsubsection{Guards}

While we have a pattern-matching construct at the top level of a function definition, we
have not seen a way to branch on a Boolean value so far. This is a feature that is
readily available at the level of \SuSLik, since the same conditional construct we
use to implement pattern matching can also use other Boolean expressions.

We can expose this in the functional language using a \textit{guard},
much like Haskell's guards. Say, we want to write a specialised
filter-like function. Specifically, we want a function that filters
out all elements of a list that are less than 9. This is a specific
example where the \SuSLik{} specification is noticeably more difficult
to read. For a \SuSLik{} specification of this example, see
\autoref{fig:SuSLik-filter}

\begin{figure}
  \begin{lstlisting}
predicate filterLt9(loc x, loc r) {
| (x == 0) => { r == 0 ; emp }
| not (x == 0) && head < 9 =>
    { x :-> head ** (x+1) :-> tail ** [x,2] ** filterLt9(tail, r) }
| not (x == 0) && not (head < 9) =>
    { x :-> head ** (x+1) :-> tail ** [x,2] ** filterLt9(tail, y)
      ** r :-> head ** (r+1) :-> y ** [r,2] }
}

void filterLt9(loc x1, loc r)
  { Sll(x1) ** r :-> 0 }
  { filterLt9(x1, r0) ** r :-> r0 }
{ ?? }
  \end{lstlisting}
  \caption{\SuSLik{} specification of \lstinline{filterLt9}, excluding \lstinline{Sll} which is given in \autoref{fig:List-def}}
  \label{fig:suslik-filter}
\end{figure}

On the other hand, an implementation of this in \Pika{} is:

\begin{lstlisting}
%generate filterLt9 [Sll] Sll

filterLt9 : List -> List;
filterLt9 (Nil) := Nil;
filterLt9 (Cons head tail)
  | head < 9       := filterLt9 tail;
  | not (head < 9) := Cons head (filterLt9 tail);
\end{lstlisting}

\noindent
When translating a guarded function body, the translator takes the conjunction of the Boolean guard condition with
the condition for the pattern match.

\subsubsection{\texttt{let} bindings and \texttt{if-then-else}}

Two more features that are common in functional languages are \verb|let| bindings and
\verb|if-then-else| expressions. Both of these have straightforward translations into
\SuSLik. A \verb|let| binding corresponds to introducing a new variable with an equality
constraint corresponding to the equality in the \verb|let| binding in the pure part of a \SuSLik{} assertion.

The \verb|if-then-else| construct corresponds to \SuSLik's C-like ternary operator. We can use these features together to implement a \verb|maximum| function:

\begin{lstlisting}
%generate maximum [Sll] Int

maximum : List -> Int;
maximum (Nil) := 0;
maximum (Cons x xs) :=
  let i := maximum xs
  in
  if i < x
    then x
    else i;
\end{lstlisting}

\noindent
The \verb|let| binding is required here. This is because the recursive predicate
application must use an additional variable, beyond the argument and output
parameters, to store the intermediate result \verb|i| and this is introduced in
\Pika{} using a \verb|let|. It is possible to extend the translator to
automatically introduce these \verb|let| bindings, but we relegate this to future work.

Furthermore, since the \verb|let| is required and the variable introduced by the \verb|let| is used
in the \verb|if| condition, we cannot replace the \verb|if| with a guard. Guards can only occur
at the top-level of a function definition, so it could not occur after a \verb|let|.

\subsubsection{Using multiple layouts}

To show the interaction between multiple algebraic data types, we write a
function that follows the left branches of a binary tree and collects the values
stored in those nodes into a list. This example demonstrates a binary tree
algebraic data type and a layout that corresponds to it.

\begin{lstlisting}
%generate leftList [TreeLayout] Sll

data Tree := Leaf | Node Int Tree Tree;

TreeLayout : Tree >-> layout[x];
TreeLayout (Leaf) := emp;
TreeLayout (Node payload left right) :=
  x :-> payload,
  (x+1) :-> left,
  (x+2) :-> right,
  TreeLayout left,
  TreeLayout right;

leftList : Tree -> List;
leftList (Leaf) := Nil;
leftList (Node a b c) := Cons a (leftList b);
\end{lstlisting}


\subsubsection{fold}
\label{sec:examples-fold}

A fold is a common kind of operation on a data structure in functional programming, where a binary function
is applied the elements of a data structure to create a summary value. For example, if the binary function
is the addition function, it will give the sum of all the elements of the data structure. The
classic example of such a fold is a fold on a list. In this example, we will write a
right fold over a \verb|List|.

We also demonstrate \verb|Ptr| types and the \verb|addr| builtin operation. This is somewhat similar to an \verb|@| pattern in Haskell,
but it is a low-level construct, indictating how pointers should be used, rather than a high-level construct.
Given a base type $\alpha$ (such as \verb|Int|), \verb|Ptr Int| is a type. This type is
passed around differently in the generated code. In particular, it is passed by reference.
\verb|addr| is the corresponding value-level introduction operation.
\begin{lstlisting}
%generate fold_List [Int, Sll] (Ptr Int)

fold_List : Int -> List -> Ptr Int;
fold_List z (Nil) := z;
fold_List z (Cons x xs) :=
  instantiate
    [Ptr Int, Ptr Int]
    (Ptr Int)
    f
    (addr x)
    (fold_List z xs);
\end{lstlisting}

\noindent
The compiler produces the following \SuSLik{} specification for \verb|fold_List|:

\begin{lstlisting}
predicate fold_List(int i, loc p_x ,loc __r) {
| (p_x == 0) => { __r :-> i }
| (not (p_x == 0)) => {
  p_x :-> x ** (p_x+1) :-> xs ** [p_x,2] **
  func f(p_x, __p_2, __r) **
  fold_List(i, xs, __p_2) **
  temploc __p_2 }
}
\end{lstlisting}
%
In the generated code, \verb|f| is applied to \verb|p_x| rather than
\verb|x| as a result of using \verb|Ptr| and \verb|addr|.

% \subsection{Syntax}
\label{sec:syntax}


\section{Semantics}

\subsection{Translation From \Pika{} to \PikaCore}

% Note that the lambda case of the $\val$ judgment is unusual. It requires
% the body of the lambda to be reduced.

\[
  \begin{array}{c}
    \fbox{$e ~\val$}
    \fboxNewlines
    \labinfer{i ~\val}{i \in \mathbb{Z}}
    ~~~
    \labinfer{b ~\val}{b \in \mathbb{B}}
    ~~~
    \labinfer{v ~\val}{v \in \Var}
    \\\\
    \labinfer{\sslmath{\overline{x}}{T}{\overline{\ell_i \mapsto e_i}} ~\val}
      {e_i ~\val \;\textrm{ for each $e_i$}}
    \\\\
    \labinfer{\lambda x.\; e ~\val}{e ~\val}
    ~~~
    \labinfer{\withIn{\{\overline{x} := e_1\}} e_2 ~\val}
      {e_1 ~\val & e_2 ~\val}
  \end{array}
\]
\\

\[
  \begin{aligned}
  \mathcal{E}
    ::=\; &[]\\
    \mid\; &\apply_A(\mathcal{E})\\
    \mid\; &\mathcal{E}\; e\\
    \mid\; &v\; \mathcal{E}\\
    \mid\; &\withIn{\{ \overline{x} \} := \mathcal{E}} e\\
    \mid\; &\withIn{\{ \overline{x} \} := v} \mathcal{E}\\
    \mid\; &\sslmath{\overline{x}}{T}{\overline{\ell \mapsto v}, \ell \mapsto \mathcal{E}, \overline{\ell \mapsto e}}\\
    \mid\; &\lambda x.\; \mathcal{E}
  \end{aligned}
\]

\[
  \begin{array}{c}
    \fbox{$e \pcstep e$}
    \fboxNewlines
    \labinfer{\mathcal{E}[e] \pcstep \mathcal{E}[e']}
      {e \pcNotionStep e'}
  \end{array}
\]

Where $\mathcal{L} \sqcup \mathcal{L}'$ is the language union that combines common non-terminals shared by the $\mathcal{L}$ and $\mathcal{L}'$.

\[
  \begin{array}{c}
    \fbox{$e \pcNotionStep e'\textrm{ where $e, e' \in (\Pika \sqcup \PikaCore)$}$}
    \fboxNewlines
    \labinfer[PC-Int]{E[\Delta;C;\Gamma] \vdash i \pcNotionStep i}{E[\Delta;C;\Gamma] \vdash i : \Int}
    ~~~
    \labinfer[PC-Bool]{E[\Delta;C;\Gamma] \vdash b \pcNotionStep b}{E[\Delta;C;\Gamma] \vdash b : \Bool}
    \\\\
    \labinfer[PC-Var]{E[\Delta;C;\Gamma] \vdash x \pcNotionStep \sslmath{}{}{}}
      {E[\Delta;C;\Gamma] \vdash x : \alpha
      &E[\Delta;C;\Gamma] \vdash \alpha \isA \Layout{X}}
    \\\\
    \labinfer[PC-Lambda]{E[\Delta;C;\Gamma] \vdash \lambda x.\; e \pcNotionStep \lambda x.\; e}{}
    \\\\
    \labinfer[PC-Unfold-Layout-Ctr]{E[\Delta;C;\Gamma] \vdash \apply_A(C\; \overline{e}) \pcNotionStep \sslmath{\overline{x}}{T}{\overline{h'}}}
      {\fresh{\overline{x}}
      & (A\; (C\; \overline{y}) := \sslmath{\overline{x}}{T}{\overline{h}}) \in E
      & \overline{h'} = \overline{h}[\overline{y} := \overline{e}]
      }
    \\\\
    \labinfer[PC-App]{E[\Delta;C;\Gamma] \vdash e_1\;(\apply_A(e_2)) \pcNotionStep \withIn{\{ \overline{x} \} := e_2'} e_1\; \{\overline{x}\}}
      {}
    \\\\
    \labinfer[PC-With-App]{E[\Delta;C;\Gamma] \vdash e\; (\withIn{\{ \overline{x} \} := e_1} e_2) \pcNotionStep \withIn{\{ \overline{x} \} := e_1} e\; e_2}
      {\overline{x} \not\in \FV(e)}
    \\\\
    \labinfer[PC-Apply]{E[\Delta;C;\Gamma] \vdash \apply_A(e) \pcNotionStep e}
      {}
    % \\\\
    % \labinfer[PC-Unfold-Layout-Var]{E[\Delta;C;\Gamma] \vdash \apply_A(x)
      % \pcNotionStep \sslmath{\overline{y}}{\overline{h}}}
    %   {}
  \end{array}
\]

Define the big-step relation induced by the small-step relation by the following

\[
  \begin{array}{c}
    \labinfer{e \pcStep e'}
      {e \pcsteps e' & e' ~\val}
  \end{array}
\]

\begin{theorem} $\pcStep{} \subseteq \Pika \times \PikaCore$
\end{theorem}

\subsubsection{Definition Translation}

Translation of function definitions from \Pika{} to \PikaCore{} is accomplished by the following relation.

\[
  \begin{array}{c}
    \fbox{$E \vdash D^{\Pika} \defStep^{A} D^{\PikaCore}$}
    \fboxNewlines
    \labinfer[PC-Def]{E \vdash f\; (\textrm{Ctr}\; \overline{x}) := e \defStep^{A} f\; \{ \overline{\ell \mapsto y} \} := e' }
      {(A : \Layout{X}) \in E
      &E[\Delta;C;\Gamma] \vdash e \pcStep e'
      }
  \end{array}
\]

% \subsection{Denotational Semantics}
%
% We give a denotational semantics for \PikaCore. Since \Pika{} is translated to \PikaCore, this can
% also be seen as a denotational semantics for \Pika.
%
% The function $\argCount{\tau}$ gives the number of fields in abstract
% heap associated to the type $\tau$. In the case of function types, this
% is the number of fields associated to the \textit{result} of the function.
%
% \begin{align*}
%   % &\typesem{\Int} = \Var \ra \lifted{\mathbb{Z}}
%   % \\
%   % &\typesem{\Bool} = \Var \ra \lifted{\mathbb{B}}
%   % \\
%   % &\typesem{\SSL(\langle x_1, \tau_1 \rangle, \cdots, \langle x_n, \tau_n \rangle)} = \Var^n \ra \lifted{\Heap}
%   % \\
%   % &\typesem{\tau_1 \ra \tau_2} = \typesem{\tau_1} \ra \typesem{\tau_2}
%   &\typesem{\Int} = \namesem{\Int} \ra \Heap
%   \\
%   &\typesem{\Bool} = \namesem{\Bool} \ra \Heap
%   \\
%   &\typesem{\SSL(n)} = \namesem{\SSL(n)} \ra \Heap
%   \\
%   &\typesem{\tau_1 \ra \tau_2} = \namesem{\tau_1 \ra \tau_2} \ra \typesem{\tau_1} \ra \typesem{\tau_2}
% \end{align*}
%
% \begin{align*}
%   &\argCount{\Int} = 1
%   \\
%   &\argCount{\Bool} = 1
%   \\
%   &\argCount{\SSL(n)} = n
%   \\
%   &\argCount{\tau_1 \ra \tau_2} = \argCount{\tau_2}
% \end{align*}
% \\
%
% \begin{figure}
% \begin{center}
%   \fbox{
%     $\sem{e} \in \typesem{\tau}\textrm{ where $E[\bullet] \vdash e : \tau$}$
%   }
% \end{center}
% \begin{align*}
%   &\sem{i}_r = \{ r \mapsto i \}\tag{where $i \in \mathbb{Z}$}
%   \\
%   &\sem{b}_r = \{ r \mapsto b \}\tag{where $i \in \mathbb{B}$}
%   \\
%   &\sem{\withIn{a := e_1} e_2}_r = \sem{e_1}_a \osep \sem{e_2}_r
%   \\
%   &\sem{\sslmath{\overline{x}}{\overline{\ell_i \mapsto e_i}}}_r =
%     \sem{e_1[\overline{x := r}]}_{\ell_1[\overline{x := r}]} \osep \cdots \osep \sem{e_n[\overline{x := r}]}_{\ell_n[\overline{x := r}]}
%   \\
%   &\sem{e\; v}_r =
%       \sem{e}_r(\sem{v}_{-})
%   \\
%   &\sem{\lambda x.\; e}_r(f)
%       = \sem{e}_r \osep f(x)
% \end{align*}
%   \caption{Denotation function for \PikaCore ({\color{red}TODO}: Finish)}
% \end{figure}

% \subsection{Equational soundness}
%
% \begin{theorem}[Type denotation]
%   If $\tau$ is a quantifier-free type without layout constraints then
%   \[
%     \sem{E[\Delta;C;\Gamma] \vdash e : \tau} \in \typesem{\tau}
%   \]
% \end{theorem}
%
% \begin{theorem}[Denotation continuity]
%   $\sem{E[\Delta;C;\Gamma] \vdash e : \tau_1 \ra \tau_2}$ is a continuous function from $\typesem{\tau_1}$ to $\typesem{\tau_2}$.
% \end{theorem}
%
% \begin{theorem}[Equational soundness]
%   \[
%       \boxmath{\sem{E[\Delta;C;\Gamma] \rhd e_1 \defeq e_2 : \tau}}
%     \models
%       \boxmath{\sem{E[\Delta;C;\Gamma] \vdash e_1 : \tau}}
%         =
%       \boxmath{\sem{E[\Delta;C;\Gamma] \vdash e_2 : \tau}}
%   \]
% \end{theorem}
%

\section{Extensions of \suslik}
\label{sec:suslik-extensions}

We have shown the translation from the functional specifications into
SSL specifications. However, some of the SSL specifications are not
supported in the original \suslik and existing variants.
%
In this section, we present how to extend the \suslik to support more
features to make the whole thing work. We will show the extensions on
the following three aspects:

\begin{itemize}
  \item How to describe and call an existing function within SSL predicates.
  \item How to make the result of one function call as the input of another function call.
  \item How to synthesise programs with inductive predicates without the help of pure theory.
\end{itemize}

\subsection{Function Predicates}
\label{sec:funcPred}

Without any modification upon the implementation, we find the SSL
predicate within some restrictions can be used to describe function
relations other than data structures (named function predicates). The
definition of \textbf{function predicates} is as follows:

\begin{definition}[Function Predicates]
    \label{def:funcPred}
    \normalfont
    Given any non-higher-order n-ary function $f$\lstinline{(x1, ..., xn)} in the functional language, the function predicate to synthesise $f$ has the following format:
    \begin{lstlisting}[language=SynLang]
    predicate predf(T x1, ... ,T xn, T output){...}
    \end{lstlisting}
%
    where \lstinline{T} $\in$ \{\lstinline{loc}, \lstinline{int}\}.
    The type of \lstinline{xi} (and \lstinline{output}) is decided by
    the type of $f$. If it is an integer in $f$, then its type is
    \lstinline{int}; otherwise, it is \lstinline{loc} (for any data
    structure in \tool).
\end{definition}

Since the input of the whole workflow is functional programs, the
"output" in the definition is to provide another location for the
output of the function. And the specification to synthesise function
$f$ should have the following format:

\begin{lstlisting}[language=SynLang]
void f(loc x1, ... ,loc xn)
{x1 :-> v1 ** x2 :-> l2 ** sll(l2) ** ... ** xn :-> vn ** output :-> 0}
{x1 :-> v1 ** x2 :-> l2 ** ... ** xn :-> vn ** output :-> output0 ** 
 predf(v1, l2, ..., vn, output0)}
\end{lstlisting}

\subsection{SSL Rules for \func Structure}

As we show in previous examples, the reason we can have \func structure is that the points-to structure in the post-condition is always eliminated after some write operations. For example, the in-placed \lstinline{inc1} functions specification is satisfied via the \writer operation (\autoref{fig:write}) on the location.

\begin{lstlisting}[language=SynLang]
void inc_y(loc y, loc x)
{x :-> vx ** y :-> vy}
{x :-> vx + xy ** y :-> vy}
\end{lstlisting}

\begin{figure}[t]
    \centering
    \begin{mathpar}
      \inferrule[\writer]
      {
      \mcode{\vars{e} \subseteq \env}
      \\
      \mcode{e \neq e'}
      \\
      \trans{\mcode{{\asn{\phi; \ispointsto{x}{e}{} \osep P}}}}
              {\mcode{\asn{\psi; \ispointsto{x}{e}{} \osep Q}}} {\mcode{\prog}}
      }
      {
      \mcode{
      \trans{\asn{\phi; {x} \pts e' \osep P}}
      {\asn{\psi; {x} \pts e \osep Q}}
      {{\deref{x} = e\ ;\ \prog}}
      }
      }
    \end{mathpar}
    
    \caption{The \writer rule in SSL}
    \label{fig:write}
\end{figure}

\begin{figure}[t]
  \centering
  \begin{mathpar}
    \inferrule[\funcwrite]
      {
      \mcode{\forall i \in [1, n], \vars{e_i} \subseteq \env}
      \\
      \trans{\mcode{{\asn{\phi; P}}}}
            {\mcode{\asn{\psi; Q}}} {\mcode{\prog}}
      }
      {
      \mcode{\trans{\asn{\phi; {x} \pts e \osep P}}
      {\asn{\psi; \func\ f(e_1,\ldots,e_n,{x})\osep Q}}
      {{f(e_1,\ldots,e_n,{x})\ ;\ \prog}}
      }
      }
  \end{mathpar}
  
  \caption{The \funcwrite rule in SSL}
  \label{fig:funcwrite}
\end{figure}

The core insight of \func structure is: since the function synthesised
by function predicate behaves like the pure function, it is the same
as the \writer rule in the sense that only the output location is
modified. Thus, we add the new \funcwrite rule into the zoo of SSL
rules (see \autoref{fig:funcwrite}).
%
To make the \func structure correctly equal to some ``write''
operation, the following restrictions should hold, which are achieved
by the translation:

\begin{itemize}
    \item If \lstinline[language=SynLang]{func f(x1, ..., xn, output)} appears in a post-condition, then no write rule can be applied to any \lstinline[language = SynLang]{xi}. This is to avoid the ambiguity of the \func.
    \item The type of function f is consistent.
\end{itemize}

Note that based on the setting of the function predicate, the parameters of the function call are pointers, while the parameters of the function predicate are content to which pointers point. Furthermore, we have the \func generated from function predicates and with the format defined in \autoref{sec:funcPred}. As a result, the equivalent original SSL that duplicates points-to of one location is not a problem, since they can be merged as one.

\subsection{Temporary Location for the Sequential Application}

Though with rich expressiveness, SSL has difficulty in expressing the sequential application of functions. For example, given the \func structure available, the following function is not expressible within one function predicate:

\begin{lstlisting}[language=SynLang]
f x y = g (h x) y
\end{lstlisting}

If we attempt to express it, we will have the following part in the predicate:

\begin{lstlisting}[language=SynLang]
predicate f(loc x, loc y, loc output)
{... ** func h(x, houtput) ** func g(houtput, y, output)}
\end{lstlisting}

However, \lstinline{houtput} is not a location in the pre-condition, which is not allowed in SSL. Thus, we introduce a new keyword \lstinline{temp} to denote the temporary location for the sequential application. The new definition of \lstinline{func} is as follows:

\begin{lstlisting}[language=SynLang]
predicate f(loc x, loc y, loc output)
{... ** temp houtput ** func h(x, houtput) ** func g(houtput, y, output)}
\end{lstlisting}

Roughly speaking, the \lstinline{temp} structure will help to allocate a new location for the output of the first function, and then use it as the input of the second function. After all appearances of \lstinline{houtput} is eliminated, we will deallocate the location.

Note that the temporary variable is possible to appear in two different structures: recursive function predicates or \func call. The reason we don't need to consider the basic arithmetic operations is that the integer will be directly used as the predicate parameter, instead of the location as the parameter. For example, the sum of a list can be expressed as:

\begin{lstlisting}[language=SynLang]
predicate sum(loc l, int output){
| l == 0 => {output == 0; emp}
| l != 0 => {output == output1 + v; [l, 2] ** l :-> v ** l + 1 :-> lnxt ** sum(lnxt, output1)}
}
\end{lstlisting}

Such sequential application is common in functional programming, especially in the recursive function. For example, it is not elegant to flatten a list of lists without the sequential application. 

\begin{lstlisting}[language=Pika]
    flatten :: [[a]] -> [a]
    flatten [] = []
    flatten (x:xs) = x ++ flatten xs
\end{lstlisting}

We can express this function, but with some strange structure to store all temporary lists.

\begin{lstlisting}[language=SynLang]
predicate flatten(loc x, loc output){
| x == 0 => {output :-> 0}
| x != 0 => {[x, 2] ** x :-> x0 ** sll(x0) ** x + 1 :-> xnxt **
[output, 2] ** func append(x, outputnxt, output) ** output + 1 :-> outputnxt **
flatten(xnxt, outputnxt)}
}
\end{lstlisting}

With such a function predicate, though we can synthesise the function
whose result stored in \lstinline{output} is the flattened list, the
list \lstinline{output} is containing a lot of intermediate values,
which is neither consistent with the definition in the source language
nor space efficient.

\begin{figure}[t]
  \centering
  \begin{mathpar}
    \inferrule[\tempfuncalloc]
      {
      \trans{\mcode{{\asn{\phi; x \pts a \osep P}}}}
      {\asn{\psi; \func\ f(e_1,\ldots,e_n,{x})\osep temp (x, 1) \osep Q}} {\mcode{\prog}}
      }
      {
      \mcode{
      \trans{\asn{\phi; P}}
      {\asn{\psi; \func\ f(e_1,\ldots,e_n,{x})\osep temp (x, 0) \osep Q}}
      {{let\ x\ =\ malloc(1)\ ;\ \prog}}
      }
      }
  \end{mathpar}
  % \begin{mathpar}
  %   \inferrule[\temppredalloc]
  %     {
  %     \trans{\mcode{{\asn{\phi; x \pts x0 \osep x0 \pts a \osep P}}}}
  %     {\asn{\psi; p(e_1,\ldots,e_n,{x_0})\osep temp (x, 2, x_0) \osep x \pts x_0 \osep Q}} {\mcode{\prog}}
  %     }
  %     {
  %     \mcode{
  %     \trans{\asn{\phi; P}}
  %     {\asn{\psi; p(e_1,\ldots,e_n,{x})\osep temp (x, 0, \_) \osep Q}}
  %     {}
  %     }
  %     \\
  %     \mcode{
  %       let\ x\ =\ malloc(1);
  %       let\ x0\ =\ malloc(1);
  %       *x\ =\ x0\ ; \prog
  %     }
  %     }
  % \end{mathpar}
  
  \caption{New allocation rule for \lstinline{temp} in SSL }
  \label{fig:newalloc}
\end{figure}

\begin{figure}[t]
  \centering
  \begin{mathpar}
    \inferrule[\tempfuncfree]
      {
      \trans{\mcode{{\asn{\phi; P}}}}
      {\asn{\psi; Q}} {\mcode{\prog}}
      }
      {
      \mcode{\not\exists x\in Q\ \wedge
      \trans{\asn{\phi; P}}
      {\asn{\psi;  temp (x, 1) \osep Q}}
      {}
      }\\
      \mcode{let\ x0\ =\ *x\ ;\ type\_free(x0);\ free(x);\ \prog}
      }
  \end{mathpar}
  % \begin{mathpar}
  %   \inferrule[\temppredfree]
  %     {
  %     \trans{\mcode{{\asn{\phi; P}}}}
  %     {\asn{\psi; Q}} {\mcode{\prog}}
  %     }
  %     {
  %     \mcode{\not\exists x_0\in Q\ \wedge
  %     \trans{\asn{\phi; P}}
  %     {\asn{\psi;  temp (x, 2, x_0) \osep  Q}}
  %     {}
  %     }\\
  %     \mcode{let\ x0\ =\ *x\ ;\ let\ x00\ =\ *x0\ ;\ ds\_free(x00);\ free(x0);\ free(x);\ \prog}
  %     }
  % \end{mathpar}
  
  \caption{New deallocating rule for \lstinline{temp} in SSL}
  \label{fig:newfree}
\end{figure}

The new rules consist of allocating (\autoref{fig:newalloc}) and deallocating rules (\autoref{fig:newfree}). Based on the definition of the \func structure and the function predicate, the allocated locations are different, where the \lstinline{temp} location for \func is directly used; while the \lstinline{temp} location for function predicate should allocate a new location for function predicates. As for the deallocation, not only the \lstinline{temp} location(s) but also the content they point to should be deallocated. That is the reason we have the \lstinline{type_free} function, which is syntax sugar to deallocate the content of a location based on the type information. For example, if the type of the location is \lstinline{tree}, then the \lstinline{type_free} will deallocate the content of the location via \lstinline{tree_free} function, which is synthesised based on the SSL predicate \lstinline{tree} as follows.
\begin{lstlisting}[language=SynLang]
void tree_free(loc x)
  {tree(x)}
  {emp}
\end{lstlisting}
Specifically, if the location contains the value with type \lstinline{int}, then the \lstinline{type_free} will do nothing.
Thus, the function predicate with \lstinline{temp} is much better, in the sense that no extra space is used, and the synthesised function is consistent with the source language.

\begin{lstlisting}[language=SynLang]
predicate flatten(loc x, loc output){
| x == 0 => {output :-> 0}
| x != 0 => {[x, 2] ** x :-> x0 ** sll(x0) ** x + 1 :-> xnxt ** 
temp outputnxt ** flatten(xnxt, outputnxt) ** func append(x, outputnxt, output)}
}
\end{lstlisting}

\subsection{Avoiding Excessive Heap Manipulation with Read-Only Locations}

The existing \suslik depends on the set theory to express the pure
relation. However, it is not trivial to automatically generate the
pure part of SSL specifications from the functional specifications. To
see why the set theory is needed, the following simple example shows
the functionality of the set theory, with \lstinline{sll_n} being the
\textbf{s}ingle-\textbf{l}inked \textbf{l}ist with \textbf{n}o set.

\begin{lstlisting}[language=SynLang]
predicate cp(loc x, loc y) {
|  x == 0        => {y == 0; emp }
|  not (x == 0)  => {
    [y, 2] ** y :-> v ** (y + 1) :-> ynxt **
    [x, 2] ** x :-> v ** (x + 1) :-> xnxt ** cp(xnxt, ynxt) }
}
\end{lstlisting}

While the intent of the function predicate \lstinline{cp} is to copy
the list \lstinline{x} to \lstinline{y}, without the set theory, the
output program will be somewhat surprising to see:

\begin{lstlisting}[language=SynLang]
void copy (loc x, loc y) {
  if (x == 0) {
  } else {
    let n = *(x + 1);
    copy(n, y);
    let y01 = *y;
    let y0 = malloc(2);
    *y = y0;
    *(y0 + 1) = y01;
    let vy = *y0
    *x = vy;
  }
}
\end{lstlisting}

The problem here is that, when we have the pure relation in the
predicate to indicate that the values are the same, the synthesiser
finds another possible way: instead of copying the value of
\lstinline{x} to \lstinline{y}, we can just change the value of x to
initial value \lstinline{vy} after \lstinline[language = c]{malloc}.
This is not the user intent, and the output program is not correct.
%
Turns out, the solution is not that difficult: we simply need add a
new kind of heaplet in the specification language, call
\textit{constant points-to}, which has a similar idea as read-only
borrows~\cite{costea2020concise}.
%
% Unlike the complex permissions in the robosuslik designed for
% different purposes, we only need to get rid of the set theory in the
% function predicate. Since the source language is a functional
% language, it will be natural to make the input remain unmodified. 
%
The only difference of the constant points-to from the original
\textit{points-to} heaplet is that the value of the location is
constant, which means that the \writer rule in SSL is not applicable.

% \subsection{In-place Function by Basic In-place Description}

% With all extensions above, we have a complete specification language for the translation result of a basic functional source. However, all result programs are performed as pure functions. That is a normal result, considering the definition of functional language. But in some cases, the user may want to have the in-place function, which means that the input will be modified. For example, the user may want to have a function to reverse a list, and the input list will be reversed after the function call. In this case, such intent is impossible to be expressed by the basic source language, which is the reason we have another extension in the functional specification. As shown previously, the function predicate in SSL is to describe the pure function. But for the in-place function, it is kind of necessary for the user to describe the memory behaviour of the function. 

% As the example in Sec. \ref{sec:overview} shows, the good point is that the higher-order in-place function can be achieved by the basic in-place function. For the simplest case, no more effort is needed when the basic in-place function itself is  following the restriction of \func structure. However, it is possible that some basic in-place functions are modifying multiple locations. For example, the in-place version of function \lstinline{cons} has the following specification after translation into \suslik.

% \begin{lstlisting}[language=SynLang]
%   void cons(loc x, loc xs)
%   {x :-> v ** xs :-> vxs ** sll(vxs)}
%   {[x, 2] ** x :-> v ** x + 1 :-> vxs ** sll(vxs) ** xs :-> 0}
% \end{lstlisting}

% Then the previous \func will not work. But following the basic idea of \func, where the execution of function call performs similar as \writer , we can have the following extension of \inplace structure, which is similar to \func except that the \lstinline{output} becomes a list of locations which are modified in the specification.
% \begin{lstlisting}
%   in_place f(e, ..., [x1, .., xn])
% \end{lstlisting}
% And a similar extension of SSL rules is shown in \autoref{fig:inplacewrite}.

% \begin{figure}[t]
%   \centering
%   \begin{mathpar}
%     \inferrule[\inplacewrite]
%       {
%       \mcode{\forall i \in [1, n], \vars{e_i} \subseteq \env}
%       \\
%       \env; \trans{\mcode{{\asn{\phi; P}}}}
%             {\mcode{\asn{\psi; Q}}} {\mcode{\prog}}
%       }
%       {
%       \mcode{
%       \env; \translong{\asn{\phi; {x_1} \pts e_1' \sep \ldots \sep {x_n} \pts e_n' \sep P}}
%       {\asn{\psi; \inplace~f(e_1,\ldots,e_n,[{x_1},\ldots,{x_n}])\sep Q}}
%       {}
%       }
%       \\
%       \mcode{{f(e_1,\ldots,e_n,{x_1},\ldots, {x_n})\ ;\ \prog}}
%       }
%   \end{mathpar}
  
%   \caption{The \inplacewrite rule in SSL \todo{Gamma bug}}
%   \label{fig:inplacewrite}
% \end{figure}

% Moreover, we need to deal with the combination of basic in-place functions and temporary location, where the automatically free should be handled subtly.

\section{Synthesis Limitations And Solutions}
\label{sec:limitations}

There is a significant limitation to our approach so far: \SuSLik{} will often fail to synthesize code from valid specifications. One instance of this problem is the following: In a certain class of functions, \SuSLik{} requires a different number of pointer indirections than it will with other functions. We have not found a way to perfectly characterize such functions and, since this will change the way the functions will need to be called, this represents a major impediment to our approach. It is worth noting that this is also not the \textit{only} situation in which \SuSLik{} fails to synthesize a program.

From this, we draw the conclusion that \SuSLik{} is not well-suited to this task. In fact, this task is very different from the original goals of \SuSLik, which could be stated: \textit{Given a potentially ambiguous specification, generate any program that satisfies this specification.} However, in our case, we already know which specific program we want to generate. This is fully specified by the abstract machine semantics for \Pika.

With this in mind, we take a new approach. What if we compile \Pika{} directly to C, but we add a \textit{new} feature that allows us to express synthesis problems at the level of \Pika{} code? This takes the form of the \verb|synth| keyword, which we explain in detail in Section~\ref{sec:synth}. In this case, we can get the advantages of a traditional compiler while also exploiting the synthesis capabilities of \SuSLik. Furthermore, due to \Pika's design, we already have easy interoperability with \SuSLik. Also, recall that \SuSLik's target language, \SuSLang, can be straightforwardly translated directly to C. This allows us to easily bring all of these things together in our new implementation with the new \Pika{} synthesis feature.


\section{\texttt{synth} Keyword: Synthesis Inside \Pika{} 2}
\label{sec:synth}

Next, we introduce a feature that allows us to synthesise a program from a special kind of \Pika{} type signature. This special kind of type signature allows us to directly access \SuSLik's pure propositions. These type signatures are called \synth-signatures and the syntax for invoking the \Pika{} synthesis on a \synth-signature is:
\[
  \synth\; f : \sigma
\]
\noindent
where $\sigma$ is a \synth-signature. The syntax of a \synth-signature is given in Figure~\ref{fig:synth-sig-syntax}.

\begin{figure}
\[
\begin{aligned}
  \textnormal{\synth-signatures}\;\; &\sigma ::= p \synthsep \tau'\\
  \textnormal{Pure propositions}\;\; &p ::= x \mid i \mid b \mid p \mathbin{\&\&} p \mid p \mathbin{||} p
    \mid \Not(p) \mid p == p \mid \{\} \mid \{p\} \mid p \app p \mid p + p\\
  \textnormal{Extended types}\;\; &\tau' ::= ... \mid A\; \overline{G}\\
  \textnormal{Ghost variables}\;\; &G ::= @x
\end{aligned}
\]
  \caption{\synth-signature syntax}
  \label{fig:synth-sig-syntax}
\end{figure}

\noindent
We also need to extend layout definitions so that they can include ``ghost parameters.'' These parameters will directly correspond to ghost variable parameters in the resulting \SuSLik{} specification.

This extension to layouts is best explain by way of example. First, consider the original definition of the singly-linked list type:

\begin{lstlisting}[language=Pika]
Sll : List >-> layout[x]
Sll Nil := emp;
Sll (Cons head tail) :=
  x :-> head ** (x+1) :-> nxt
    ** Sll tail [nxt];
\end{lstlisting}

\noindent
Note that in \Pika{} 2, we distinguish between the pattern variables (such as \verb|tail|) and the SSL parameters (such as \verb|nxt|). In the recursive application of \verb|Sll|, we apply it to \verb|tail| and we substitute \verb|nxt| for \verb|x|.

This does not use any ghost variables in \Pika{} or in the resulting \SuSLik{} specification. A common pattern in \SuSLik{} specifications is to use a \verb|set| type ghost variable to keep track of the elements in a data structure. Then, we can require that the elements are preserved or that new elements be added when using this in specifications. The feature we add to layouts allows us to use this feature directly:

\begin{lstlisting}[language=Pika]
Sll : @(s : set) List >-> layout[x];
Sll Nil := s == {} ;; emp;
Sll (Cons head tail) :=
  s == {head} ++ s1
    ;;
  x :-> head ** (x+1) :-> nxt
    ** Sll tail @s1 [nxt];
\end{lstlisting}

\noindent
Putting this all together, we can use this new kind of parameter together with the \synth{} keyword to specify a list append function:

\begin{lstlisting}[language=Pika]
synth append :
  s == s1 ++ s2
    ;;
  Sll @s1 -> Sll @s2 -> Sll @s;
\end{lstlisting}

\noindent
Using this feature, we can use \SuSLik{} specifically for \textit{synthesis} tasks inside \Pika. That is, we are able to directly use \SuSLik{} for its intended purpose.



\section{Examples}
\label{sec:examples}

\subsection{\Pika{} 1}

Working \Pika{} 1 examples are presented in Figure~\ref{fig:size-comparison}. For each example, we compare the AST size of the \Pika{} code with that of the generated \SuSLik{} code. This is a rough measure of the relative expressiveness of \Pika{} and \SuSLik{} on these examples. Each example in the list consists of a single function which was tested by hand.

The \verb|filterLt9| example makes use of the Boolean guard feature. Guards must always occur at the level of a function
definition, rather than inside of a subexpression. On the other hand, the \verb|maximum| example uses the \lstinline[language=Pika]{if-then-else}
construct. This is necessary since it occurs inside of a \lstinline[language=Pika]{let}. The \verb|leftList| example uses two different
ADTs: lists and binary trees. A singly-linked list layout is used for the list and a linked representation is used as a layout for the binary tree.

\begin{figure}[b]
\setlength{\abovecaptionskip}{5pt}
\setlength{\belowcaptionskip}{-15pt}
\begin{center}
  \begin{table}[H]
  \begin{tabular}{|c|c|c|c|}
    \hline
    Name & \suslik size & \tool 1 size & \tool 1 size / \suslik size\\
    \hline
    \verb|append| & 164 & 84 & 0.512 \\
    \verb|cons| & 82 & 52 & 0.634 \\
    \verb|filterLt9| & 136 & 66 & 0.485 \\
    \verb|singleton| & 70 & 47 & 0.671 \\
    \verb|map| & 147 & 93 & 0.633 \\
    \verb|mapSum| & 138 & 97 & 0.703 \\
    \verb|maximum| & 106 & 61 & 0.575 \\
    \verb|leftList| & 144 & 104 & 0.722 \\
    \verb|take| & 180 & 112 & 0.622\\
    \hline
  \end{tabular}
  \end{table}
\end{center}
  \caption{\tool 1 spec size vs generated SSL spec size measured in number of AST nodes}
  \label{fig:size-comparison}
\end{figure}

\subsection{\Pika{} 2}

Version 2 of \Pika{} is a rewrite of version 1 and it is missing a couple of features of version 1. However, it has the \synth{} feature. It also has the groundwork for layout polymorphism to be implemented. The second aspect, in particular, is a substantial change from version 1 and motivated the rewrite.

Figure~\ref{fig:pika-2-examples} is a table of \Pika{} 2 examples together with the number of functions they have (not including \synth). These functions were tested using the unit testing framework built into \Pika{} 2.

The \lstinline[language=Pika]{synth} feature is used in the \verb|set| example for two functions: The \verb|append| function mentioned in \autoref{sec:synth} and the
\verb|setToSll| function that converts set values to linked list values:

\begin{lstlisting}[language=Pika]
synth setToSll : SetLayout @s -> Sll @s;
\end{lstlisting}

\begin{figure}
  \begin{tabular}{|c|c|}
    \hline
    Name & \# of functions\\
    \hline
    \verb|add1Head| & 1\\
    \verb|anagram| & 4\\
    \verb|fact| & 1\\
    \verb|filterLt| & 1\\
    \verb|heap| & 6\\
    \verb|leftList| & 1\\
    \verb|mapAdd| & 1\\
    \verb|maximum| & 1\\
    \verb|set| & 3\\
    \verb|sum| & 1\\
    \verb|take| & 1\\
    \verb|treeSize| & 1\\
    \hline
  \end{tabular}
  \caption{\Pika{} 2 examples}
  \label{fig:pika-2-examples}
\end{figure}                   


\section{Related Work}
\label{sec:related-work}

\tool is built upon the \suslik synthesis framework. \suslik provides
a synthesis mechanism for heap-manipulating programs using a variant
of separation logic.~\cite{polikarpova:2019:suslik} However, it does
not have any high-level abstractions. In particular, writing \suslik
specifications directly involves a significant amount of pointer
manipulation. Further, it does not provide abstraction over specific
memory layouts. As described in \autoref{sec:language}, \tool
addresses these limitations.

The \tname{Dargent} language~\cite{chen:2023:dargent} also includes a
notion of layouts and layout polymorphism for a class of algebraic
data types, which differs from our treatment of layouts in two primary
ways:

\begin{enumerate}

\item In \tool, abstract memory locations (with offsets) are used. In
  contrast, \tname{Dargent} uses offsets that are all
  relative to a single ``beginning'' memory location. The \tool
  approach is more amenable to heap allocation, though
  this requires a separate memory manager of some kind. This is
  exposed in the generated language with \verb|malloc| and
  \verb|free|.
  On the other hand, the technique taken by \tname{Dargent} allows
  for greater control over memory management. This makes
  dealing with memory more complex for the programmer, but it is no
  longer necessary to have a separate memory manager.

  \item Algebraic data types in the present language include
    \emph{recursive} types and,
    as a result, \tool has recursive layouts for these ADTs. This
    feature is not currently available in \tname{Dargent}.
  \end{enumerate}

Furthermore, layout polymorphism also works differently. While
\tname{Dargent} tracks layout instantiations at a type-level with type
variables, in the present work we simply only check to see if a layout
is valid for a given type when type-checking. In particular, we cannot
write type signatures that \textit{require} the same layout in
multiple parts of the type (for instance, in a function type
\verb|List -> List| we have no way at the type-level of requiring that
the argument \verb|List| layout and the result \verb|List| layout are
the same). \Pika{} 2 makes progress towards eliminating this restriction but, as of now, it is layout monomorphic. This more rudimentary approach that \tool currently takes
could be extended in future work.
%
Overall, the examples in the \tname{Dargent} paper tend to focus on
the manipulation of integer values. In contrast, we have focused
largely on data structure manipulation, which follow the primary
motivation of \suslik.

\begin{figure}
  \begin{tabular}{|c|c|c|}
    \hline
    Feature & \Pika & \tname{Dargent}\\
    \hline
    Bit-level layouts & \xmark & \cmark\\
    Recursive layouts & \cmark & \xmark\\
    Synthesis from type signatures & \cmark & \xmark\\
    Custom record accessors & \xmark & \cmark\\
    \hline
  \end{tabular}
  \caption{Comparison to \tname{Dargent}}
\end{figure}

\tname{Synquid} is another synthesis framework with a functional
surface language. While \tname{Synquid} allows an even higher-level
program specification than \tool through its liquid types, it does not
provide any access to low-level data structure
representation.~\cite{polikarpova:2016:synquid} In contrast, \tool's
level of abstraction is similar to that of a traditional functional
language but, similar to \tname{Dargent}, it also allows control over
the data structure representation in memory. The \synth{} feature
has similarities to \tname{Synquid}'s approach to program synthesis, but
\Pika{} 2's \synth{} allows lower level constraints to be expressed.



\section{Future Work}
\label{sec:future-work}

One direction for future work is the extension of \Pika{} 2 to include support for higher-order functions and finishing the implementation of layout polymorphic type checking.

The reverse transformation deserves further investigation: if we go from
an SSL specification to \Pika{} program and then compile to, \eg, C, can
we synthesise additional programs that a traditional SSL synthesisers
would struggle with? What are the limitations of this approach?

Finally, is it possible to derive translations for languages such as \Pika{}
from abstract machine semantics? In this paper, we have given a
language with abstract machine semantics. We then give a translation
of that language into SSL. We then show that the final states given by
the abstract machine semantics are models for the SSL propositions
produced by our translation. But is it possible to begin by specifying the
abstract machine semantics and then mathematically (or automatically)
\textit{derive} an appropriate translation into SSL, with the
requirement that the translation satisfies the soundness theorem?



\section{Conclusion}
\label{sec:conclusion}

We have seen a technique for generating an SSL specification from a functional program, such that this translation is sound with respect to the standard semantics. In order to do this, we built the notion of a \textit{layout} of an algebraic datatype on top of separation logic. Then we observed that \SuSLik, the tool we use to synthesize programs from SSL, has notable limitations in its ability to synthesize the SSL specifications we generate. This motivated the creation of the \synth{} keyword. This new keyword allows us to have direct access to the synthesis features of \SuSLik{} from inside the \Pika{} programming language.



\appendix

\bibliographystyle{ACM-Reference-Format}
\bibliography{paper}

\section{Soundness Proof}
\label{sec:soundness-proof}

First, we introduce an operator that will make it more convenient to talk about the conjunction of two SSL propositions:
\[
  (p_1 ; s_1) \otimes (p_2 ; s_2) = (p_1 \land p_2 ; s_1 \sep s_2)
\]

\noindent
Next, we present a lemma regarding this operator that will be used in the soundness proof:
\begin{lemma}[$\otimes$ pairing]\label{thm:otimes-entail}
  If $(\sigma_1, h_1) \models p$ and $(\sigma_2, h_2) \models q$ and $\sigma_1 \subseteq \sigma_2$ and $h_1 \mathbin{\bot} h_2$,\\
  then $(\sigma_2, h_1 \circ h_2) \models p \otimes q$
\end{lemma}

\noindent
Now we proceed to soundness.

\soundnessThm*

\begin{proof}
  Proceeding by induction on $e$:
  \begin{itemize}
    \item $e = i$ for some $i \in \mathbb{Z}$.\\
      We have 
      \begin{itemize}
        \item $\Tsem{i}{V,r} = (v == i ; \emp)$
        \item $v \fresh V$
        \item $v = r$
        \item $\EndSigma = \StartSigma \cup \{ (r,i) \}$
        \item $\End{h'} = \Start{h} = \emptyheap$
      \end{itemize}
      Plugging these in, we find that we want to prove
      \[
        (\StartSigma \cup \{ (r,i) \}, \emptyheap) \models (r == i ; \emp)
      \]
      This immediately follows from the rules of separation logic.

    \item $e = b$ for some $b \in \mathbb{B}$. This case proceeds exactly as the previous case.

    \item $e = e_1 + e_2$.
      From the $\Tsem{\cdot}{}$ hypothesis, we have
      \begin{itemize}
        \item $(e_1, V_0) \tstep (p_1, s_2, V_1, v_1)$
        \item $(e_2, V_1) \tstep (p_2, s_2, V_2, v_2)$
        \item $v \fresh V_2$
        \item $v = r$
        \item $\Tsem{e_1 + e_2}{V, r} = (v == v_1 + v_2 \land p_1 \land p_2 ; s_1 \sep s_2)$
        \item $\Tsem{e_1}{V_0,v_1} = (p_1 ; s_1)$
        \item $\Tsem{e_2}{V_1,v_2} = (p_2 ; s_2)$
        % \item $ $
      \end{itemize}

      Plugging these in, we want to show
      \[
        \EndPair \models (v == v_1 + v_2 \land p_1 \land p_2 ; s_1 \sep s_2)
      \]

      Furthermore, we see that
      \[
        \Tsem{e}{V,r} = (v == v_1 + v_2 ; \emp) \otimes \Tsem{e_1}{V_0,v_1} \otimes \Tsem{e_2}{V_1,v_2}
      \]

      From the {\sc AM} relation hypothesis we have
      \begin{enumerate}
        \item $\EndSigma = \sigma_y \cup \{(r, z)\}$
        \item $z = x' + y'$
        \item $(x, \sigma, \mathcal{F}, h_1) \step (x', \sigma_x, h_1', \mathcal{F}, v_x)$
        \item $(y, \sigma_x, \mathcal{F}, h_2) \step (y', \sigma_y, h_2', \mathcal{F}, v_y)$
        \item $v_1 = v_x$
        \item $v_2 = v_y$
        \item $h = h_1 \circ h_2$
        \item $h' = h_1' \circ h_2'$
      \end{enumerate}

      From the first six items, we can derive
      \[
        (\EndSigma, \emptyheap) \models (v == v_1 + v_2 ; \emp)
      \]

      From the inductive hypotheses, we get
      \begin{itemize}
        \item $(\sigma_x, h_1') \models \Tsem{e_1}{\dom(\sigma_x), v_x}$
        \item $(\sigma_y, h_2') \models \Tsem{e_2}{\dom(\sigma_y), v_y}$
      \end{itemize}

      By Lemma~\ref{thm:otimes-entail}
      \[
        (\sigma_y, h_1' \circ h_2') \models \Tsem{e_1}{\dom(\sigma_x), v_x} \otimes \Tsem{e_2}{\dom(\sigma_y), v_y}
      \]

      and we know that $\sigma_x \subseteq \sigma_y$.

      By Lemma~\ref{thm:otimes-entail}, we conclude
      \[
        (\sigma_y \cup \{(r, z)\}, h') \models (v == v_1 + v_2 ; \emp) \otimes \Tsem{e_1}{\dom(\sigma_x),v_1} \otimes \Tsem{e_2}{\dom(\sigma_y),v_2}
      \]

    \item $e = v$ for some $v \in \Var$.
      Therefore the only two {\sc AM} rules that apply are {\sc AM-Base-Var} and {\sc AM-Loc-Var}. Both
      cases proceed in the same way.

      We know
      \begin{itemize}
        \item $h = h' = \emptyheap$
      \end{itemize}

      So we want to show
      \[
        (\EndSigma, \emptyheap) \models (\verb|true| ; \emp)
      \]
      This trivially holds.

    \item $e = \lowerS{A}{v}$ for some layout $A$ and $v \in \Var$.
      The applicable {\sc AM} rule is {\sc AM-Lower}.
      From the premises of {\sc AM-Lower}, we have
      \begin{itemize}
        \item $h' = \hat{h} \cdot H'$
      \end{itemize}
      where $H'$ comes from applying the layout $A$ to the constructor application expression
      obtained by reducing $e$. As a result, $H'$ exactly fits the specification of
      one of the branches of the $A$ layout.

      $v$ must be associated with some high-level value in $\mathcal{F}$ and $\hat{h}$
      is the part of the heap that is updated when this is reduced. Since all expressions
      are required to be well-typed, $\hat{h}$ must satisfy some collection of heaplets
      in the $A$ layout branch that is associated to $H'$.

      Note that $\Tsem{e}{V,r} = A(v)$

      From these facts, we can conclude
      \[
        \EndPair \models A(v)
      \]

    \item $e = \lowerS{A}{C\; e_1 \cdots e_n}$. This case is similar to the previous case,
      except that we do not need to lookup a variable in the store before dealing with the constructor
      application.

    \item $e = \instantiateS{A,B}{f}(v)$ for some $v \in \Var$.
        The rule {\sc AM-Instantiate} applies here.

        The {\sc S-Inst-Var} rule requires that we satisfy the inductive predicate
        associated to $f$, that is $\mathcal{I}_{A,B}(f)(v)$. Note that the SSL propositions in the definition of
        such an inductive predicate, given by {\sc FnDef}, will always not only explicitly
        describe the memory used in its result (given by the layout $B$) but also the memory used
        in its argument (given by the layout $A$).

        Now, consider that the {\sc AM-Instantiate} rule uses the layout $A$ to the function argument to
        update the heap and uses the $B$ layout to produce the function's result on the heap. This will match the
        inductive predicate associated to $f$ instantiated at the layouts $A$ and $B$, as required.

      %   Among our hypotheses, we have
      % \begin{itemize}
      %   \item $\End{h'} = \hat{h}_1 \circ \hat{h}_2 \circ \cdots \circ \hat{h}_{n+1}$
      %   \item $\EndSigma = \sigma_f$
      %   \item $(A[x]\; (C\; a_1 \cdots a_n) := H) \in \Sigma$
      %   \item $(f\; (C\; b_1 \cdots b_n) := e_f) \in \Sigma$
      %   \item $(\lowerS{B}{e_f[b_1 := r_1]\cdots[b_n := r_n]}, \sigma_{n+1}, \hat{h}_1) \step (e_f', \sigma_f, \hat{h}_2, r)$
      % \end{itemize}
      %
      % Let $\hat{h} = \hat{h}_1 \circ \hat{h}_2 \circ \cdots \circ \hat{h}_n$. Note that we stop at $n$, not $n+1$. This is the
      % difference between $\hat{h}$ and $\End{h'}$.
      %
      % We can recursively apply the present lemma to the last item, since it is a derivation of an abstract machine reduction which is
      % strictly smaller than the abstract machine reduction derivation we started with. From this, we obtain:
      %
      % \[
      %   (\EndSigma, \hat{h}) \models \lowerS{B}{e_f[b_1 := r_1]\cdots[b_n := r_n]}
      % \]


      % and from the {\sc AM} relation hypothesis we have
      % \begin{itemize}
      %   \item $(x, \sigma, \mathcal{F}, h) \step (x', \sigma_x, h, \mathcal{F}, v_x)$
      % \end{itemize}

      \item $e = \instantiateS{A,B}{f}(C\; e_1 \cdots e_n)$.
        This is much like the previous case. The only difference is that the {\sc S-Inst-Constr} rule
        unfolds the inductive predicate $\mathcal{I}_{A,B}(f)(C\; e_1 \cdots e_n)$. The resulting SSL proposition
        will still be satisfied by the model $\EndPair$ given by {\sc AM-Instantiate}.

      \item $e = \instantiateS{B,C}{f}(\instantiateS{A,B}{g}(e_0))$.
        This case combines to instantiates together. The main condition we need to check here is that the
        two instantiate applications use disjoint parts of the heap in $\EndPair$ in {\sc AM-Instantiate}.

        This can be seen to be true by the fact that the resulting heap is built up out of the subexpressions
        using $\circ$, ensuring that the parts of the heap are disjoint.
  \end{itemize}
  % {\color{red} ???}
\end{proof}


\end{document}
\endinput
%%
