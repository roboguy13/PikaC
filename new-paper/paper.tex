%%
%% This is file `sample-acmsmall.tex',
%% generated with the docstrip utility.
%%
%% The original source files were:
%%
%% samples.dtx  (with options: `acmsmall')
%% 
%% IMPORTANT NOTICE:
%% 
%% For the copyright see the source file.
%% 
%% Any modified versions of this file must be renamed
%% with new filenames distinct from sample-acmsmall.tex.
%% 
%% For distribution of the original source see the terms
%% for copying and modification in the file samples.dtx.
%% 
%% This generated file may be distributed as long as the
%% original source files, as listed above, are part of the
%% same distribution. (The sources need not necessarily be
%% in the same archive or directory.)
%%
%%
%% Commands for TeXCount
%TC:macro \cite [option:text,text]
%TC:macro \citep [option:text,text]
%TC:macro \citet [option:text,text]
%TC:envir table 0 1
%TC:envir table* 0 1
%TC:envir tabular [ignore] word
%TC:envir displaymath 0 word
%TC:envir math 0 word
%TC:envir comment 0 0
%%
%%
%% The first command in your LaTeX source must be the \documentclass
%% command.
%%
%% For submission and review of your manuscript please change the
%% command to \documentclass[manuscript, screen, review]{acmart}.
%%
%% When submitting camera ready or to TAPS, please change the command
%% to \documentclass[sigconf]{acmart} or whichever template is required
%% for your publication.
%%
%%
\documentclass[acmsmall]{acmart}
\usepackage{graphicx}
\usepackage{proof}
\usepackage{stmaryrd}
\usepackage{amsmath}
%\usepackage{amssymb}
\usepackage{xcolor}
\usepackage{tcolorbox}
\usepackage{tikz}
\usepackage{tikz-cd}
\usepackage{syntax}
\usepackage{listings}
\lstset{basicstyle=\ttfamily, mathescape=true, literate={~} {$\sim$}{1}}

% From https://tex.stackexchange.com/questions/235118/making-a-thicker-cdot-for-dot-product-that-is-thinner-than-bullet
\makeatletter
\newcommand*\bigcdot{\mathpalette\bigcdot@{.5}}
\newcommand*\bigcdot@[2]{\mathbin{\vcenter{\hbox{\scalebox{#2}{$\m@th#1\bullet$}}}}}
\makeatother

\newcommand {\Pika} {\textsf{Pika}}
\newcommand {\PikaCore} {\textsf{PikaCore}}

\newcommand {\instExpr} {\keyword{inst}}

\newcommand {\labinfer} [3] [] {\infer[{\textsc{#1}}]{#2}{#3}}

% \newcommand {\keyword} [1] {\textsf{#1}}
\newcommand {\keyword} [1] {\textbf{#1}}

\newcommand {\Int} {\keyword{Int}}
\newcommand {\Bool} {\keyword{Bool}}
\newcommand {\Type} {\keyword{Type}}
% \newcommand {\layout} [1] {\keyword{layout}(#1)}
\newcommand {\layout} {\keyword{layout}}

\newcommand {\data} {\keyword{data}}

\newcommand {\inL} {\keyword{inL}}
\newcommand {\inR} {\keyword{inR}}
\newcommand {\fold} {\keyword{fold}}
\newcommand {\unfold} {\keyword{unfold}}
\newcommand {\fst} {\keyword{fst}}
\newcommand {\snd} {\keyword{snd}}
\newcommand {\either} {\keyword{either}}

\newcommand {\outs} {\longrightarrow}

\newcommand {\Pair} {\otimes}
\newcommand {\Either} {\oplus}
\newcommand {\RecTy} {\mu}

\newcommand {\monic} {\rightarrowtail}

\newcommand {\type} {\textsf{type}}
\newcommand {\ctype} {\textsf{ctype}}
\newcommand {\adt} {\textsf{adt}}

\newcommand {\LayoutDefs} {\mathcal{L}}

\newcommand {\matchWith} [1] {\keyword{case}\,#1\,\keyword{of}\,}
\newcommand {\layoutmatch} [1] {\keyword{layoutcase}\,#1\,\keyword{of}\,}

\newcommand {\withIn} [1] {\keyword{with}\,#1\,\keyword{in}\,}
\newcommand {\letPure} [1] {\keyword{letpure}\,#1\,\keyword{in}\,}

\newcommand{\cond}{\textnormal{cond}}

\newcommand{\Pure}{\textnormal{Pure}}
\newcommand{\Spatial}{\textnormal{Spatial}}


% % Based on https://tex.stackexchange.com/questions/451786/how-do-i-put-a-circle-around-a-symbol
% \makeatletter
% \newcommand {\osep}{\mathbin{\mathpalette\make@circled\ast}}
% \newcommand{\make@circled}[2]{%
%   \ooalign{$\m@th#1\smallbigcirc{#1}$\cr\hidewidth$\m@th#1#2$\hidewidth\cr}%
% }
% \newcommand{\smallbigcirc}[1]{%
%   \vcenter{\hbox{\scalebox{0.77778}{$\m@th#1\bigcirc$}}}%
% }
% \makeatother

% \newcommand {\lowerExpr} {\keyword{lower}}
% \newcommand {\liftExpr} {\keyword{lift}}
\newcommand {\instantiate} {\keyword{instantiate}}
\newcommand {\apply} {\keyword{apply}}

\newcommand {\liftExpr} {\keyword{lift}}

\newcommand {\tyApply} {\mathbin{@}}

\newcommand {\isA} {\sim}

\newcommand {\ra} {\rightarrow}
\newcommand {\Ra} {\Rightarrow}

\newcommand {\Dargent} {{\sc Dargent}}

\newcommand {\highlight} [1] {\begin{tcolorbox}[hbox] #1 \end{tcolorbox}}

\newcommand {\tyLambda} {\mathrm{\Lambda}}

\newcommand {\argCount} [1] {N\sem{#1}}
\newcommand {\namesem} [1] {\Var^{\argCount{#1}}}

% \newcommand {\pcsubscript} {\textsf{PC}}
% \newcommand {\step} {\longrightarrow}
% \newcommand {\steps} {\longrightarrow^{*}}
% \newcommand {\pcstep} {\step_{\pcsubscript}}
% \newcommand {\pcsteps} {\steps_{\pcsubscript}}
% \newcommand {\pcNotionStep} {\rightsquigarrow_{\pcsubscript}}

\newcommand {\FV} {\textsf{FV}}

% \newcommand {\bigstep} {\Downarrow}
% \newcommand {\pcStep} {\bigstep_{\pcsubscript}}

% \newcommand {\defStep} {\bigstep_{\pcsubscript}}

\newcommand {\val} {\textsf{value}}

\newcommand {\fresh} [1] {#1 \textrm{ fresh}}

\newcommand {\recspec} [1] {\langle #1 \rangle}

\newcommand {\Layout} [1] {\keyword{Layout}(#1)}

% \newcommand {\todo} [1] {\text{{\color{red}TODO}: #1}}
%
\newcommand {\dom} {\textsf{dom}}
\newcommand {\range} {\textsf{range}}

% \newcommand {\ssl} [2] {\keyword{ssl}(#1) \{ #2 \}}
% \newcommand {\ssl} [1] {\keyword{ssl} \{ #1 \}}
\newcommand {\sslmath} [3] {\ssl\{ #1 \}\{ #2 \}\{ #3 \}}
\newcommand {\ssl} {\keyword{layout}}
\newcommand {\SSL} {\keyword{SSL}}

\newcommand {\synth} {\keyword{synth}}

\newcommand {\synthsep} {\mathbin{;;}}
\newcommand {\Not} {\keyword{not}}

\newcommand {\app} {\mathbin{++}}

\newcommand {\SuSLik} {SuSLik}
\newcommand {\SuSLang} {SuSLang}

% \newcommand {\Var} {\textsf{Var}}

% \newcommand {\Heap} {\textsf{Heap}}

\newcommand {\lifted} [1] {#1_{\bot}}

\newcommand {\eq} {\stackrel{\bigcdot}{=}}
\newcommand {\defeq} {:=}
\newcommand {\ok} [1] {\;\textsf{ok}_{#1}}

\newcommand {\typesem} [1] {\mathcal{A}_{#1}}

\newcommand {\True} {\keyword{True}}
\newcommand {\False} {\keyword{False}}

\newcommand {\generate} {\keyword{generate}}

\newcommand {\vdashpc} {\vdash_{\textsf{PC}}}

\newcommand {\fboxNewlines} {\\\\\\}

\newcommand {\layoutParams} {\textsf{params}}

\newcommand{\eg}{\emph{e.g.}}
\newcommand{\etal}{\emph{et~al.}}

\newcommand{\sep}{*}


\newcommand{\step} {\ensuremath{\longmapsto}}
\newcommand{\tstep} {\ensuremath{\Downarrow}}
\newcommand{\defstep} [1] {\ensuremath{\xmapsto{\textnormal{fn-def}}_{#1}}}
\newcommand{\Cstep} {\ensuremath{\longmapsto_{\mathfrak{C}}}}

\newcommand{\Expr} {\textnormal{Expr}}
\newcommand{\Heap} {\textnormal{Heap}}
\newcommand{\Store} {\textnormal{Store}}
\newcommand{\Var} {\textnormal{Var}}
\newcommand{\Loc} {\textnormal{Loc}}
\newcommand{\Val} {\textnormal{Val}}
\newcommand{\FsVal} {\textnormal{FsVal}}
% \newcommand{\labinfer} [3] [] {\infer[{\textsc{#1}}]{#2}{#3}}
% \newcommand{\fresh} {\mathbin{\#}}

\newcommand{\freshLoc} [1] {#1\;\textnormal{fresh}}
\newcommand{\freshVar} [1] {#1\;\textnormal{fresh}}

\newcommand{\FsStore} {\textnormal{FsStore}}
\newcommand{\VarSet} {V}

\newcommand{\lowerS} [2] {\texttt{lower}_{#1}(#2)}
\newcommand{\instantiateS} [2] {\texttt{inst}_{#1}(#2)}

\newcommand{\emptyheap} {\varnothing}

\newcommand{\emp} {\texttt{emp}}
\newcommand{\pointsto} {\ensuremath{\mathbin{\texttt{:->}}}}

\newcommand{\tname}[1]{\textsf{#1}}

\newcommand {\tool} {\Pika\xspace}
\newcommand {\suslik} {\SuSLik\xspace}


\newcommand{\sem} [1] {\llbracket#1\rrbracket}
\newcommand{\Tsem} [2] {\mathcal{T}\sem{#1}_{#2}}

\newcommand{\concrete} [1] {\textnormal{\;concrete$_{#1}$}}

\newcommand{\Start} [1] {{\color{cyan} #1}}
\newcommand{\End} [1] {{\color{orange} #1}}

\newcommand{\StartSigma} {\Start{\sigma}}
\newcommand{\EndSigma} {\End{\sigma'}}

\newcommand{\StartPair} {(\StartSigma, \Start{h})}
\newcommand{\EndPair} {(\EndSigma, \End{h'})}


\definecolor{pblue}{rgb}{0.13,0.13,1}
\definecolor{pgreen}{rgb}{0,0.5,0}
\definecolor{pred}{rgb}{0.7,0,0}
\definecolor{pgrey}{rgb}{0.46,0.45,0.48}
% \definecolor{ppurple}{rgb}{0.55,1,0,0.15}

\newcommand{\todo}[1]{\textcolor{pred}{(TODO: {#1})}}
\newcommand{\zy}[1]{\textcolor{pred}{(Ziyi: {#1})}}

\lstdefinelanguage{SynLang}{
  keywords={new, let, if, else, null, return, while},
  ndkeywords={bool, int, void, loc},
  mathescape=true,
  showspaces=false,
  showtabs=false,
  breaklines=true,
  showstringspaces=false,
  breakatwhitespace=true,
  lineskip=-0.9pt,
  morecomment=[l]{//}, % l is for line comment
  morecomment=[s]{/*}{*/}, % s is for start and end delimiter
  basewidth={0.54em, 0.4em},%
  basicstyle=\footnotesize\ttfamily,
  keywordstyle={\color{pred}\ttfamily\bfseries},
  ndkeywordstyle={\color{pblue}\ttfamily\bfseries},
  commentstyle={\color{ccomment}\itshape},
  numbers=none,
}



\newcommand{\pts}{\mapsto}
\newcommand{\code}[1]{\lstinline[language=SynLang,basicstyle=\small\ttfamily,mathescape=true]{#1}}

\newcommand{\rulename}[1]{\textsc{#1}}
\newcommand{\func}{\textit{func}\xspace}
\newcommand{\writer}{\rulename{Write}\xspace}
\newcommand{\writeri}{\rulename{WriteRO}\xspace}
\newcommand{\funcwrite}{\rulename{Funcwrite}\xspace}
\newcommand{\inplace}{\textit{inplace}\xspace}
\newcommand{\tempfuncalloc}{\rulename{Tempfuncalloc}\xspace}
\newcommand{\temppredalloc}{\rulename{Temppredalloc}\xspace}
\newcommand{\tempfuncfree}{\rulename{Tempfuncfree}\xspace}
\newcommand{\temppredfree}{\rulename{Temppredfree}\xspace}
\newcommand{\inplacewrite}{\rulename{Inplacewrite}\xspace}



% \newcommand{\mcode}[1]{{\ensuremath{\tt #1}}}
\newcommand{\mcode}[1]{{\ensuremath{#1}}}
\newcommand{\vars}[1]{\mathsf{Vars}\left({#1}\right)}

\newcommand{\set}[1]{\left\{{#1}\right\}}
\newcommand{\bl}[1]{{\color{pblue}{#1}}}
\newcommand{\asn}[1]{{\bl{\set{#1}}}}

\newcommand{\trans}[3]{\left.\bl{#1} \!\leadsto\! \bl{#2} \right| #3}
\newcommand{\translong}[3]{\left.\bl{#1} \!\leadsto\! \bl{#2} \right| #3}
\newcommand{\prog}{c}
\newcommand{\osep}{\ast}
\newcommand{\env}{\Gamma}
\newcommand{\deref}[1]{*{#1}}

\newcommand{\iann}[1]{\mcode{ {#1}}} %immutability annotation
\newcommand{\writeontop}[2]{\mathrel{\stackrel{\makebox[0pt]{\mbox{\tiny{#2}}}}{#1}}}
\newcommand{\spointsto}[2]{{#1}~{\pts}~{#2}}
\newcommand{\opointsto}[2]{\spointsto{\angled{#1}}{#2}}
\newcommand{\ispointsto}[3]{#1{\writeontop{\ \!\pts\ \!}{\iann{#3}}}{#2}} % simple points-to, no angled parentheses


% From https://tex.stackexchange.com/questions/502652/define-tcolorbox-in-math-mode
\newtcbox{\boxtext}{on line,colback=white,colframe=black,size=fbox,arc=3pt,boxrule=0.8pt}
\newcommand{\boxmath}[1]{\boxtext{$#1$}}

% https://tex.stackexchange.com/questions/24132/overline-outside-of-math-mode
\makeatletter
\newcommand*{\textoverline}[1]{$\overline{\hbox{#1}}\m@th$}
\makeatother




%%
%% \BibTeX command to typeset BibTeX logo in the docs
\AtBeginDocument{%
  \providecommand\BibTeX{{%
    Bib\TeX}}}

%% Rights management information.  This information is sent to you
%% when you complete the rights form.  These commands have SAMPLE
%% values in them; it is your responsibility as an author to replace
%% the commands and values with those provided to you when you
%% complete the rights form.
\setcopyright{acmcopyright}
\copyrightyear{2018}
\acmYear{2018}
\acmDOI{XXXXXXX.XXXXXXX}


%%
%% These commands are for a JOURNAL article.
\acmJournal{JACM}
\acmVolume{37}
\acmNumber{4}
\acmArticle{111}
\acmMonth{8}

%%
%% Submission ID.
%% Use this when submitting an article to a sponsored event. You'll
%% receive a unique submission ID from the organizers
%% of the event, and this ID should be used as the parameter to this command.
%%\acmSubmissionID{123-A56-BU3}

%%
%% For managing citations, it is recommended to use bibliography
%% files in BibTeX format.
%%
%% You can then either use BibTeX with the ACM-Reference-Format style,
%% or BibLaTeX with the acmnumeric or acmauthoryear sytles, that include
%% support for advanced citation of software artefact from the
%% biblatex-software package, also separately available on CTAN.
%%
%% Look at the sample-*-biblatex.tex files for templates showcasing
%% the biblatex styles.
%%

%%
%% The majority of ACM publications use numbered citations and
%% references.  The command \citestyle{authoryear} switches to the
%% "author year" style.
%%
%% If you are preparing content for an event
%% sponsored by ACM SIGGRAPH, you must use the "author year" style of
%% citations and references.
%% Uncommenting
%% the next command will enable that style.
%%\citestyle{acmauthoryear}

%%
%% end of the preamble, start of the body of the document source.
\begin{document}

%%
%% The "title" command has an optional parameter,
%% allowing the author to define a "short title" to be used in page headers.
\title{Compilation With Program Synthesis}

% %%
% %% The "author" command and its associated commands are used to define
% %% the authors and their affiliations.
% %% Of note is the shared affiliation of the first two authors, and the
% %% "authornote" and "authornotemark" commands
% %% used to denote shared contribution to the research.
% \author{Ben Trovato}
% \authornote{Both authors contributed equally to this research.}
% \email{trovato@corporation.com}
% \orcid{1234-5678-9012}
% \author{G.K.M. Tobin}
% \authornotemark[1]
% \email{webmaster@marysville-ohio.com}
% \affiliation{%
%   \institution{Institute for Clarity in Documentation}
%   \streetaddress{P.O. Box 1212}
%   \city{Dublin}
%   \state{Ohio}
%   \country{USA}
%   \postcode{43017-6221}
% }
%
% \author{Lars Th{\o}rv{\"a}ld}
% \affiliation{%
%   \institution{The Th{\o}rv{\"a}ld Group}
%   \streetaddress{1 Th{\o}rv{\"a}ld Circle}
%   \city{Hekla}
%   \country{Iceland}}
% \email{larst@affiliation.org}
%
% \author{Valerie B\'eranger}
% \affiliation{%
%   \institution{Inria Paris-Rocquencourt}
%   \city{Rocquencourt}
%   \country{France}
% }
%
% \author{Aparna Patel}
% \affiliation{%
%  \institution{Rajiv Gandhi University}
%  \streetaddress{Rono-Hills}
%  \city{Doimukh}
%  \state{Arunachal Pradesh}
%  \country{India}}
%
% \author{Huifen Chan}
% \affiliation{%
%   \institution{Tsinghua University}
%   \streetaddress{30 Shuangqing Rd}
%   \city{Haidian Qu}
%   \state{Beijing Shi}
%   \country{China}}
%
% \author{Charles Palmer}
% \affiliation{%
%   \institution{Palmer Research Laboratories}
%   \streetaddress{8600 Datapoint Drive}
%   \city{San Antonio}
%   \state{Texas}
%   \country{USA}
%   \postcode{78229}}
% \email{cpalmer@prl.com}
%
% \author{John Smith}
% \affiliation{%
%   \institution{The Th{\o}rv{\"a}ld Group}
%   \streetaddress{1 Th{\o}rv{\"a}ld Circle}
%   \city{Hekla}
%   \country{Iceland}}
% \email{jsmith@affiliation.org}
%
% \author{Julius P. Kumquat}
% \affiliation{%
%   \institution{The Kumquat Consortium}
%   \city{New York}
%   \country{USA}}
% \email{jpkumquat@consortium.net}
%
% %%
% %% By default, the full list of authors will be used in the page
% %% headers. Often, this list is too long, and will overlap
% %% other information printed in the page headers. This command allows
% %% the author to define a more concise list
% %% of authors' names for this purpose.
% \renewcommand{\shortauthors}{Trovato et al.}

%%
%% The abstract is a short summary of the work to be presented in the
%% article.
\begin{abstract}
  Synthetic Separation Logic (SSL) is a specification language used by the \SuSLik{} program synthesis tool.
  \SuSLik{} will take a Hoare-style specification expressed in SSL and synthesize a program that satisfies this
  specification in a C-like language. However, these specifications are very low-level and SSL does not allow for much abstraction. In this paper,
  we explore a connection between SSL specifications and functional programming. We give a first-order functional
  programming language, \Pika, which has both a standard semantics and a translation into SSL that respects the standard semantics.

  Along the way, we encounter limitations in the \SuSLik{} tool. In confronting these limitations, we develop a new feature
  in \Pika{} that allows the programmer to express synthesis tasks at a higher-level.

  % A clear and well-documented \LaTeX\ document is presented as an
  % article formatted for publication by ACM in a conference proceedings
  % or journal publication. Based on the ``acmart'' document class, this
  % article presents and explains many of the common variations, as well
  % as many of the formatting elements an author may use in the
  % preparation of the documentation of their work.
\end{abstract}

%%
%% The code below is generated by the tool at http://dl.acm.org/ccs.cfm.
%% Please copy and paste the code instead of the example below.
%%
\begin{CCSXML}
% <ccs2012>
%  <concept>
%   <concept_id>00000000.0000000.0000000</concept_id>
%   <concept_desc>Do Not Use This Code, Generate the Correct Terms for Your Paper</concept_desc>
%   <concept_significance>500</concept_significance>
%  </concept>
%  <concept>
%   <concept_id>00000000.00000000.00000000</concept_id>
%   <concept_desc>Do Not Use This Code, Generate the Correct Terms for Your Paper</concept_desc>
%   <concept_significance>300</concept_significance>
%  </concept>
%  <concept>
%   <concept_id>00000000.00000000.00000000</concept_id>
%   <concept_desc>Do Not Use This Code, Generate the Correct Terms for Your Paper</concept_desc>
%   <concept_significance>100</concept_significance>
%  </concept>
%  <concept>
%   <concept_id>00000000.00000000.00000000</concept_id>
%   <concept_desc>Do Not Use This Code, Generate the Correct Terms for Your Paper</concept_desc>
%   <concept_significance>100</concept_significance>
%  </concept>
% </ccs2012>
\end{CCSXML}

% \ccsdesc[500]{Do Not Use This Code~Generate the Correct Terms for Your Paper}
% \ccsdesc[300]{Do Not Use This Code~Generate the Correct Terms for Your Paper}
% \ccsdesc{Do Not Use This Code~Generate the Correct Terms for Your Paper}
% \ccsdesc[100]{Do Not Use This Code~Generate the Correct Terms for Your Paper}

% %%
% %% Keywords. The author(s) should pick words that accurately describe
% %% the work being presented. Separate the keywords with commas.
% \keywords{Do, Not, Us, This, Code, Put, the, Correct, Terms, for,
%   Your, Paper}

% \received{20 February 2007}
% \received[revised]{12 March 2009}
% \received[accepted]{5 June 2009}

%%
%% This command processes the author and affiliation and title
%% information and builds the first part of the formatted document.
\maketitle

\section{Introduction}

Synthetic separation logic (SSL) is the specification language for the \SuSLik{} program synthesizer. \SuSLik{} will take a precondition and postcondition pair together
with a function prototype and use these to synthesize a function that satisfies that specification in a C-like language called \SuSLang.

We have developed a first-order functional programming language, \Pika, which will compile to SSL specifications. \SuSLik{} can then take these SSL specifications and generate \SuSLang. Finally, the resulting \SuSLang{} code can be converted to C and then run. In addition to this translational semantics, we also provide a standard abstract machine semantics for \Pika{} and use this to show correctness of the translation into SSL.

\paragraph{Contributions} In this paper, we make the following contributions:
\begin{itemize}
  \item We explore the connection between SSL and functional programming using a pair of formal semantics for \Pika: A standard abstract machine semantics and a
    semantics that translates \Pika{} into SSL specifications. The second semantics is shown to be sound relative to the first semantics. (Section~\ref{sec:semantics})

  \item In order to provide seamless interoperability between SSL and \Pika, we provide a separation between algebraic datatypes and their representation in memory.
    At a type-level, we track this representation in memory. Furthermore, we describe a kind of polymorphism over memory layouts of an algebraic datatype. (Section~\ref{sec:layouts})

  \item We address the limitations of this approach. In particular, we develop an extension of \Pika{} that allows \SuSLik's synthesis capabilities to be exposed more naturally inside the \Pika{} language. (Section~\ref{sec:synth})

  \item A comparison of examples that work on the first version of \Pika{} is compared with examples that are enabled by the new feature in the second version of \Pika. This new version of \Pika{} is not backwards compatible with the first version, so we also examine which examples in the old version no longer work in the new version. (Section~\ref{sec:examples})
\end{itemize}

\section{Overview}
\label{sec:overview}

\subsection{Syntax}
\label{sec:syntax}

\subsection{Layouts}
\label{sec:layouts}

\section{Formal Semantics}
\label{sec:semantics}

\section{Synthesis Limitations And Solutions}
\label{sec:limitations}

There is a significant limitation to our approach so far: \SuSLik will often fail to synthesize code from valid specifications. One instance of this problem is the following: In a certain class of functions, \SuSLik{} requires a different number of pointer indirections than it will with other functions. We have not found a way to perfectly characterize such functions and, since this will change the way the functions will need to be called, this represents a major impediment to our approach. It is worth noting that this is also not the \textit{only} situation in which \SuSLik{} fails to synthesize a program.

From this, we draw the conclusion that \SuSLik{} is not well-suited to this task. In fact, this task is very different from the original goals of \SuSLik, which could be stated: \textit{Given a potentially ambiguous specification, generate any program that satisfies this specification.} However, in our case, we already know which specific program we want to generate. This is fully specified by the abstract machine semantics for \Pika.

With this in mind, we take a new approach. What if we compile \Pika{} directly to C, but we add a \textit{new} feature that allows us to express synthesis problems at the level of \Pika{} code? This takes the form of the \verb|synth| keyword, which we explain in detail in Section~\ref{sec:synth}. In this case, we can get the advantages of a traditional compiler while also exploiting the synthesis capabilities of \SuSLik. Furthermore, due to \Pika's design, we already have easy interoperability with \SuSLik. Also, recall that \SuSLik's target language, \SuSLang, can be straightforwardly translated directly to C. This allows us to easily bring all of these things together in our new implementation with the new \Pika{} synthesis feature.

\section{\synth{} Keyword: Synthesis Inside \Pika}
\label{sec:synth}

Next, we introduce a feature that allows us to synthesize a program from a special kind of \Pika{} type signature. This special kind of type signature allows us to directly access \SuSLik's pure propositions. These type signatures are called \synth-signatures and the syntax for invoking the \Pika{} synthesis on a \synth-signature is:
\[
  \synth\; f : \sigma
\]
\noindent
where $\sigma$ is a \synth-signature. The syntax of a \synth-signature is given in Figure~\ref{fig:synth-sig-syntax}.

\begin{figure}
\[
\begin{aligned}
  \textit{\synth-signatures}\;\; \sigma &::= p \synthsep \tau'\\
  \textit{Pure propositions}\;\; p &::= x \mid i \mid b \mid p \mathbin{\&\&} p \mid p \mathbin{||} p
    \mid \Not(p) \mid p == p \mid \{\} \mid \{p\} \mid p \app p \mid p + p \mid p - p\\
  \textit{Extended types}\;\; \tau' &::= ... \mid A\; \overline{G}\\
  \textit{Ghost variables}\;\; G &::= @x
\end{aligned}
\]
  \caption{\synth-signature syntax}
  \label{fig:synth-sig-syntax}
\end{figure}

\noindent
We also need to extend layout definitions so that they can include "ghost parameters." These parameters will directly correspond to ghost variable parameters in the resulting \SuSLik{} specification.

This extension to layouts is best explain by way of example. First, consider the original definition of the singly-linked list type:

\begin{lstlisting}
Sll : layout[x](List)
Sll Nil := emp;
Sll (Cons head tail) :=
  x :-> head ** (x+1) :-> nxt
    ** Sll tail [nxt];
\end{lstlisting}

\noindent
This does not use any ghost variables in \Pika{} or in the resulting \SuSLik{} specification. A common pattern in \SuSLik{} specifications is to use a \verb|set| type ghost variable to keep track of the elements in a data structure. Then, we can require that the elements are preserved or that new elements be added when using this in specifications. The feature we add to layouts allows us to use this feature directly:

\begin{lstlisting}
Sll : @(s : set) layout[x](List);
Sll Nil := s == {} ;; emp;
Sll (Cons head tail) :=
  s == {head} ++ s1
    ;;
  x :-> head ** (x+1) :-> nxt
    ** Sll tail @s1 [nxt];
\end{lstlisting}

\noindent
Putting this all together, we can use this new kind of parameter together with the \synth{} keyword to specify a list append function:

\begin{lstlisting}
synth append :
  s == s1 ++ s2
    ;;
  Sll @s1 -> Sll @s2 -> Sll @s;
\end{lstlisting}

\noindent
Using this feature, we can use \SuSLik{} specifically for \textit{synthesis} tasks inside \Pika. That is, we are able to directly use \SuSLik{} for its intended purpose.

\section{Examples}
\label{sec:examples}

\subsection{Version 1}

\subsection{Version 2}

\section{Future Work}
\label{sec:future-work}

\section{Conclusion}
\label{sec:conclusion}

\end{document}
\endinput
%%
%% End of file `sample-acmsmall.tex'.
