\section{Formal Semantics}
\label{sec:semantics}


% \subsection{Formal Semantics of \tool}
% \label{sec:formal}

In this section, we have the following plan:

\begin{itemize}
  \item We define abstract machine semantics for executing a subset of \tool programs. This semantics
    is given by the big-step relation $\step$ which we define later.
  \item We define the translation from that subset of \tool into SSL. This translation is given by the
    function $\Tsem{e}{V,r}$ from \tool expressions into SSL propositions. The $r$ is a variable name to
    be used in the resulting proposition and $V$ is a collection of fresh names.
  \item We prove a soundness theorem. Given any well-typed expression $e$ and an abstract machine reduction producing
    the store-heap pair $\EndPair$, the SSL translation of $e$ should be satisfied by SSL model $\EndPair$. This is stated formally, and proven,
    in \autoref{thm:gen-soundness}.
  % \item We prove a soundness theorem. In particular, we first look at the heap-store pair that results from executing the abstract
  % machine on an arbitrary function application expression \verb|f e| (omitting the \verb|instantiate| for brevity).
  % We then show that the heap-store pair we got from the abstract machine is a model for the 
  % SSL proposition given by translating \verb|f e| into SSL.
\end{itemize}

This subset of \tool does not have guards or \verb|if then else| expressions, but it does have pattern
matching. It also has the requirement that functions can only have one argument. Unlike the implementation, there
is no elaboration. As a result, every algebraic data type value must be lowered to a specific layout at every usage
and every function application must be explicitly instantiated with a layout for the argument and a layout for the result. We
also limit the available integer and Boolean operations for brevity.

The grammar for this subset is given in \autoref{fig:subset-grammar}. The grammar for types, layout definitions and algebraic data type definitions remain the same
as before and are therefore omitted from the following grammar.


\begin{figure}
\begin{grammar}
  <i> ::= $\cdots$ $\mid$ -2 $\mid$ -1 $\mid$ 0 $\mid$ 1 $\mid$ 2 $\mid$ $\cdots$

  <b> ::= \texttt{true} $\mid$ \texttt{false}

  <e> ::= <var> \alt <i> \alt <b> \alt <e> + <e> \alt C $\overline{\langle e \rangle}$ \alt $\instantiateS{A,B}{f}(\langle \textit{e} \rangle)$ \alt $\lowerS{A}{\langle \textit{e} \rangle}$

  <fn-def> ::= $\overline{\langle \textit{fn-case} \rangle}$

  <fn-case> ::= f <pattern> := <e>
\end{grammar}
  \caption{Grammar for restricted \tool subset}
  \label{fig:subset-grammar}
\end{figure}

\noindent
The semantics for SSL are largely derived~\cite{polikarpova:2019:suslik} from standard separation logic semantics.~\cite{rowe:2017:auto-cyclic-term}

\subsection{Overview of the Two Interpretations}

The soundness theorem will link the abstract machine semantics to the translation. In fact, the abstract machine
semantics and the translation are similar to each other. For the abstract machine we manipulate \textit{concrete} heaps, while
for the translation we generate \textit{symbolic} heaps.

Comparing the two further, there are two main points (beyond what we've already mentioned) where these two interpretations of \tool differ:

% \tbox{
    % \begin{center} \textbf{Two main differences between interpretations} \end{center}

      \begin{enumerate}
        \item When we need to unfold a layout, how do we know which layout branch to choose?
        \item How do we translate function applications (including, but not limited to, recursive applications)?
      \end{enumerate}
  % }
% \\

% \subsubsection{Abstract Machine}
\paragraph{Abstract machine}
First, consider the abstract machine semantics. In this case, we are able to choose which branch of a layout to use by evaluating the expression we are applying it
          to until the expression is reduced to a constructor value (where a ``constructor value'' is either a value or a constructor applied to constructor values). If the expression is well-typed, this will always be a constructor
          of the algebraic data type corresponding to the layout. The two rules that this applies to are {\sc AM-Lower} and
          {\sc AM-Instantiate}.
        To interpret a function application, we interpret its arguments and substitute the results into the body
          of the function. We then proceed to interpret the substituted function body. This process is performed by the {\sc AM-Instantiate} rule.
  % }
% \\

% \subsubsection{SSL Translation}
\paragraph{SSL translation}
Next, consider the SSL translation. Here we can determine which layout branch to use by generating a Boolean condition that will be
          true if and only if the SSL proposition on the right-hand side of the branch holds for the heap. Note
          that we assume that the programmer-supplied layout definitions are \textit{injective} functions from
          algebraic data types to SSL assertions (up to bi-implication).
We can directly translate function applications into SSL inductive predicate applications. Since inductive predicates
          already allow for recursive applications, there is no special handling necessary for recursion.
      % \end{enumerate}
  % }
% \\

% We will return to these points when we talk about each of the two interpretations.
%
% The differences on the first point, again, results primarily from the difference between concrete heaps and symbolic heaps.
%
% For the second
% point, on the other hand, there is an additional reason for the differences. Both \tool and SSL have a notion of application. In the former,
% this is function application and in the latter it is predicate application. In both cases, these applications can be recursive. Since
% \tool functions are translated into inductive predicates, function applications are translated into predicate applications. As SSL has
% recursion, we do not need to handle this specially.
%
% For the abstract machine, in contrast, we must actually \textit{run} the body of the function.

After defining these interpretations, we show how the abstract machine relation $\step$ and the SSL interpretation function $\Tsem{e}{V,r}$ relate to each other by the Soundness \autoref{thm:gen-soundness}.

% We sketch the
% statement of the theorem here, leaving the details for later. In particular, the part marked with question marks is filled in later. We
% show the {\color{cyan} start state} in {\color{cyan} cyan} and the {\color{orange} end state} in {\color{orange} orange}.
%
%
%   \[
%     \begin{aligned}
%     &\begin{aligned}
%       (\fbox{      ???      }
%       &\land (\exists \VarSet, r.\; ({\color{cyan} \sigma}, {\color{cyan} h}) \models \Tsem{e}{\VarSet, r})\\
%       &\land (f_{A,B}(e), {\color{cyan} \sigma}, {\color{cyan} h}) \step (e_2, {\color{orange} \sigma'}, {\color{orange} h'}, v'))
%      \end{aligned}
%     \\
%         & \implies
%       (\exists \VarSet, r.\; ({\color{orange} \sigma'}, {\color{orange} h'}) \models \Tsem{f_{A,B}(e)}{\VarSet, r})
%     \end{aligned}
%   \]
%
% At a high-level, the theorem says: if we start in a machine state which is a model for the SSL interpretation of a given expression $e$ (this SSL proposition is given by $\Tsem{e}{\VarSet,r}$), then
% any state that the abstract machine steps to from the original state, given an application expression $f_{A,B}(e)$, is a model for the SSL interpretation
% $\Tsem{f_{A,B}(e)}{\VarSet,r}$. This is subject to the condition that there exists \textit{some} model for $\Tsem{f_{A,B}(e)}{\VarSet,r}$. The reason for this requirement
% is that the SSL interpretation of an arbitrary \tool expression is not guaranteed to be satisfiable. There are further restrictions
% that will be given when we fill in the question marks, but this is deferred to a later subsection.


\subsection{Abstract Machine Semantics}
In this section, we will define an abstract machine semantics for \tool and relate this to
the standard semantics for SSL.

% \subsubsection{Notation and Setup}

% The only difference between the two sorts of values is that the set of \tool values also includes
% constructor application values, while \BaseVal{} does not.

The set of values is $\Val = \mathbb{Z} \cup \mathbb{B} \cup \Loc$. Each of these three sets
is disjoint from the other two. In particular, note that $\Loc$ and $\mathbb{Z}$ are disjoint.

\begin{figure}
\[
\begin{array}{c}
  \labinfer[\FsVal-Base]{x \in \FsVal}
    {x \in \Val}
  ~~~
  \labinfer[\FsVal-Constr]{(C\; x_1 \cdots x_n) \in \FsVal}
    {x_1 \in \FsVal & \cdots & x_n \in \FsVal}
\end{array}
\]
  \caption{\FsVal{} judgment rules}
  \label{fig:FsVal-rules}
\end{figure}

There is also a set of \tool values \FsVal. This includes all the elements of \Val{}, but also includes
``constructor values'' given by the rules in \autoref{fig:FsVal-rules}.
In addition to the store and heap of standard SSL semantics, the abstract machine semantics uses an \FsStore. This is a partial function from locations
to \tool values: $\FsStore = \Loc \rightharpoonup \FsVal$. The primary purpose of this is to recover constructor values when given a location.

%\noindent
The general format of the transition relation is $(e, \sigma, h, \mathcal{F}) \step (v, \sigma', h', \mathcal{F}', r)$, where the expression $e$ results
in the store being updated from $\sigma$ to $\sigma'$, the heap being updated from $h$ to $h'$, $v$ is the \Val{} obtained by
evaluating $e$, the initial and final $\FsStore$ are $\mathcal{F}$ and $\mathcal{F}'$ and the result is stored in variable $r$. We assume that there is a global environment $\Sigma$ which contains all layout definition equations and function definition equations.
% We will have the following properties for the \FsStore:
%
% % \begin{lemma}
% %   $(e, \sigma, h, \mathcal{F}) \step (v, \sigma', h', \mathcal{F}', r) \implies (dom(h) = dom(\mathcal{F}) \land dom(h') = dom(\mathcal{F}'))$
% % \end{lemma}
%
% \begin{lemma}
%   $(e, \sigma, h, \mathcal{F}) \step (v, \sigma', h', \mathcal{F}', r) \implies (v \in dom(h') \land v \in \Loc \land h'(v) \in \Loc \implies h'(v) \in dom(\mathcal{F}'))$
% \end{lemma}

% Describing this relation more explicitly, we have $\step\; \subset (\Expr \times \Store \times \Heap \times \FsStore) \times (\Val \times \Store \times \Heap \times \times \FsStore \times \Var)$.

% \noindent
% The syntax $dom(\sigma)$ allows us to refer to the set of names in a store. It is defined as
% \[
%   dom(\sigma) = \{ v \mid (v, a) \in \sigma \}
% \]

Given a heap $h$ and a layout body $H$, we will make use of the
notation $h \cdot [H]$. This extends the heap with the location
assignments given in $H$. We say that the layout body $H$ is
\textit{acting} on the heap $h$. This is defined in
\autoref{fig:layout-act}. The intuition for this is that $h$ gets
updated using the symbolic heap description in $H$. For example, if
$h$ is the empty heap and $H$ is $a \pointsto 7$, then the updated
heap $h \cdot [H]$ will contain only the value $7$ at the location
$a$. It is assumed that $H$ does not have any variables on the
right-hand side of $\pointsto$.

\begin{figure}
\[
\begin{array}{c}
  \labinfer[L-Emp]{h \cdot [\emp] = h}
    {}
  ~~~
  \labinfer[L-PointsTo]{h \cdot [\ell \pointsto a, \kappa] = h'[\ell \mapsto a]}
    {h' = h \cdot [\kappa] & a \in \Val}
  \\\\
  \labinfer[L-Apply]{h \cdot [A[x](e), \kappa] = h \cdot [\kappa]}
    {e \in \Val}
\end{array}
\]
  \caption{Rules for layout bodies acting on heaps}
  \label{fig:layout-act}
\end{figure}

% \subsubsection{Abstract Machine Rules}
% The rules for the abstract machine transition relation are as follows.

The abstract machine semantics provides big-step operational semantics
for evaluating \tool expressions on a heap machine. Its rules, given by
\autoref{fig:abs-machine}, make use of standard SSL models:

\begin{align*}
  &\text{Model} &\mathcal{M} ::= (\sigma, h)\\
  &\text{Store} &\sigma : \text{Var} \rightharpoonup \text{Val}\\
  &\text{Heap} &h : \text{Loc} \rightharpoonup \text{Val}\\
\end{align*}

\noindent
Note that a compound expression, consisting of multiple
subexpressions, uses \emph{disjoint} parts of the heap for each
subexpression. This can be seen in the {\sc AM-Add}, {\sc AM-Lower}
and {\sc AM-Instantiate} rules, which is essential for the proof of
Soundness \autoref{thm:gen-soundness}.

\begin{figure}
{
\[
\begin{array}{c}
  \labinfer[AM-Int]{(i, \sigma, \emptyheap, \mathcal{F}) \step (i, \sigma', \emptyheap, \mathcal{F}, r) }
    {\freshVar{r} & i \in \mathbb{Z} & \sigma' = \sigma \cup \{ (r, i) \}}
  ~~~
  \labinfer[AM-Bool]{(b, \sigma, \emptyheap, \mathcal{F}) \step (b, \sigma', \emptyheap, \mathcal{F}, r) }
    {\freshVar{r} & b \in \mathbb{B} & \sigma' = \sigma \cup \{ (r, b) \}}
  \\\\
  \labinfer[AM-Var-Base]{(v, \sigma, \emptyheap, \mathcal{F}) \step (\sigma(v), \sigma, \emptyheap, \mathcal{F}, v)}
    {v \in dom(\sigma) & \sigma(v) \not\in \Loc}
  \\\\
  \labinfer[AM-Var-Loc]{(v, \sigma, \emptyheap, \mathcal{F}) \step (\mathcal{F}(\sigma(v)), \sigma, \emptyheap, \mathcal{F}, v)}
    {v \in dom(\sigma) & \sigma(v) \in \Loc}
  \\\\
  % \labinfer[AM-Constr]{(C\; e_1 \cdots e_n, \sigma, h, \mathcal{F}) \step (C\; e_1 \cdots e_n, \sigma, h, \mathcal{F})}
  %   {{\color{red} TODO} \textnormal{fix this}}
  % \\\\
  \labinfer[AM-Add]{(x + y, \sigma, h, \mathcal{F}) \step (z, \sigma', h, \mathcal{F}, r)}
    {\begin{gathered}
      (x, \sigma, \mathcal{F}, h_1) \step (x', \sigma_x, h_1', \mathcal{F}, v_x)
      ~~~ (y, \sigma_x, \mathcal{F}, h_2) \step (y', \sigma_y, h_2', \mathcal{F}, v_y)
    \\  \freshVar{r}
    ~~~ h = h_1 \circ h_2
    ~~~ h' = h_1' \circ h_2'
    ~~~ z = x' + y'
    ~~~ \sigma' = \sigma_y \cup \{ (r, z) \}
     \end{gathered}
    }
  \\\\
  \labinfer[AM-Lower]{(\lowerS{A}{e}, \sigma_0, h, \mathcal{F}) \step ((C\; e_1' \cdots e_n'), \sigma', h' \cdot [H'], \mathcal{F}', r)}
     {\begin{gathered}
          (A[x]\; (C\; a_1 \cdots a_n) := H) \in \Sigma
       ~~~ (e, \sigma_0, h_0) \step (C\; e_1 \cdots e_n, \sigma_1, h_1, y_1)
       \\  (e_i, \sigma_i, h_i) \step (e_i', \sigma_{i+1}, h_{i}', v_i)\; \textnormal{for each $1 \le i \le n$}
       \\ h' = h_1' \circ h_2' \circ \cdots \circ h_n'
       ~~~ h = h_0 \circ h_1 \circ \cdots \circ h_n
       ~~~ \sigma' = \sigma_{n+1} \cup \{ (r, \ell) \}
       \\  \freshLoc{\ell}
       ~~~ \freshVar{r}
       \\  H' = H[x := \ell][a_1 := \sigma_2(v_1)][a_2 := \sigma_3(v_2)]\cdots[a_n := \sigma_{n+1}(v_n)]
       \\  \mathcal{F}' = \mathcal{F} \cup \{ (\ell, (C\; e_1' \cdots e_n')) \}
     \end{gathered}
    }
  \\\\
  \labinfer[AM-Instantiate]{(\instantiateS{A,B}{f}(e), \sigma, \mathcal{F}, h) \step (e_f', \sigma', h'', \mathcal{F}', r)}
    {\begin{gathered}
        (A[x]\; (C\; a) := H) \in \Sigma
    ~~~ (f\; (C\; b) := e_f) \in \Sigma
    \\  (e, \sigma, h) \step (C\; e_1, \sigma_1, h_1, y)
    \\  (e_1, \sigma_1, h_1) \step (e_1', \sigma_2, h_2, r)
    \\ \freshLoc{\ell}
    ~~~ \freshVar{r}
      ~~~ \freshVar{y}
    \\  H' = H[x := \ell][a := e_1']
    ~~~  h' = h_1 \cdot [H']
    ~~~ \sigma' = \sigma_f \cup \{ (r, \ell) \}
    \\  \mathcal{F}' = \mathcal{F} \cup \{ (\ell, (C\; e_1')) \}
    \\  (\lowerS{B}{e_f[b := y]}, \sigma_2, h') \step (e_f', \sigma_f, h'', r)
     \end{gathered}
    }
\end{array}
\]
}

%   % \labinfer[AM-Constr]{(C\; e_1 \cdots e_n, \sigma, h, \mathcal{F}) \step (C\; e_1 \cdots e_n, \sigma, h, \mathcal{F})}
%   %   {{\color{red} TODO} \textnormal{fix this}}
% \[
% \begin{array}{c}
%   \labinfer[$\mathfrak{C}$-Constr]{(C\; e_1 \cdots e_n, \sigma, h, \mathcal{F}) \Cstep (C\; e_1 \cdots e_n, \sigma, h, \mathcal{F})}
%     {}
%   \\\\
%   \labinfer[$\mathfrak{C}$-NonConstr]{(e, \sigma, h, \mathcal{F}) \Cstep (e', \sigma', h', \mathcal{F}')}
%     {(e, \sigma, h, \mathcal{F}) \step (e', \sigma', h', \mathcal{F}', r)}
% \end{array}
% \]

  \caption{Abstract machine semantics rules}
  \label{fig:abs-machine}
\end{figure}

% Reviewing the two points of difference between the interpretations mentioned at the start of the section:

\subsection{Translating \tool Specifications into SSL}

We will define two translations: One from \tool \textit{expressions} into SSL propositions and the other from
\tool \textit{definitions} into SSL inductive predicate definitions. We start with the former.

\subsubsection{Translating Expressions}

In the rules given in \autoref{fig:expr-rules}, the notation $\mathcal{I}_{A,B}(f)$ gives the name of the inductive predicate that the \tool
function $f$ translates to when it is instantiated to the layouts $A$ and $B$.

We start by defining the translation rules for expressions. We use these translation rules in \autoref{thm:ssem-fn} to
define a translation function $\Tsem{\cdot}{\VarSet,r}$. Then, we will define translation rules for function definitions.

The translation relation for expressions has the form $(e, \VarSet) \tstep (p, s, \VarSet', v)$,
where $p$ and $s$ are the pure part and spatial part (respectively) of an SSL assertion and $\VarSet,\VarSet' \in \mathcal{P}(\Var)$.

\begin{figure}
  % \resizebox{.9 \textwidth}{!}
  {
\[
\begin{array}{c}
  \labinfer[S-Int]{(i, \VarSet) \tstep (v == i, \emp, \VarSet \cup \{v\}, v)}
    {i \in \mathbb{Z} &
     \freshVar{v}}
  \\\\
  \labinfer[S-Bool]{(b, \VarSet) \tstep (v == b, \emp, \VarSet \cup \{v\}, v)}
    {b \in \mathbb{B} &
     \freshVar{v}
    }
  \\\\
  \labinfer[S-Var]{(v, \VarSet) \tstep (\texttt{true}, \emp, \VarSet, v)}
    {v \in \Var}
  \\\\
  \ruleSAdd
  \\\\
  \labinfer[S-Lower-Var]{(\lowerS{A}{v}, \VarSet) \tstep (\texttt{true}, A(v), \VarSet \cup \{v\}, v)}
    {\begin{gathered}
      v \in \Var
     \end{gathered}
    }
  \\\\
  \labinfer[S-Lower-Constr]{
        (\lowerS{A}{C\; e_1 \cdots e_n}, \VarSet_1)
          \tstep (p_1 \land \cdots \land p_n, H' \sep s_1 \sep \cdots \sep s_n, \VarSet', x)}
    {\begin{gathered}
          (A[x]\; (C\; a_1 \cdots a_n) := H) \in \Sigma
      \\  (e_i, \VarSet_i) \tstep (p_i, s_i, \VarSet_{i+1}, v_i)\; \textnormal{for each $1 \le i \le n$}
      \\  \freshVar{v}
      \\  \VarSet' = \VarSet_{n+1} \cup \{x\}
      ~~~ H' = H[a_1 := v_1]\cdots[a_n := v_n]
     \end{gathered}
    }
  \\\\
  \labinfer[S-Inst-Var]{(\instantiateS{A,B}{f}(v), \VarSet) \tstep (\texttt{true}, \mathcal{I}_{A,B}(f)(v, r), \VarSet \cup \{r\}, r)}
    {v \in \VarSet & \freshVar{r}}
  \\\\
  \labinfer[S-Inst-Constr]{
        (\instantiateS{A,B}{f}(C\; e_1 \cdots e_n), \VarSet_1)
                                     \tstep (p \land p_1 \land \cdots \land p_n, s \sep s_1 \sep \cdots \sep s_n, \VarSet', r)
      }
    {\begin{gathered}
          (A[x]\; (C\; a_1 \cdots a_n) := H) \in \Sigma
      \\  (e_i, \VarSet_i) \tstep (p_i, s_i, \VarSet_{i+1}, v_i)\; \textnormal{for each $1 \le i \le n$}
      \\  \freshVar{x}
      \\  \VarSet' = \VarSet_{n+1} \cup \{x\}
      ~~~ H' = H[a_1 := v_1]\cdots[a_n := v_n]
      \\
          (f\; (C\; b_1 \cdots b_n) := e_f) \in \Sigma
      ~~~ e_f' = e_f[b_1 := v_1]\cdots[b_n := v_n]
      \\  (\lowerS{B}{e_f'}, \VarSet_{n+1}) \tstep (p, s, \VarSet', r)
     \end{gathered}
    }
  \\\\
  \labinfer[S-Inst-Inst]{
      (\instantiateS{B,C}{f}(\instantiateS{A,B}{g}(e)), \VarSet) \tstep (p_1 \land p_2, s_1 \sep s_2, \VarSet_2, r_2)
    }
    {\begin{gathered}
      (\instantiateS{A,B}{g}(e), \VarSet) \tstep (p_1, s_1, \VarSet_1, r_1)
      \\ (\instantiateS{B,C}{f}(r_1), \VarSet_1) \tstep (p_2, s_2, \VarSet_2, r_2)
     \end{gathered}
    }
\end{array}
\]
}
\caption{Expression Translation Rules}
\label{fig:expr-rules}
\end{figure}

\begin{lemma}[$\Tsem{\cdot}{}$ function] \label{thm:ssem-fn} $(\cdot, \VarSet) \tstep (\cdot, \cdot, \cdot, r)$ is a computable function $\Expr \rightarrow (\Pure \times \Spatial \times \mathcal{P}(\Var))$, given fixed
  $\VarSet$ and $r$ where $r \not\in \VarSet$.

By throwing away the third element of the tuple in the codomain, we obtain a function $\Expr \rightarrow (\Pure \times \Spatial)$ from expressions to
SSL propositions.

  Call this function $\Tsem{\cdot}{\VarSet, r}$. That is, we define the function as follows where $r \not\in \VarSet$:
  \[
    \Tsem{e}{\VarSet, r} = (p ; s) \iff (e, \VarSet) \tstep (p, s, \VarSet', r)\; \textnormal{for some $\VarSet'$}
  \]
\end{lemma}

We highlight the computability of this function to emphasise the fact
that it can be used directly in an implementation of this subset of
\tool.

\subsubsection{Translating Function Definitions}

The next step is to define the translation for \tool function definitions. In order to do this, we must first figure out how to determine the
appropriate layout branch to use when unfolding a layout, a problem we highlighted earlier. Once this is accomplished, the rest of
the translation can be defined. When this problem was solved for the abstract machine semantics,
it was possible to simply evaluate the \tool expression until it a constructor application expression was reached. From there, it is possible to just look at the constructor name and match it against the appropriate layout branch.

For the translation, however, we do not have the luxury of being able to evaluate expressions. Instead, we must instead rely on the fact
that, in SSL, a ``pure'' (Boolean) condition can determine which inductive predicate branch to use. The question becomes: \textit{Given an algebraic
data type and a layout for that ADT, how do we
generate an appropriate Boolean condition for a given constructor for the ADT}?

The solution is to find a Boolean condition which, given that the inductive predicate holds, is true \textit{if and only if} the layout branch corresponding to that
ADT constructor is satisfiable. In more detail, to define the branches of an inductive predicate $\mathcal{I}_{A}(x)$, given an ADT $\alpha$, a constructor $C : \beta_1 \rightarrow \beta_2 \rightarrow \cdots \rightarrow \beta_n \rightarrow \alpha$, a layout $A : \alpha \monic \layout[x]$ with a branch $A[x]\; (C\; a_1 \cdots a_n) := H$ and given that $\mathcal{I}_{A}(x)$ holds, find a Boolean expression $b$ with one free variable $x$ such that
\[
  b \iff \exists \sigma, h.\; (\sigma, h) \models H
\]

\begin{lemma}[$\cond$ function]
  \label{thm:cond}
  There is a computable function $\cond(\cdot,\cdot)$ that takes in any layout $A : \alpha \monic \layout[x]$ with a branch $A[x]\; (C\; a_1 \cdots a_n) := H$ for a given constructor
  $C : \beta_1 \rightarrow \beta_2 \rightarrow \cdots \rightarrow \beta_n \rightarrow \alpha$ and it produces a Boolean expression with one free variable $x$ such that the following holds under the assumption that $\mathcal{I}_{A}(x)$ holds.
  Note that $\mathcal{I}_{A}(x)$ is the name of the generated inductive predicate corresponding to the layout $A$.
  \[
    \cond(A, C) \iff \exists \sigma, h.\; (\sigma, h) \models H
  \]
\end{lemma}

With this function in hand, we are now ready to define the translation for \tool function definitions. This
definition is in \autoref{fig:FnDef-rule}.

\begin{figure}
\[
\begin{array}{c}
  \labinfer[FnDef]{(f\; (C\; a_1 \cdots a_n) := e) \defstep{A,B} (\mathcal{I}_{A,B}(f)(x, r) : c \Ra \{p ; s\})}
    {\begin{gathered}
          \VarSet = \{ v_1 , \cdots , v_n \}\; \textnormal{where $v_1, \cdots, v_n$ are distinct variables}
      \\  r \in \Var
      ~~~ r \not\in \VarSet
      ~~~ c = \cond(A, C, x)
      \\  (p, s) = \Tsem{\instantiateS{A,B}{f}(C\; p_1 \cdots p_n)}{\VarSet, r}
     \end{gathered}
    }
\end{array}
\]
\caption{Translation rule for function definitions}
\label{fig:FnDef-rule}
\end{figure}

% \noindent
% As mentioned before, we are looking at a subset of \tool where
% functions only have a single argument. However, the solution we have just
% described is the general solution that works for multiple arguments as well.

\subsection{Typing Rules}

Typing rules for \tool expressions are given in
\autoref{fig:typing-rules}. These rules differ from standard typing
rules for a functional language due to the existence of layouts and
their associated constructors, like \verb|instantiate| and
\verb|lower|. If an expression is well-typed, then each use of
\verb|instantiate| and \verb|lower| only uses layouts together with
the ADT that they are defined for.

The rules also make use of a \textit{concreteness judgment}. The rules
for this judgment are given in \autoref{fig:concreteness-rules}. The
intuition of this judgment is that a type is ``concrete'' iff values
of that type can be directly represented in the heap machine
semantics. For example, an ADT type is \textit{not} concrete because a
layout has not been specified. However, once a particular layout is
specified for the ADT type, it becomes concrete. Base types, like
\verb|Int|, are also concrete.

Rules for the ensuring that global definitions are well-typed are
given in \autoref{fig:globals-typing-rules}. In this figure, $\Delta$
is the set of all (global) constructor type definitions.

\begin{figure}[t]
\[
\begin{array}{c}
  \labinfer[T-Int]{\Gamma \vdash i : \Int}
    {i \in \mathbb{Z}}
  ~~~
  \labinfer[T-Bool]{\Gamma \vdash b : \Bool}
    {b \in \mathbb{B}}
  ~~~
  \labinfer[T-Var]{\Gamma \vdash v : \alpha}
    {(v : \alpha) \in \Gamma}
  \\\\
  \labinfer[T-Fn-Global]{\Gamma \vdash f : \alpha \rightarrow \beta}
    {(f : \alpha \rightarrow \beta) \in \Sigma}
  \\\\
  \labinfer[T-Add]{\Gamma \vdash x + y : \Int}
    {\Gamma \vdash x : \Int & \Gamma \vdash y : \Int}
  \\\\
  \labinfer[T-Lower-Var]{\Gamma \vdash \lowerS{A}{v} : A}
    {(v : \alpha) \in \Gamma & (A : \alpha \monic \layout[x]) \in \Sigma}
  \\\\
  \labinfer[T-Lower-Constr]{\Gamma \vdash \lowerS{A}{C\; e_1 \cdots e_n} : B}
    {\begin{gathered}
      (C : \alpha_1 \rightarrow \cdots \rightarrow \alpha_n \rightarrow \beta) \in \Sigma
      \\ (B : \beta \monic \layout[x]) \in \Sigma
      \\  \Gamma \vdash e_i \concrete{\alpha_i}\; \textnormal{for each $i$ with $1 \le i \le n$}
     \end{gathered}
    }
  \\\\
  % \labinfer[T-Lower]{\Gamma \vdash \lowerS{A}{e} : A}
  %   {\Gamma \vdash e : \alpha & (A : \alpha \monic \layout[x]) \in \Sigma}
  % \\\\
  \labinfer[T-Instantiate]{\Gamma \vdash \instantiateS{A,B}{f}(e) : B}
    {\begin{gathered}
          (A : \alpha \monic \layout[x]) \in \Sigma
      ~~~ (B : \beta \monic \layout[y]) \in \Sigma
      \\  \Gamma \vdash f : \alpha \rightarrow \beta
      ~~~ \Gamma \vdash e : A
     \end{gathered}
    }
  \\\\
  \labinfer[T-Constr]{\Gamma \vdash C\; e_1 \cdots e_n : \beta}
    {(C : \alpha_1 \rightarrow \cdots \rightarrow \alpha_n \rightarrow \beta) \in \Sigma &
     \Gamma \vdash e_i : \alpha_i\; \textnormal{for each $i$ with $1 \le i \le n$}
    }
\end{array}
\]
  \caption{Typing rules}
  \label{fig:typing-rules}
\end{figure}


\begin{figure}[b]
%\vspace{-5pt}
% \setlength{\belowcaptionskip}{-12pt}
% \setlength{\abovecaptionskip}{0pt}
\centering
% \begin{subfigure}{0.49\textwidth}
\begin{minipage}{1.0\linewidth}
{\footnotesize{
\[
  \begin{array}{c}
    \labinfer[C-Int]{e \concrete{\Int}}
      {\Gamma \vdash e : \Int}
    ~~~
    \labinfer[C-Bool]{e \concrete{\Bool}}
      {\Gamma \vdash e : \Bool}
    \\\\
    \labinfer[C-Layout]{e \concrete{\alpha}}
      {(A : \alpha \monic \layout[x]) \in \Sigma &
       \Gamma \vdash e : A
      }
  \end{array}
\]
}}
\end{minipage}
  \caption{Concreteness judgment rules}
  \label{fig:concreteness-rules}
% \end{subfigure}
%%
% \begin{subfigure}{0.49\textwidth}
\begin{minipage}{1.0\linewidth}
{\footnotesize{
\[
\begin{array}{c}
  \labinfer[G-Fn]{(f\; (C\; b_1 \cdots b_n) := e) \Rightarrow f : \beta \rightarrow \gamma}
    {(C : \alpha_1 \rightarrow \alpha_2 \rightarrow \cdots \rightarrow \alpha_n \rightarrow \beta) \in \Delta\\
    b_1 : \alpha_1, b_2 : \alpha_2, \cdots, b_n : \alpha_n \vdash e : \gamma
    }
\end{array}
\]
}}
\end{minipage}
  \caption{Global definition typing}
  \label{fig:globals-typing-rules}
% % \end{subfigure}
% %\setlength{\belowcaptionskip}{-15pt}
% \caption{Rules for concreteness judgement and typing global definitions}
% \label{fig:meh}
\end{figure}

