\section{\synth{} Keyword: Synthesis Inside \Pika{} 2}
\label{sec:synth}

Next, we introduce a feature that allows us to synthesise a program from a special kind of \Pika{} type signature. This special kind of type signature allows us to directly access \SuSLik's pure propositions. These type signatures are called \synth-signatures and the syntax for invoking the \Pika{} synthesis on a \synth-signature is:
\[
  \synth\; f : \sigma
\]
\noindent
where $\sigma$ is a \synth-signature. The syntax of a \synth-signature is given in Figure~\ref{fig:synth-sig-syntax}.

\begin{figure}
\[
\begin{aligned}
  \textnormal{\synth-signatures}\;\; &\sigma ::= p \synthsep \tau'\\
  \textnormal{Pure propositions}\;\; &p ::= x \mid i \mid b \mid p \mathbin{\&\&} p \mid p \mathbin{||} p
    \mid \Not(p) \mid p == p \mid \{\} \mid \{p\} \mid p \app p \mid p + p\\
  \textnormal{Extended types}\;\; &\tau' ::= ... \mid A\; \overline{G}\\
  \textnormal{Ghost variables}\;\; &G ::= @x
\end{aligned}
\]
  \caption{\synth-signature syntax}
  \label{fig:synth-sig-syntax}
\end{figure}

\noindent
We also need to extend layout definitions so that they can include ``ghost parameters.'' These parameters will directly correspond to ghost variable parameters in the resulting \SuSLik{} specification.

This extension to layouts is best explain by way of example. First, consider the original definition of the singly-linked list type:

\begin{lstlisting}[language=Pika]
Sll : List >-> layout[x]
Sll Nil := emp;
Sll (Cons head tail) :=
  x :-> head ** (x+1) :-> nxt
    ** Sll tail [nxt];
\end{lstlisting}

\noindent
This does not use any ghost variables in \Pika{} or in the resulting \SuSLik{} specification. A common pattern in \SuSLik{} specifications is to use a \verb|set| type ghost variable to keep track of the elements in a data structure. Then, we can require that the elements are preserved or that new elements be added when using this in specifications. The feature we add to layouts allows us to use this feature directly:

\begin{lstlisting}[language=Pika]
Sll : @(s : set) List >-> layout[x];
Sll Nil := s == {} ;; emp;
Sll (Cons head tail) :=
  s == {head} ++ s1
    ;;
  x :-> head ** (x+1) :-> nxt
    ** Sll tail @s1 [nxt];
\end{lstlisting}

\noindent
Putting this all together, we can use this new kind of parameter together with the \synth{} keyword to specify a list append function:

\begin{lstlisting}[language=Pika]
synth append :
  s == s1 ++ s2
    ;;
  Sll @s1 -> Sll @s2 -> Sll @s;
\end{lstlisting}

\noindent
Using this feature, we can use \SuSLik{} specifically for \textit{synthesis} tasks inside \Pika. That is, we are able to directly use \SuSLik{} for its intended purpose.


