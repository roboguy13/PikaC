\section{Static Semantics}

\subsection{Type System}

$E$ is the global environment. We write:

\begin{itemize}
  \item $E[\Delta]$ for $E$ extended with the local list of types $\Delta$
  \item $E[\Delta;C;\Gamma]$ for $E$ extended with the local list of types $\Delta$, the
    layout constraints $C$ and the typing context $\Gamma$.
\end{itemize}

\noindent
In particular, $E$ is a set that includes:
\begin{itemize}
  \item $(A : \Type)$ for every user-defined algebraic data type $A$
  \item $(\alpha \isA \layout{A})$ for every user-defined layout $\alpha$ for a data type $A$
\end{itemize}

\noindent
This follows the style of typing rules given in the Coq manual.~\cite{Coq-typing-rules} The use of layout constraints and layout
polymorphism largely follows the type system used by \Dargent, though there are some differences.~\cite{Dargent} These differences
are discussed further in Section~\ref{sec:related-work}.
% \[
%   \lowerExpr_\ell(e) = \apply_\ell(e)
% \]

A general type system is given in Fig.~\ref{fig:type-form}, Fig.~\ref{fig:layout-constraints} and Fig.~\ref{fig:typing-judgment}. Type signatures
in Pika are restricted to only have types of the form

\[
  \overline{\alpha_i \isA \layout{A_i}} \Ra \tau
\]

\noindent
where $\tau$ is a type with no layout constraints and the type variables $\overline{\alpha}$ are allowed to occur free.

\begin{figure}
  \[
  \begin{array}{c}
    \fbox{$E[\Delta] \vdash \tau ~\type$}
    \\\\\\
    \labinfer{E[\Delta, \tau ~\type] \vdash \tau ~\type}{}
    \\\\
    \labinfer{E[\Delta] \vdash A ~\type}{(A : \Type) \in E}
    \\\\
    \labinfer{E[\Delta] \vdash \Int ~\type}{}
    ~~~
    \labinfer{E[\Delta] \vdash \Bool ~\type}{}
    \\\\
    \labinfer{E[\Delta] \vdash \tau_1 \ra \tau_2 ~\type}{E[\Delta] \vdash \tau_1 ~\type & E[\Delta] \vdash \tau_2 ~\type}
    \\\\
    \highlight{
    \labinfer{E[\Delta] \vdash (\alpha \isA \layout{A}) \Ra \tau ~\type}{E[\Delta,\alpha ~\type] \vdash \tau ~\type}
    }
  \end{array}
  \]
  \caption{Type formation rules}
  \label{fig:type-form}
\end{figure}

% \begin{figure}
%   \[
%   \begin{array}{c}
%     \labinfer{\Delta \vdash \alpha ~\ctype}{\Delta \vdash \alpha ~\type}
%     ~~~
%     \labinfer{\Delta \vdash \forall (\alpha : \layout{A}).\; \beta ~\ctype}{\Delta \vdash A ~\adt & \Delta,\alpha ~\type \vdash C \Ra \beta ~\ctype}
%   \end{array}
%   \]
%   \caption{Constrained type formation rules}
% \end{figure}


% For info on typing rules for pattern matching, see:
%  - https://coq.inria.fr/distrib/current/refman/language/core/inductive.html#inductive-definitions
%  - Lecture Notes on Pattern Matching
%    15-814: Types and Programming Languages
%    Frank Pfenning
%    Lecture 12
%    Thursday, October 8, 2020

\begin{figure}
  \[
    \begin{array}{c}
      \fbox{$E[C] \vdash \alpha \isA \layout{A}$}
      \\\\\\
      \labinfer{E[C] \vdash \Int \isA \layout{\Int}}{}
      ~~~
      \labinfer{E[C] \vdash \Bool \isA \layout{\Bool}}{}
      \\\\
      \labinfer{E[C] \vdash L \isA \layout{A}}{(L \isA \layout{A}) \in E}
      \\\\
      \labinfer{E[C, \alpha \isA \layout{A}] \vdash \alpha \isA \layout{A}}{}
      % TODO: Should the rule below be included?
      % \\\\
      % \labinfer{E[\Delta] \vdash (\ell_1 \ra \ell_2) \isA \layout{A \ra B}}{E[\Delta] \vdash \ell_1 \isA \layout{A} & E[\Delta] \vdash \ell_2 \isA \layout{B}}
    \end{array}
  \]
  \caption{Layout constraint relation $\isA$}
  \label{fig:layout-constraints}
\end{figure}

\begin{figure}
  \[
    \begin{array}{c}
      \fbox{$E[\Delta;C;\Gamma] \vdash e : \tau$}
      \\\\\\
      \labinfer[T-Int]{E[\Delta;C;\Gamma] \vdash i : \Int}{i \in \mathbb{Z}}
      ~~~
      \labinfer[T-Bool]{E[\Delta;C;\Gamma] \vdash b : \Bool}{b \in \mathbb{B}}
      \\\\
      \labinfer[T-Var]{E[\Delta;C;\Gamma,v : \tau] \vdash v : \tau}{}
      \\\\
      % % \labinfer[T-Int-Layout]{\Delta;\Gamma \vdash \Int : \layout{\Int}}{}
      % % ~~~
      % % \labinfer[T-Bool-Layout]{\Delta;\Gamma \vdash \Bool : \layout{\Bool}}{}
      % % \\\\
      % % \labinfer{\Delta;\Gamma \vdash \alpha \ra \beta : \layout{A \ra B}}{\Delta;\Gamma \vdash \alpha : \layout{A} & \Delta;\Gamma \vdash \beta : \layout{B}}
      \highlight{
      \labinfer[T-Layout-Lambda]{E[\Delta;C;\Gamma] \vdash \Lambda (\alpha \isA \layout{A}).\; e : (\alpha \isA \layout{A}) \Ra \tau}
        {E[\Delta] \vdash \alpha ~\type & E[\Delta;C,\alpha \isA \layout{A};\Gamma] \vdash e : \tau}
      }
      % % \\\\
      % % \labinfer[T-Lower]{\Delta;\Gamma \vdash \lowerExpr_\alpha(e) : \forall (\alpha : \layout{A}).\; \alpha}
      % %   {\Delta, \alpha ~\type;\Gamma, \alpha : \layout{A} \vdash e : A}
      \\\\
      \highlight{
      \labinfer[T-Layout-App]{E[\Delta;C;\Gamma] \vdash \apply_{\alpha'}(e) : \tau[\alpha\mapsto\alpha']}
        {E[C] \vdash \alpha' \isA \layout{A} & E[\Delta;C;\Gamma] \vdash e : (\alpha \isA \layout{A}) \Ra \tau}
      }
      \\\\
      \highlight{
      % \textnormal{{\color{red} TODO:} Does this rule make sense?}
      % \\
        \labinfer[T-Layout-Lift]{E[\Delta;C;\Gamma] \vdash \liftExpr(e) : A}{E[C] \vdash \alpha \isA \layout{A} & E[\Delta;C;\Gamma] \vdash e : \alpha}
      }
      % \labinfer[T-Instantiate]{\Delta;\Gamma \vdash \instantiate_{\alpha,\beta}(e) : A \ra B}
      %   {\Delta;\Gamma \vdash e : \alpha \ra \beta & \Delta;\Gamma \vdash \alpha : \layout{A} & \Delta;\Gamma \vdash \beta : \layout{B}}
      % \\\\
      % \fbox{{\color{red} TODO:} Finish this rule}
      % \\
      % \labinfer[T-Match]{\matchWith{e_0} C_1\; \overline{x}_1 \Ra e_1 \mid \cdots \mid C_n\; \overline{x}_n \Ra e_n}{}
      \\\\
      % \labinfer[T-Layout-Def]{E[] \vdash f : \layout{A}}{}
      % \\\\
      \labinfer[T-Lambda]{E[\Delta;C;\Gamma] \vdash \lambda x.\; e : \tau_1 \ra \tau_2}
        {E[\Delta;C;\Gamma, x : \tau_1] \vdash e : \tau_2}
      \\\\
      \labinfer[T-App]{E[\Delta;C;\Gamma] \vdash e_1\; e_2 : \tau_2}
        {E[\Delta;C;\Gamma] \vdash e_1 : \tau_1 \ra \tau_2
        &E[\Delta;C;\Gamma] \vdash e_2 : \tau_1}
      \\\\
      \labinfer[T-InL]{E[\Delta;C;\Gamma] \vdash \inL(e) : \tau_1 \Either \tau_2}{E[\Delta;C;\Gamma] \vdash e : \tau_1}
      ~~~
      \labinfer[T-InR]{E[\Delta;C;\Gamma] \vdash \inR(e) : \tau_1 \Either \tau_2}{E[\Delta;C;\Gamma] \vdash e : \tau_2}
      \\\\
      \labinfer[T-Pair]{E[\Delta;C;\Gamma] \vdash (e_1, e_2) : \tau_1 \Pair \tau_2}{E[\Delta;C;\Gamma] \vdash e_1 : \tau_1 & E[\Delta;C;\Gamma] \vdash e_2 : \tau_2}
      \\\\
      \labinfer[T-Fst]{E[\Delta;C;\Gamma] \vdash \fst(e) : \tau_1}{E[\Delta;C;\Gamma] \vdash e : \tau_1 \Pair \tau_2}
      ~~~
      \labinfer[T-Snd]{E[\Delta;C;\Gamma] \vdash \snd(e) : \tau_2}{E[\Delta;C;\Gamma] \vdash e : \tau_1 \Pair \tau_2}
      \\\\
      \labinfer[T-Match]{E[\Delta;C;\Gamma] \vdash \matchWith{e} \inL(x) \Ra e_1 \mid \inR(y) \Ra e_2 : \tau}
        {E[\Delta;C;\Gamma] \vdash e : \tau_1 \Either \tau_2
        &E[\Delta;C;\Gamma, x : \tau_1] \vdash e_1 : \tau
        &E[\Delta;C;\Gamma, y : \tau_2] \vdash e_2 : \tau}
      \\\\
      \labinfer[T-Fold]{E[\Delta;C;\Gamma] \vdash \fold_\tau(e) : \tau}
        {\tau = \mu\alpha.\tau_1 & E[\Delta;C;\Gamma] \vdash e : \tau_1[\alpha\mapsto\tau]}
      \\\\
      \labinfer[T-Unfold]{E[\Delta;C;\Gamma] \vdash \unfold_\tau(e) : \tau_1[\alpha\mapsto\tau]}
        {\tau = \mu\alpha.\tau_1 & E[\Delta;C;\Gamma] \vdash e : \tau}

      \\\\
      \highlight{
      \labinfer[T-SSL]{E[\Delta;C;\Gamma] \vdash \ssl{\overline{x}}{\overline{x \mapsto e}} : \SSL({\overline{\langle x, \tau \rangle}})}
        {E[\Delta;C;\Gamma] \vdash e_i : \tau_i \ensuremath{\textnormal{ for each $x_i$}}}
      }
    \end{array}
  \]
  \caption{Typing judgment}
  \label{fig:typing-judgment}
\end{figure}

\begin{figure}
  \[
    \begin{array}{c}
      \labinfer[T-Def-Eq]{E[\Delta;C;\Gamma] \rhd e_1 \defeq e_2 : \tau}
        {E[\Delta;C;\Gamma] \vdash e_1 : \tau
        &E[\Delta;C;\Gamma] \vdash e_2 : \tau}
    \end{array}
  \]
  \caption{Definitional equation typing}
  \label{fig:def-eq}
\end{figure}

% \begin{figure}
%   \[
%     \begin{array}{c}
%       % \labinfer[T-Lower]{\Delta;\Gamma \vdash \lowerExpr_\ell'(e) : \tau[\ell\mapsto\ell']}
%       %   {\vdash \ell' \isA \layout{A} & \Delta;\Gamma \vdash e : \forall (\ell \isA \layout{A}).\; \tau}
%       % \\\\
%       \labinfer[T-Instantiate]{\Delta;\Gamma \vdash \instantiate_{\ell_1',\ell_2'}(e) : (\alpha \ra \beta)[\ell_1\mapsto\ell_1'][\ell_2\mapsto\ell_2']}
%         {\begin{gathered}
%           \Delta \vdash \ell_1' \isA \layout{A}
%           ~~~ \Delta \vdash \ell_2' \isA \layout{B}
%           \\ \Delta;\Gamma \vdash e : \forall (\ell_1 \isA \layout{A}).\; \forall (\ell_2 \isA \layout{B}).\; \alpha \ra \beta
%         \end{gathered}}
%     \end{array}
%   \]
%   \caption{Derived types for layout constructs}
% \end{figure}

\subsection{ANF Translation}

[English overview of ANF translation phase]

\subsection{Desugaring}

Well-typed functions written in the surface language
are desugared into a simplified language with no layout polymorphism. Instead of using layout polymorphism, this language explicitly manipulates layouts.

% \begin{figure}
%   \caption{Additional typing rules for desugared language}
% \end{figure}

\begin{figure}
  \[
    \labinfer{E \vdash \apply_\alpha(e) \desugars \cdots}{(\alpha = e'') \in E & \textrm{{\color{red}TODO:} finish}}
  \]
  \caption{Desugaring judgment}
  \label{fig:desugaring}
\end{figure}

