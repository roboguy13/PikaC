% From https://tex.stackexchange.com/questions/235118/making-a-thicker-cdot-for-dot-product-that-is-thinner-than-bullet
\makeatletter
\newcommand*\bigcdot{\mathpalette\bigcdot@{.5}}
\newcommand*\bigcdot@[2]{\mathbin{\vcenter{\hbox{\scalebox{#2}{$\m@th#1\bullet$}}}}}
\makeatother

\newcommand {\Pika} {\textsf{Pika}}
\newcommand {\PikaCore} {\textsf{PikaCore}}

\newcommand {\instExpr} {\keyword{inst}}

\newcommand {\labinfer} [3] [] {\infer[{\textsc{#1}}]{#2}{#3}}

\newcommand {\keyword} [1] {\textsf{#1}}

\newcommand {\Int} {\keyword{Int}}
\newcommand {\Bool} {\keyword{Bool}}
\newcommand {\Type} {\keyword{Type}}
\newcommand {\layout} [1] {\keyword{layout}(#1)}

\newcommand {\inL} {\keyword{inL}}
\newcommand {\inR} {\keyword{inR}}
\newcommand {\fold} {\keyword{fold}}
\newcommand {\unfold} {\keyword{unfold}}
\newcommand {\fst} {\keyword{fst}}
\newcommand {\snd} {\keyword{snd}}
\newcommand {\either} {\keyword{either}}

\newcommand {\Pair} {\otimes}
\newcommand {\Either} {\oplus}
\newcommand {\RecTy} {\mu}

\newcommand {\monic} {\rightarrowtail}

\newcommand {\type} {\textsf{type}}
\newcommand {\ctype} {\textsf{ctype}}
\newcommand {\adt} {\textsf{adt}}

\newcommand {\LayoutDefs} {\mathcal{L}}

\newcommand {\matchWith} [1] {\keyword{case}\,#1\,\keyword{of}\,}
\newcommand {\layoutmatch} [1] {\keyword{layoutcase}\,#1\,\keyword{of}\,}

\newcommand {\withIn} [1] {\keyword{with}\,#1\,\keyword{in}\,}
\newcommand {\letPure} [1] {\keyword{letpure}\,#1\,\keyword{in}\,}

% Based on https://tex.stackexchange.com/questions/451786/how-do-i-put-a-circle-around-a-symbol
\makeatletter
\newcommand {\osep}{\mathbin{\mathpalette\make@circled\ast}}
\newcommand{\make@circled}[2]{%
  \ooalign{$\m@th#1\smallbigcirc{#1}$\cr\hidewidth$\m@th#1#2$\hidewidth\cr}%
}
\newcommand{\smallbigcirc}[1]{%
  \vcenter{\hbox{\scalebox{0.77778}{$\m@th#1\bigcirc$}}}%
}
\makeatother

% \newcommand {\lowerExpr} {\keyword{lower}}
% \newcommand {\liftExpr} {\keyword{lift}}
\newcommand {\instantiate} {\keyword{instantiate}}
\newcommand {\apply} {\keyword{apply}}

\newcommand {\liftExpr} {\keyword{lift}}

\newcommand {\tyApply} {\mathbin{@}}

\newcommand {\isA} {\sim}

\newcommand {\ra} {\rightarrow}
\newcommand {\Ra} {\Rightarrow}

\newcommand {\Dargent} {{\sc Dargent}}

\newcommand {\highlight} [1] {\begin{tcolorbox}[hbox] #1 \end{tcolorbox}}

\newcommand {\tyLambda} {\mathrm{\Lambda}}

\newcommand {\sem} [1] {\llbracket #1 \rrbracket}

\newcommand {\argCount} [1] {N\sem{#1}}
\newcommand {\namesem} [1] {\Var^{\argCount{#1}}}

\newcommand {\steps} {\longrightarrow}
\newcommand {\desugars} {\steps_d}
\newcommand {\anfsteps} {\steps_{\textsf{ANF}}}

% \newcommand {\ssl} [2] {\keyword{ssl}(#1) \{ #2 \}}
% \newcommand {\ssl} [1] {\keyword{ssl} \{ #1 \}}
\newcommand {\ssl} [1] {\{ #1 \}}
\newcommand {\SSL} {\keyword{SSL}}

\newcommand {\Var} {\textsf{Var}}

\newcommand {\Heap} {\textsf{Heap}}

\newcommand {\lifted} [1] {#1_{\bot}}

\newcommand {\eq} {\stackrel{\bigcdot}{=}}
\newcommand {\defeq} {:=}
\newcommand {\ok} [1] {\;\textsf{ok}_{#1}}

\newcommand {\typesem} [1] {\mathcal{A}_{#1}}

\newcommand {\True} {\keyword{True}}
\newcommand {\False} {\keyword{False}}

% From https://tex.stackexchange.com/questions/502652/define-tcolorbox-in-math-mode
\newtcbox{\boxtext}{on line,colback=white,colframe=black,size=fbox,arc=3pt,boxrule=0.8pt}
\newcommand{\boxmath}[1]{\boxtext{$#1$}}

% https://tex.stackexchange.com/questions/24132/overline-outside-of-math-mode
\makeatletter
\newcommand*{\textoverline}[1]{$\overline{\hbox{#1}}\m@th$}
\makeatother

