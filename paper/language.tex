\section{Syntax}

\Pika{} code is translated into \PikaCore{} before being translated into the target language. \PikaCore{}
is a layout monomorphic version of \Pika{}. While it retains most of the functional syntax of \Pika,
all data structures are manipulated as collections of SSL heaplets. This includes both construction
of data structure values and pattern matching.

\subsection{Shared Syntax}

The following syntactic forms are shared between the \Pika{} surface language and \PikaCore.

\begin{align*}
  &i ::= \cdots \mid -2 \mid -1 \mid 0 \mid 1 \mid 2 \mid \cdots
  \\
  &b ::= \True{} \mid \False
  \\
  &e ::= x \mid i \mid b \mid \lambda x.\; e
  \\
  &\tau ::= \Int \mid \Bool \mid \tau \ra \tau
\end{align*}

\subsection{\Pika{} Syntax}
\begin{align*}
  &e ::= \cdots \mid \apply_{\alpha}(e) \mid \liftExpr_{\alpha}(e) \mid \Lambda (\alpha \isA \layout{X}).\; e
  \\
  &\tau ::= \cdots \mid X \mid \alpha \mid e\; e \mid (\alpha \isA \layout{X}) \Ra \tau
\end{align*}

\subsection{\PikaCore{} Syntax}
\begin{align*}
  &e ::= \cdots \mid e\; \overline{a} \mid \sslmath{\overline{x}}{\overline{h}} \mid \withIn{\{ \overline{x} \} := e} e
  \\
  &a ::= \{ \overline{x} \} \mid \{ i \} \mid \{ b \}
  \\
  &\tau ::= \cdots \mid \SSL(n)
  \\
  &h ::= \ell \mapsto e
  \\
  &\ell ::= x \mid (x + n)
  \\
  &n ::= 0 \mid 1 \mid 2 \mid \cdots
\end{align*}

The construct $\sslmath{\overline{x}}{\overline{h}}$ is an SSL assertion with heaplets $\overline{h}$ and
bound variables $\overline{x}$. Note that it can have free variables as well.

% The general form of a one-parameter \PikaCore{} function definition is
% \begin{align*}
%   &f : \SSL(m) \ra \SSL(n)\\
%   &f\; \{ \overline{\ell \mapsto y} \} \outs \{ \overline{r} \} := e\\
%   &f\; \{ \overline{\ell' \mapsto y} \} \outs \{ \overline{r} \} := e'\\
%   &\cdots
% \end{align*}
%
% \noindent
% where $\overline{r}$ are the names of the output parameters. These output parameters
% must be the same in each branch and it is also required that the number of output parameters matches the type signature: $|\overline{r}| = n$.
%
% Output parameters are given in \PikaCore{} as part of the \verb|with-in| construct:
%
% \begin{align*}
%   \withIn{\{ a, b \} := f\; \{ x \}} e
% \end{align*}
%
% This applies the function $f$ to the input parameter $x$ and gives $a$ and $b$ as output parameters. These two
% output parameters of the application are brought into scope for the \PikaCore{} expression $e$.
%
