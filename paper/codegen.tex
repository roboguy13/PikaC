\section{C Code Generation}

The translation of \PikaCore{} function into C is divided into the stages.

\begin{enumerate}
  \item \label{stage:in-out} For each branch, determine the abstract heap for the input and a collection of
    heaplets for the output.
  \item Generate the appropriate assignment statements, \verb|malloc| calls, etc from the
    previous stage
\end{enumerate}

\subsection{Example: leftList}
First, lets look at a \Pika{} function that takes a tree and gives back a list by traversing the leftmost branches.

\begin{lstlisting}
%generate leftList[TreeLayout, Sll]

data List := Nil | Cons Int List;
data Tree := Tip | Node Int Tree Tree;

Sll : List >-> SSL(1);
Sll Nil {x} := emp;
Sll (Cons h t) {x} := x :-> h, (x+1) :-> nxt, Sll t {nxt};

TreeLayout : Tree >-> SSL(1);
TreeLayout Tip {x} := emp;
TreeLayout (Node v left right) {x} :=
  y :-> v, (y+1) :-> p, (y+2) :-> q,
  TreeLayout left {p}, TreeLayout right {q} ;

leftList : (a ~ layout(Tree), b ~ layout(List)) =>
  a -> b
leftList Tip := Nil;
leftList [a b] (Node v left right) :=
  Cons v (leftList$_{\verb|a|,\verb|b|}$ left);
\end{lstlisting}

\noindent
This is first translated into \PikaCore:

\begin{lstlisting}
leftList' : SSL(1) -> SSL(1);
leftList' {} => {r} := {};
leftList' {x :-> h, (x+1) :-> p, (x+2) :-> q} => {r} :=
  with {t} := leftList' {p}
  in
  {r :-> h, (r+1) :-> t};
\end{lstlisting}

\noindent
Generated C code:

\begin{lstlisting}
typedef struct Sll {
  int x_0;
  struct Sll* x_1;
} Sll;

typedef struct TreeLayout {
  int y_0;
  struct TreeLayout* y_1;
  struct TreeLayout* y_2;
} TreeLayout;

void leftList(TreeLayout* arg, Sll** r) {
  if (arg == 0) {
      // pattern match {}

      // {}
    *r = 0;
  } else {
      // pattern match {x :-> h, (x+1) :-> p
      //               , (x+2) :-> q}
    int v = arg->y_0;
    TreeLayout* p = arg->y_1;
    TreeLayout* q = arg->y_2;

      // with {t} := leftList' {p} in ...
    Sll* t = 0;
    leftList(p, &t);

      // {y :-> h, (y+1) :-> t}
    *r = malloc(sizeof(Sll));
    *r->x_0 = v;
    *r->x_1 = t;
  }
}
\end{lstlisting}

% \noindent
% In Stage~\ref{stage:in-out}

\subsection{Example: convertList}

\begin{lstlisting}
%generate convertList[Dll, Sll]
%generate convertList[Sll, Dll]

Dll : List >-> SSL(2)
Dll Nil {x z} := emp;
Dll (Cons h t) {x z} :=
  x :-> h, (x+1) :-> w, (x+2) :-> z,
  Dll t {w x};

convertList : (a ~ layout(List), b ~ layout(List)) =>
  a -> b;
convertList Nil := Nil;
convertList [a b] (Cons h t) := Cons h (convertList$_{\verb|a|,\verb|b|}$ t);
\end{lstlisting}

Generated \PikaCore:

\begin{lstlisting}
convertList1 : SSL(2) -> SSL(1);
convertList1 {} -> {r} := {};
convertList1 {x :-> h, (x+1) :-> w, (x+2) :-> z} -> {r} :=
  with {nxt} := convertList1 {w x}
  in
  {r :-> h, (r+1) :-> nxt}

convertList2 : SSL(1) -> SSL(2);
convertList2 {} -> {r z} := {};
convertList2 {x :-> h, (x+1) :-> nxt} -> {r z} :=
  with {w r} := convertList2 {nxt}
  in
  {r :-> h, (r+1) :-> w, (r+2) :-> z}
\end{lstlisting}

Generated C code:

\begin{lstlisting}
typedef struct Dll {
  int x_0;
  struct Dll* x_1;
  struct Dll* x_2;
} Dll;

void convertList1(Dll* arg, Sll** r) {
  if (arg == 0) {
      // pattern match {}

      // {}
    *r = 0;
  } else {
      // pattern match {x :-> h, (x+1) :-> w, (x+2) :-> z}
    int h = arg->x_0;
    Dll* w = arg->x_1;
    Dll* z = arg->x_2;

      // allocations for result
    *r = malloc(sizeof(Sll));

      // with {nxt} := convertList1 {w x} in ...
    Sll* nxt = 0;
    convertList1(arg, &nxt);

      // {r :-> h, (r+1) :-> nxt}
    (*r)->x_0 = h;
    (*r)->x_1 = nxt;
  }
}

void convertList2(Sll* arg, Dll** r, Dll** z) {
  if (arg == 0) {
      // pattern match {}

      // {}
    *r = 0;
  } else {
      // pattern match {x :-> h, (x+1) :-> nxt}
    int h = arg->x_0;
    Sll* nxt = arg->x_1;

      // allocations for result
    *r = malloc(sizeof(Dll));

      // with {w r} := convertList2 {nxt} in ...
    Dll* w = 0;
    convertList2(nxt, &w, r);

      // {r :-> h, (r+1) :-> w, (r+2) :-> z}
    (*r)->x_0 = h;
    (*r)->x_1 = w;
    (*r)->x_2 = *z;
  }
}
\end{lstlisting}

