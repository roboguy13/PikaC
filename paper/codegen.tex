\section{C Code Generation}

The translation of PikaCore function into C is divided into the stages.

\begin{enumerate}
  \item \label{stage:in-out} For each branch, determine the abstract heap for the input and a collection of
    heaplets for the output.
  \item Generate the appropriate assignment statements, \verb|malloc| calls, etc from the
    previous stage
\end{enumerate}

\subsection{Example}
First, lets look at a Pika function that takes a tree and gives back a list by traversing the leftmost branches.

\begin{lstlisting}
%generate leftList[TreeLayout, Sll]

data List := Nil | Cons Int List;
data Tree := Tip | Bin Int Tree Tree;

Sll : layout(List);
Sll Nil := emp;
Sll (Cons h t) := x :-> h, (x+1) :-> t, Sll t;

TreeLayout : layout(Tree);
TreeLayout Tip := emp;
TreeLayout (Bin v l r) :=
  y :-> v, (y+1) :-> l, (y+2) :-> r, TreeLayout l, TreeLayout r;

leftList : (a ~ layout(Tree), b ~ layout(List)) => a -> b
leftList Tip := Nil;
leftList [a b] (Bin v l r) := Cons v ($\apply_a$ (leftList ($\liftExpr_b$ l));
\end{lstlisting}

\noindent
This is first translated into PikaCore:

\begin{lstlisting}
leftList' :
  ($\mu s.$ SSL(x : Int, (x+1) : s, (x+2) : s))
  -> $\mu t.$ SSL(y : Int, (y+1) : t)
leftList' {} = ssl {}
leftList' {x :-> h, (x+1) :-> l, (x+2) :-> r} =
  with nxt := leftList' l
  in
  ssl {y :-> h, (y+1) :-> nxt}
\end{lstlisting}

\noindent
In Stage~\ref{stage:in-out}

