\section{Elaboration}

The body of a layout polymorphic \Pika{} function will often make use of the 
layout parameters that it is given. For example, consider the function \texttt{singleton}.

\begin{flalign*}
  &\texttt{singleton} : (a \isA \layout{\textrm{List}}) \Ra \Int \ra a\\
  &\texttt{singleton}\; i := \textrm{Cons}\; i\; \textrm{Nil}
\end{flalign*}

\noindent
We know just by looking at the type that we want to give back a \textrm{List} using
whatever layout is given as $a$. However, without elaboration, this is not enough. We must
explicitly lower the constructors to use the layout $a$ like this:

\begin{flalign*}
  &\texttt{singleton} : (a \isA \layout{\textrm{List}}) \Ra \Int \ra a\\
  &\texttt{singleton}\; [a]\; i := \apply_a(\textrm{Cons}\; i\; (\apply_a(\textrm{Nil})))
\end{flalign*}

\noindent
Elaboration will turn the first definition of \texttt{singleton} into the second definition, using
a modified type inference algorithm. \Pika{} definitions must be fully elaborated before being
translated into \PikaCore.

