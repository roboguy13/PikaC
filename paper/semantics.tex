\section{Denotational Semantics}

We give a denotational semantics for PikaCore. Since Pika is translated to PikaCore, this can
also be seen as a denotational semantics for Pika.

The function $\argCount{\tau}$ gives the number of fields in abstract
heap associated to the type $\tau$. In the case of function types, this
is the number of fields associated to the \textit{result} of the function.

\begin{align*}
  % &\typesem{\Int} = \Var \ra \lifted{\mathbb{Z}}
  % \\
  % &\typesem{\Bool} = \Var \ra \lifted{\mathbb{B}}
  % \\
  % &\typesem{\SSL(\langle x_1, \tau_1 \rangle, \cdots, \langle x_n, \tau_n \rangle)} = \Var^n \ra \lifted{\Heap}
  % \\
  % &\typesem{\tau_1 \ra \tau_2} = \typesem{\tau_1} \ra \typesem{\tau_2}
  &\typesem{\Int} = \namesem{\Int} \ra \Heap
  \\
  &\typesem{\Bool} = \namesem{\Bool} \ra \Heap
  \\
  &\typesem{\SSL(p)} = \namesem{\SSL(p)} \ra \Heap
  \\
  &\typesem{\tau_1 \ra \tau_2} = \namesem{\tau_1 \ra \tau_2} \ra \typesem{\tau_1} \ra \typesem{\tau_2}
\end{align*}

\begin{align*}
  &\argCount{\Int} = 1
  \\
  &\argCount{\Bool} = 1
  \\
  &\argCount{\SSL(\langle x_1, \tau_1 \rangle, \cdots, \langle x_n, \tau_n \rangle)} = n
  \\
  &\argCount{\tau_1 \ra \tau_2} = \argCount{\tau_2}
\end{align*}
\\

\begin{figure}
\begin{center}
  \fbox{
    $\sem{E[\Delta;C;\Gamma] \vdash e : \tau} \in \typesem{\tau}$
  }
\end{center}
\begin{align*}
  &\sem{E[\Delta;C;\Gamma] \vdash i : \Int}_r = \{ r \mapsto i \}
  \\
  &\sem{E[\Delta;C;\Gamma] \vdash b : \Bool}_r = \{ r \mapsto b \}
  \\
  &\sem{E[\Delta;C;\Gamma] \vdash \withIn{a := e_1 : \tau_1} e_2 : \tau_2}_r = \sem{E[\Delta;C;\Gamma] \vdash e_1 : \tau_1}_a \osep \sem{E[\Delta;C;\Gamma] \vdash e_2 : \tau_2}_r
  \\
  &\sem{E[\Delta;C;\Gamma] \vdash \ssl{\overline{x}}{\overline{x_i \mapsto e_i}} : \SSL(\overline{\langle x_i, \tau_i \rangle})}_r =
    \sem{E[\Delta;C;\Gamma] \vdash e_1 : \tau_1}_{r_1} \osep \cdots \osep \sem{E[\Delta;C;\Gamma] \vdash e_n : \tau_n}_{r_n}
  \\
  &\sem{E[\Delta;C;\Gamma] \vdash e\; v : \tau_2}_r =
      \sem{E[\Delta;C;\Gamma] \vdash e : \tau_1 \ra \tau_2}_r(\sem{E[\Delta;C;\Gamma] \vdash v : \tau_2})
  \\
  &\sem{E[\Delta;C;\Gamma] \vdash \lambda x.\; e : \tau_1 \ra \tau_2}_r(f)
      = \sem{E[\Delta;C;\Gamma, x : \tau_1] \vdash e : \tau_2}_r \osep f(x)
\end{align*}
  \caption{Semantic function for PikaCore}
\end{figure}

% \subsection{Equational soundness}
%
% \begin{theorem}[Type denotation]
%   If $\tau$ is a quantifier-free type without layout constraints then
%   \[
%     \sem{E[\Delta;C;\Gamma] \vdash e : \tau} \in \typesem{\tau}
%   \]
% \end{theorem}
%
% \begin{theorem}[Denotation continuity]
%   $\sem{E[\Delta;C;\Gamma] \vdash e : \tau_1 \ra \tau_2}$ is a continuous function from $\typesem{\tau_1}$ to $\typesem{\tau_2}$.
% \end{theorem}
%
% \begin{theorem}[Equational soundness]
%   \[
%       \boxmath{\sem{E[\Delta;C;\Gamma] \rhd e_1 \defeq e_2 : \tau}}
%     \models
%       \boxmath{\sem{E[\Delta;C;\Gamma] \vdash e_1 : \tau}}
%         =
%       \boxmath{\sem{E[\Delta;C;\Gamma] \vdash e_2 : \tau}}
%   \]
% \end{theorem}
%
