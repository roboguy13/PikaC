\section{Semantics}

\subsection{Translation From \Pika{} to \PikaCore}

% Note that the lambda case of the $\val$ judgment is unusual. It requires
% the body of the lambda to be reduced.

\[
  \begin{array}{c}
    \fbox{$e ~\val$}
    \fboxNewlines
    \labinfer{i ~\val}{i \in \mathbb{Z}}
    ~~~
    \labinfer{b ~\val}{b \in \mathbb{B}}
    ~~~
    \labinfer{v ~\val}{v \in \Var}
    \\\\
    \labinfer{\sslmath{\overline{x}}{\overline{\ell_i \mapsto e_i}} ~\val}
      {e_i ~\val \;\textrm{ for each $e_i$}}
    \\\\
    \labinfer{\lambda x.\; e ~\val}{e ~\val}
    ~~~
    \labinfer{\withIn{\{\overline{x} := e_1\}} e_2 ~\val}
      {e_1 ~\val & e_2 ~\val}
  \end{array}
\]
\\

\[
  \begin{aligned}
  \mathcal{E}
    ::=\; &[]\\
    \mid\; &\apply_A(\mathcal{E})\\
    \mid\; &\mathcal{E}\; e\\
    \mid\; &v\; \mathcal{E}\\
    \mid\; &\withIn{\{ \overline{x} \} := \mathcal{E}} e\\
    \mid\; &\withIn{\{ \overline{x} \} := v} \mathcal{E}\\
    \mid\; &\sslmath{\overline{x}}{\overline{\ell \mapsto v}, \ell \mapsto \mathcal{E}, \overline{\ell \mapsto e}}\\
    \mid\; &\lambda x.\; \mathcal{E}
  \end{aligned}
\]

\[
  \begin{array}{c}
    \fbox{$e \pcstep e$}
    \fboxNewlines
    \labinfer{\mathcal{E}[e] \pcstep \mathcal{E}[e']}
      {e \pcNotionStep e'}
  \end{array}
\]

Where $\mathcal{L} \sqcup \mathcal{L}'$ is the language union that combines common non-terminals shared by the $\mathcal{L}$ and $\mathcal{L}'$.

\[
  \begin{array}{c}
    \fbox{$e \pcNotionStep e'\textrm{ where $e, e' \in (\Pika \sqcup \PikaCore)$}$}
    \fboxNewlines
    \labinfer[PC-Int]{E[\Delta;C;\Gamma] \vdash i \pcNotionStep i}{E[\Delta;C;\Gamma] \vdash i : \Int}
    ~~~
    \labinfer[PC-Bool]{E[\Delta;C;\Gamma] \vdash b \pcNotionStep b}{E[\Delta;C;\Gamma] \vdash b : \Bool}
    \\\\
    \labinfer[PC-Lambda]{E[\Delta;C;\Gamma] \vdash \lambda x.\; e \pcNotionStep \lambda x.\; e}{}
    \\\\
    \labinfer[PC-Unfold-Layout-Ctr]{E[\Delta;C;\Gamma] \vdash \apply_A(C\; \overline{e}) \pcNotionStep \sslmath{\overline{x}}{\overline{h'}}}
      {\fresh{\overline{x}}
      & (A\; (C\; \overline{y}) := \sslmath{\overline{x}}{\overline{h}}) \in E
      & \overline{h'} = \overline{h}[\overline{y} := \overline{e}]
      }
    \\\\
    \labinfer[PC-App]{E[\Delta;C;\Gamma] \vdash e_1\;(\apply_A(e_2)) \pcNotionStep \withIn{\{ \overline{x} \} := e_2'} e_1\; \{\overline{x}\}}
      {}
    \\\\
    \labinfer[PC-With-App]{E[\Delta;C;\Gamma] \vdash e\; (\withIn{\{ \overline{x} \} := e_1} e_2) \pcNotionStep \withIn{\{ \overline{x} \} := e_1} e\; e_2}
      {\overline{x} \not\in \FV(e)}
    \\\\
    \labinfer[PC-Apply]{E[\Delta;C;\Gamma] \vdash \apply_A(e) \pcNotionStep e}
      {}
    % \\\\
    % \labinfer[PC-Unfold-Layout-Var]{E[\Delta;C;\Gamma] \vdash \apply_A(x)
      % \pcNotionStep \sslmath{\overline{y}}{\overline{h}}}
    %   {}
  \end{array}
\]

Define the big-step relation induced by the small-step relation by the following

\[
  \begin{array}{c}
    \labinfer{e \pcStep e'}
      {e \pcsteps e' & e' ~\val}
  \end{array}
\]

\begin{theorem} $\pcStep{} \subseteq \Pika \times \PikaCore$
\end{theorem}

% \subsection{Denotational Semantics}
%
% We give a denotational semantics for \PikaCore. Since \Pika{} is translated to \PikaCore, this can
% also be seen as a denotational semantics for \Pika.
%
% The function $\argCount{\tau}$ gives the number of fields in abstract
% heap associated to the type $\tau$. In the case of function types, this
% is the number of fields associated to the \textit{result} of the function.
%
% \begin{align*}
%   % &\typesem{\Int} = \Var \ra \lifted{\mathbb{Z}}
%   % \\
%   % &\typesem{\Bool} = \Var \ra \lifted{\mathbb{B}}
%   % \\
%   % &\typesem{\SSL(\langle x_1, \tau_1 \rangle, \cdots, \langle x_n, \tau_n \rangle)} = \Var^n \ra \lifted{\Heap}
%   % \\
%   % &\typesem{\tau_1 \ra \tau_2} = \typesem{\tau_1} \ra \typesem{\tau_2}
%   &\typesem{\Int} = \namesem{\Int} \ra \Heap
%   \\
%   &\typesem{\Bool} = \namesem{\Bool} \ra \Heap
%   \\
%   &\typesem{\SSL(n)} = \namesem{\SSL(n)} \ra \Heap
%   \\
%   &\typesem{\tau_1 \ra \tau_2} = \namesem{\tau_1 \ra \tau_2} \ra \typesem{\tau_1} \ra \typesem{\tau_2}
% \end{align*}
%
% \begin{align*}
%   &\argCount{\Int} = 1
%   \\
%   &\argCount{\Bool} = 1
%   \\
%   &\argCount{\SSL(n)} = n
%   \\
%   &\argCount{\tau_1 \ra \tau_2} = \argCount{\tau_2}
% \end{align*}
% \\
%
% \begin{figure}
% \begin{center}
%   \fbox{
%     $\sem{e} \in \typesem{\tau}\textrm{ where $E[\bullet] \vdash e : \tau$}$
%   }
% \end{center}
% \begin{align*}
%   &\sem{i}_r = \{ r \mapsto i \}\tag{where $i \in \mathbb{Z}$}
%   \\
%   &\sem{b}_r = \{ r \mapsto b \}\tag{where $i \in \mathbb{B}$}
%   \\
%   &\sem{\withIn{a := e_1} e_2}_r = \sem{e_1}_a \osep \sem{e_2}_r
%   \\
%   &\sem{\sslmath{\overline{x}}{\overline{\ell_i \mapsto e_i}}}_r =
%     \sem{e_1[\overline{x := r}]}_{\ell_1[\overline{x := r}]} \osep \cdots \osep \sem{e_n[\overline{x := r}]}_{\ell_n[\overline{x := r}]}
%   \\
%   &\sem{e\; v}_r =
%       \sem{e}_r(\sem{v}_{-})
%   \\
%   &\sem{\lambda x.\; e}_r(f)
%       = \sem{e}_r \osep f(x)
% \end{align*}
%   \caption{Denotation function for \PikaCore ({\color{red}TODO}: Finish)}
% \end{figure}

% \subsection{Equational soundness}
%
% \begin{theorem}[Type denotation]
%   If $\tau$ is a quantifier-free type without layout constraints then
%   \[
%     \sem{E[\Delta;C;\Gamma] \vdash e : \tau} \in \typesem{\tau}
%   \]
% \end{theorem}
%
% \begin{theorem}[Denotation continuity]
%   $\sem{E[\Delta;C;\Gamma] \vdash e : \tau_1 \ra \tau_2}$ is a continuous function from $\typesem{\tau_1}$ to $\typesem{\tau_2}$.
% \end{theorem}
%
% \begin{theorem}[Equational soundness]
%   \[
%       \boxmath{\sem{E[\Delta;C;\Gamma] \rhd e_1 \defeq e_2 : \tau}}
%     \models
%       \boxmath{\sem{E[\Delta;C;\Gamma] \vdash e_1 : \tau}}
%         =
%       \boxmath{\sem{E[\Delta;C;\Gamma] \vdash e_2 : \tau}}
%   \]
% \end{theorem}
%
