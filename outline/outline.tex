\documentclass[runningheads]{llncs}

\usepackage{graphicx}
\usepackage{proof}
\usepackage{stmaryrd}
\usepackage{amsmath}
\usepackage{amssymb}
\usepackage{xcolor}
\usepackage{tcolorbox}
\usepackage{tikz}
\usepackage{tikz-cd}
\usepackage{syntax}
\usepackage{listings}

\lstset{basicstyle=\ttfamily, mathescape=true, literate={~} {$\sim$}{1}}

% From https://tex.stackexchange.com/questions/235118/making-a-thicker-cdot-for-dot-product-that-is-thinner-than-bullet
\makeatletter
\newcommand*\bigcdot{\mathpalette\bigcdot@{.5}}
\newcommand*\bigcdot@[2]{\mathbin{\vcenter{\hbox{\scalebox{#2}{$\m@th#1\bullet$}}}}}
\makeatother

\newcommand {\Pika} {\textsf{Pika}}
\newcommand {\PikaCore} {\textsf{PikaCore}}

\newcommand {\instExpr} {\keyword{inst}}

\newcommand {\labinfer} [3] [] {\infer[{\textsc{#1}}]{#2}{#3}}

% \newcommand {\keyword} [1] {\textsf{#1}}
\newcommand {\keyword} [1] {\textbf{#1}}

\newcommand {\Int} {\keyword{Int}}
\newcommand {\Bool} {\keyword{Bool}}
\newcommand {\Type} {\keyword{Type}}
% \newcommand {\layout} [1] {\keyword{layout}(#1)}
\newcommand {\layout} {\keyword{layout}}

\newcommand {\data} {\keyword{data}}

\newcommand {\inL} {\keyword{inL}}
\newcommand {\inR} {\keyword{inR}}
\newcommand {\fold} {\keyword{fold}}
\newcommand {\unfold} {\keyword{unfold}}
\newcommand {\fst} {\keyword{fst}}
\newcommand {\snd} {\keyword{snd}}
\newcommand {\either} {\keyword{either}}

\newcommand {\outs} {\longrightarrow}

\newcommand {\Pair} {\otimes}
\newcommand {\Either} {\oplus}
\newcommand {\RecTy} {\mu}

\newcommand {\monic} {\rightarrowtail}

\newcommand {\type} {\textsf{type}}
\newcommand {\ctype} {\textsf{ctype}}
\newcommand {\adt} {\textsf{adt}}

\newcommand {\LayoutDefs} {\mathcal{L}}

\newcommand {\matchWith} [1] {\keyword{case}\,#1\,\keyword{of}\,}
\newcommand {\layoutmatch} [1] {\keyword{layoutcase}\,#1\,\keyword{of}\,}

\newcommand {\withIn} [1] {\keyword{with}\,#1\,\keyword{in}\,}
\newcommand {\letPure} [1] {\keyword{letpure}\,#1\,\keyword{in}\,}

\newcommand{\cond}{\textnormal{cond}}

\newcommand{\Pure}{\textnormal{Pure}}
\newcommand{\Spatial}{\textnormal{Spatial}}


% % Based on https://tex.stackexchange.com/questions/451786/how-do-i-put-a-circle-around-a-symbol
% \makeatletter
% \newcommand {\osep}{\mathbin{\mathpalette\make@circled\ast}}
% \newcommand{\make@circled}[2]{%
%   \ooalign{$\m@th#1\smallbigcirc{#1}$\cr\hidewidth$\m@th#1#2$\hidewidth\cr}%
% }
% \newcommand{\smallbigcirc}[1]{%
%   \vcenter{\hbox{\scalebox{0.77778}{$\m@th#1\bigcirc$}}}%
% }
% \makeatother

% \newcommand {\lowerExpr} {\keyword{lower}}
% \newcommand {\liftExpr} {\keyword{lift}}
\newcommand {\instantiate} {\keyword{instantiate}}
\newcommand {\apply} {\keyword{apply}}

\newcommand {\liftExpr} {\keyword{lift}}

\newcommand {\tyApply} {\mathbin{@}}

\newcommand {\isA} {\sim}

\newcommand {\ra} {\rightarrow}
\newcommand {\Ra} {\Rightarrow}

\newcommand {\Dargent} {{\sc Dargent}}

\newcommand {\highlight} [1] {\begin{tcolorbox}[hbox] #1 \end{tcolorbox}}

\newcommand {\tyLambda} {\mathrm{\Lambda}}

\newcommand {\argCount} [1] {N\sem{#1}}
\newcommand {\namesem} [1] {\Var^{\argCount{#1}}}

% \newcommand {\pcsubscript} {\textsf{PC}}
% \newcommand {\step} {\longrightarrow}
% \newcommand {\steps} {\longrightarrow^{*}}
% \newcommand {\pcstep} {\step_{\pcsubscript}}
% \newcommand {\pcsteps} {\steps_{\pcsubscript}}
% \newcommand {\pcNotionStep} {\rightsquigarrow_{\pcsubscript}}

\newcommand {\FV} {\textsf{FV}}

% \newcommand {\bigstep} {\Downarrow}
% \newcommand {\pcStep} {\bigstep_{\pcsubscript}}

% \newcommand {\defStep} {\bigstep_{\pcsubscript}}

\newcommand {\val} {\textsf{value}}

\newcommand {\fresh} [1] {#1 \textrm{ fresh}}

\newcommand {\recspec} [1] {\langle #1 \rangle}

\newcommand {\Layout} [1] {\keyword{Layout}(#1)}

% \newcommand {\todo} [1] {\text{{\color{red}TODO}: #1}}
%
\newcommand {\dom} {\textsf{dom}}
\newcommand {\range} {\textsf{range}}

% \newcommand {\ssl} [2] {\keyword{ssl}(#1) \{ #2 \}}
% \newcommand {\ssl} [1] {\keyword{ssl} \{ #1 \}}
\newcommand {\sslmath} [3] {\ssl\{ #1 \}\{ #2 \}\{ #3 \}}
\newcommand {\ssl} {\keyword{layout}}
\newcommand {\SSL} {\keyword{SSL}}

\newcommand {\synth} {\keyword{synth}}

\newcommand {\synthsep} {\mathbin{;;}}
\newcommand {\Not} {\keyword{not}}

\newcommand {\app} {\mathbin{++}}

\newcommand {\SuSLik} {SuSLik}
\newcommand {\SuSLang} {SuSLang}

% \newcommand {\Var} {\textsf{Var}}

% \newcommand {\Heap} {\textsf{Heap}}

\newcommand {\lifted} [1] {#1_{\bot}}

\newcommand {\eq} {\stackrel{\bigcdot}{=}}
\newcommand {\defeq} {:=}
\newcommand {\ok} [1] {\;\textsf{ok}_{#1}}

\newcommand {\typesem} [1] {\mathcal{A}_{#1}}

\newcommand {\True} {\keyword{True}}
\newcommand {\False} {\keyword{False}}

\newcommand {\generate} {\keyword{generate}}

\newcommand {\vdashpc} {\vdash_{\textsf{PC}}}

\newcommand {\fboxNewlines} {\\\\\\}

\newcommand {\layoutParams} {\textsf{params}}

\newcommand{\eg}{\emph{e.g.}}
\newcommand{\etal}{\emph{et~al.}}

\newcommand{\sep}{*}


\newcommand{\step} {\ensuremath{\longmapsto}}
\newcommand{\tstep} {\ensuremath{\Downarrow}}
\newcommand{\defstep} [1] {\ensuremath{\xmapsto{\textnormal{fn-def}}_{#1}}}
\newcommand{\Cstep} {\ensuremath{\longmapsto_{\mathfrak{C}}}}

\newcommand{\Expr} {\textnormal{Expr}}
\newcommand{\Heap} {\textnormal{Heap}}
\newcommand{\Store} {\textnormal{Store}}
\newcommand{\Var} {\textnormal{Var}}
\newcommand{\Loc} {\textnormal{Loc}}
\newcommand{\Val} {\textnormal{Val}}
\newcommand{\FsVal} {\textnormal{FsVal}}
% \newcommand{\labinfer} [3] [] {\infer[{\textsc{#1}}]{#2}{#3}}
% \newcommand{\fresh} {\mathbin{\#}}

\newcommand{\freshLoc} [1] {#1\;\textnormal{fresh}}
\newcommand{\freshVar} [1] {#1\;\textnormal{fresh}}

\newcommand{\FsStore} {\textnormal{FsStore}}
\newcommand{\VarSet} {V}

\newcommand{\lowerS} [2] {\texttt{lower}_{#1}(#2)}
\newcommand{\instantiateS} [2] {\texttt{inst}_{#1}(#2)}

\newcommand{\emptyheap} {\varnothing}

\newcommand{\emp} {\texttt{emp}}
\newcommand{\pointsto} {\ensuremath{\mathbin{\texttt{:->}}}}

\newcommand{\tname}[1]{\textsf{#1}}

\newcommand {\tool} {\Pika\xspace}
\newcommand {\suslik} {\SuSLik\xspace}


\newcommand{\sem} [1] {\llbracket#1\rrbracket}
\newcommand{\Tsem} [2] {\mathcal{T}\sem{#1}_{#2}}

\newcommand{\concrete} [1] {\textnormal{\;concrete$_{#1}$}}

\newcommand{\Start} [1] {{\color{cyan} #1}}
\newcommand{\End} [1] {{\color{orange} #1}}

\newcommand{\StartSigma} {\Start{\sigma}}
\newcommand{\EndSigma} {\End{\sigma'}}

\newcommand{\StartPair} {(\StartSigma, \Start{h})}
\newcommand{\EndPair} {(\EndSigma, \End{h'})}


\definecolor{pblue}{rgb}{0.13,0.13,1}
\definecolor{pgreen}{rgb}{0,0.5,0}
\definecolor{pred}{rgb}{0.7,0,0}
\definecolor{pgrey}{rgb}{0.46,0.45,0.48}
% \definecolor{ppurple}{rgb}{0.55,1,0,0.15}

\newcommand{\todo}[1]{\textcolor{pred}{(TODO: {#1})}}
\newcommand{\zy}[1]{\textcolor{pred}{(Ziyi: {#1})}}

\lstdefinelanguage{SynLang}{
  keywords={new, let, if, else, null, return, while},
  ndkeywords={bool, int, void, loc},
  mathescape=true,
  showspaces=false,
  showtabs=false,
  breaklines=true,
  showstringspaces=false,
  breakatwhitespace=true,
  lineskip=-0.9pt,
  morecomment=[l]{//}, % l is for line comment
  morecomment=[s]{/*}{*/}, % s is for start and end delimiter
  basewidth={0.54em, 0.4em},%
  basicstyle=\footnotesize\ttfamily,
  keywordstyle={\color{pred}\ttfamily\bfseries},
  ndkeywordstyle={\color{pblue}\ttfamily\bfseries},
  commentstyle={\color{ccomment}\itshape},
  numbers=none,
}



\newcommand{\pts}{\mapsto}
\newcommand{\code}[1]{\lstinline[language=SynLang,basicstyle=\small\ttfamily,mathescape=true]{#1}}

\newcommand{\rulename}[1]{\textsc{#1}}
\newcommand{\func}{\textit{func}\xspace}
\newcommand{\writer}{\rulename{Write}\xspace}
\newcommand{\writeri}{\rulename{WriteRO}\xspace}
\newcommand{\funcwrite}{\rulename{Funcwrite}\xspace}
\newcommand{\inplace}{\textit{inplace}\xspace}
\newcommand{\tempfuncalloc}{\rulename{Tempfuncalloc}\xspace}
\newcommand{\temppredalloc}{\rulename{Temppredalloc}\xspace}
\newcommand{\tempfuncfree}{\rulename{Tempfuncfree}\xspace}
\newcommand{\temppredfree}{\rulename{Temppredfree}\xspace}
\newcommand{\inplacewrite}{\rulename{Inplacewrite}\xspace}



% \newcommand{\mcode}[1]{{\ensuremath{\tt #1}}}
\newcommand{\mcode}[1]{{\ensuremath{#1}}}
\newcommand{\vars}[1]{\mathsf{Vars}\left({#1}\right)}

\newcommand{\set}[1]{\left\{{#1}\right\}}
\newcommand{\bl}[1]{{\color{pblue}{#1}}}
\newcommand{\asn}[1]{{\bl{\set{#1}}}}

\newcommand{\trans}[3]{\left.\bl{#1} \!\leadsto\! \bl{#2} \right| #3}
\newcommand{\translong}[3]{\left.\bl{#1} \!\leadsto\! \bl{#2} \right| #3}
\newcommand{\prog}{c}
\newcommand{\osep}{\ast}
\newcommand{\env}{\Gamma}
\newcommand{\deref}[1]{*{#1}}

\newcommand{\iann}[1]{\mcode{ {#1}}} %immutability annotation
\newcommand{\writeontop}[2]{\mathrel{\stackrel{\makebox[0pt]{\mbox{\tiny{#2}}}}{#1}}}
\newcommand{\spointsto}[2]{{#1}~{\pts}~{#2}}
\newcommand{\opointsto}[2]{\spointsto{\angled{#1}}{#2}}
\newcommand{\ispointsto}[3]{#1{\writeontop{\ \!\pts\ \!}{\iann{#3}}}{#2}} % simple points-to, no angled parentheses


% From https://tex.stackexchange.com/questions/502652/define-tcolorbox-in-math-mode
\newtcbox{\boxtext}{on line,colback=white,colframe=black,size=fbox,arc=3pt,boxrule=0.8pt}
\newcommand{\boxmath}[1]{\boxtext{$#1$}}

% https://tex.stackexchange.com/questions/24132/overline-outside-of-math-mode
\makeatletter
\newcommand*{\textoverline}[1]{$\overline{\hbox{#1}}\m@th$}
\makeatother



\begin{document}

\title{Reflections On Compilation With Separation Logic-Based Synthesis}
\maketitle

\section{Introduction}
\label{sec:introduction}

There is a tool called \SuSLik{} that will take separation logic specification and generate
a program in a C-like language. While working with this tool, it became clear that there is a strong
connection between the form of separation logic used by \SuSLik{} and functional programming languages such as Haskell and Standard ML.

In particular, we made the following observation. The SuSLik specification language has \textit{inductive predicates}. These are
allow the programmer to define the heap layout of various inductively defined data structures, such as singly-linked lists and binary trees.
However, we noticed that inductive predicates could also encode function definitions!

Using this fact, functions written in a functional programming language can be translated into inductive predicates. The translation
of a function definition into an inductive predicate is very similar to translation into ANF. However, this also gives us easy
interoperability with the rest of the SuSLik specification language. For example, not only can we define data structures as inductive
predicates, we can also dictate the heap layout of our data structures (similar to \Dargent\cite{Dargent}). These ideas are explored
with examples in Section~\ref{sec:examples}. The representation of data types as inductive predicates are explored in more formal detail
in Section~\ref{sec:types}.

We develop an ANF-like intermediate language. This also enables a straightforward translation into C (Section~\ref{sec:IR}).
In Section~\ref{sec:to-ssl} and Section~\ref{sec:to-c} we describe the two translations from this intermediate representation.

Finally, this enables us to implement a program synthesis feature at the level of the functional programming language (Section~\ref{sec:synth}).
The form this takes is similar to synthesis tools based on refinement types (such as \Synquid), but based on separation logic.

\begin{figure}
  % https://q.uiver.app/#q=WzAsNixbMiwxLCJcXHRleHRub3JtYWx7UGlrYUNvcmV9X1xcdGV4dHNme0FORn0iXSxbMywwLCJcXHRleHRub3JtYWx7U1NMIHByb3Bvc2l0aW9uc30iXSxbMywyLCJcXHRleHRub3JtYWx7QyBjb2RlfSJdLFszLDEsIlxcdGV4dG5vcm1hbHtIZWFwIG1hY2hpbmUgc2VtYW50aWNzfSJdLFswLDEsIlxcdGV4dG5vcm1hbHtQaWthfSJdLFsxLDEsIlxcdGV4dG5vcm1hbHtQaWthQ29yZX0iXSxbMCwxXSxbMCwzXSxbMCwyXSxbNCw1XSxbNSwwXV0=
\[\begin{tikzcd}
	&&& {\textnormal{SSL propositions}} \\
	{\textnormal{Pika}} & {\textnormal{PikaCore}} & {\textnormal{PikaCore}_\textsf{ANF}} & {\textnormal{Heap machine semantics}} \\
	&&& {\textnormal{C code}}
	\arrow[from=2-3, to=1-4]
	\arrow[from=2-3, to=2-4]
	\arrow[from=2-3, to=3-4]
	\arrow[from=2-1, to=2-2]
	\arrow[from=2-2, to=2-3]
\end{tikzcd}\]

  \caption{Translation pipeline}
  \label{fig:pipeline}
\end{figure}

\section{Examples}
\label{sec:examples}


\section{Type System}
\label{sec:types}

\section{Intermediate Representation}
\label{sec:IR}

\section{Translation to SSL}
\label{sec:to-ssl}

\section{Translation to C}
\label{sec:to-c}

\section{\texttt{synth} keyword}
\label{sec:synth}

\section{Formal Semantics}
\label{sec:semantics}

\section{Conclusion}
\label{sec:conclusion}

% \section{Introduction}

\Pika{} is a functional programming language with layouts and layout
polymorphism. \PikaCore{} is an intermediate representation of \Pika{} where all
layout polymorphism has been removed. That is, in \PikaCore, all layout
polymorphism is specialized to specific layouts and SSL assertions are
first-class values. Furthermore, \PikaCore{} is in ANF form.

From \PikaCore, the program can be translated to several different targets. This
pipeline is given in Fig.~\ref{fig:translation-pipeline}.


\begin{figure}
  % https://q.uiver.app/#q=WzAsNixbMiwxLCJcXHRleHRub3JtYWx7UGlrYUNvcmV9X1xcdGV4dHNme0FORn0iXSxbMywwLCJcXHRleHRub3JtYWx7U1NMIHByb3Bvc2l0aW9uc30iXSxbMywyLCJcXHRleHRub3JtYWx7QyBjb2RlfSJdLFszLDEsIlxcdGV4dG5vcm1hbHtIZWFwIG1hY2hpbmUgc2VtYW50aWNzfSJdLFswLDEsIlxcdGV4dG5vcm1hbHtQaWthfSJdLFsxLDEsIlxcdGV4dG5vcm1hbHtQaWthQ29yZX0iXSxbMCwxXSxbMCwzXSxbMCwyXSxbNCw1XSxbNSwwXV0=
\[\begin{tikzcd}
	&&& {\textnormal{SSL propositions}} \\
	{\textnormal{Pika}} & {\textnormal{PikaCore}} & {\textnormal{PikaCore}_\textsf{ANF}} & {\textnormal{Heap machine semantics}} \\
	&&& {\textnormal{C code}}
	\arrow[from=2-3, to=1-4]
	\arrow[from=2-3, to=2-4]
	\arrow[from=2-3, to=3-4]
	\arrow[from=2-1, to=2-2]
	\arrow[from=2-2, to=2-3]
\end{tikzcd}\]

  \caption{Translation pipeline}
  \label{fig:translation-pipeline}
\end{figure}


% \section{Syntax}

\subsection{Shared Syntax}

The following syntactic forms are shared between the \Pika{} surface language and \PikaCore.

\begin{align*}
  &i ::= \cdots \mid -2 \mid -1 \mid 0 \mid 1 \mid 2 \mid \cdots
  \\
  &b ::= \True{} \mid \False
  \\
  &e ::= x \mid i \mid b \mid \lambda x.\; e
  \\
  &\tau ::= \Int \mid \Bool \mid \tau \ra \tau
\end{align*}

\subsection{\Pika{} Syntax}
\begin{align*}
  &e ::= \cdots \mid \apply_{\alpha}(e) \mid \liftExpr_{\alpha}(e) \mid \Lambda (\alpha \isA \layout{X}).\; e
  \\
  &\tau ::= \cdots \mid X \mid \alpha \mid e\; e \mid (\alpha \isA \layout{X}) \Ra \tau
\end{align*}

\subsection{\PikaCore{} Syntax}
\begin{align*}
  &e ::= \cdots \mid e\; \overline{a} \mid \ssl{\overline{P}} \mid \withIn{x := e : \tau} e
  \\
  &a ::= \{ \overline{x} \} \mid \{ i \} \mid \{ b \}
  \\
  &\tau ::= \cdots \mid \SSL(n)
  \\
  &P ::= \ell \mapsto e
  \\
  &\ell ::= x \mid (x + n)
  \\
  &n ::= 0 \mid 1 \mid 2 \mid \cdots
\end{align*}


% \section{Static Semantics}

In this section, we discuss two different parts of the Pika compiler: the type system and the translation
from surface Pika to PikaCore. The translation is accomplished with two stages. First, ANF translation. Then,
all layout definitions are substituted into their use sites during a desugaring stage.

\subsection{Type System}

$E$ is the global environment. We write:

\begin{itemize}
  \item $E[\Delta]$ for $E$ extended with the local list of types $\Delta$
  \item $E[\Delta;C;\Gamma]$ for $E$ extended with the local list of types $\Delta$, the
    layout constraints $C$ and the typing context $\Gamma$.
\end{itemize}

\noindent
In particular, $E$ is a set that includes:
\begin{itemize}
  \item $(X : \Type)$ for every user-defined algebraic data type $X$
  \item $(\alpha \isA \layout{X})$ for every user-defined layout $\alpha$ for a data type $X$
\end{itemize}

\noindent
This follows the style of typing rules given in the Coq manual.~\cite{Coq-typing-rules} The use of layout constraints and layout
polymorphism largely follows the type system used by \Dargent, though there are some differences.~\cite{Dargent} These differences
are discussed further in Section~\ref{sec:related-work}.
% \[
%   \lowerExpr_\ell(e) = \apply_\ell(e)
% \]

A general type system is given in Fig.~\ref{fig:type-form}, Fig.~\ref{fig:layout-constraints} and Fig.~\ref{fig:typing-judgment}. In
Fig.~\ref{fig:type-form} and Fig.~\ref{fig:typing-judgment}, all rules that deal with layouts are boxed and highlighted. Type signatures
in Pika are restricted to only have types of the form

\[
  \overline{\alpha_i \isA \layout{X_i}} \Ra \tau
\]

\noindent
where $\tau$ is a type with no layout constraints and the type variables $\overline{\alpha}$ are allowed to occur free.

\begin{figure}
  \[
  \begin{array}{c}
    \fbox{$E[\Delta] \vdash \tau ~\type$}
    \\\\\\
    \labinfer{E[\Delta, \tau ~\type] \vdash \tau ~\type}{}
    \\\\
    \labinfer{E[\Delta] \vdash X ~\type}{(X : \Type) \in E}
    \\\\
    \labinfer{E[\Delta] \vdash \Int ~\type}{}
    ~~~
    \labinfer{E[\Delta] \vdash \Bool ~\type}{}
    \\\\
    \labinfer{E[\Delta] \vdash \tau_1 \ra \tau_2 ~\type}{E[\Delta] \vdash \tau_1 ~\type & E[\Delta] \vdash \tau_2 ~\type}
    \\\\
    \highlight{
      \labinfer{E[\Delta] \vdash (\alpha \isA \layout{X}) \Ra \tau ~\type}{E[\Delta,\alpha ~\type] \vdash \tau ~\type}
    }
  \end{array}
  \]
  \caption{Type formation rules}
  \label{fig:type-form}
\end{figure}

% \begin{figure}
%   \[
%   \begin{array}{c}
%     \labinfer{\Delta \vdash \alpha ~\ctype}{\Delta \vdash \alpha ~\type}
%     ~~~
%     \labinfer{\Delta \vdash \forall (\alpha : \layout{A}).\; \beta ~\ctype}{\Delta \vdash A ~\adt & \Delta,\alpha ~\type \vdash C \Ra \beta ~\ctype}
%   \end{array}
%   \]
%   \caption{Constrained type formation rules}
% \end{figure}


% For info on typing rules for pattern matching, see:
%  - https://coq.inria.fr/distrib/current/refman/language/core/inductive.html#inductive-definitions
%  - Lecture Notes on Pattern Matching
%    15-814: Types and Programming Languages
%    Frank Pfenning
%    Lecture 12
%    Thursday, October 8, 2020

\begin{figure}
  \[
    \begin{array}{c}
      \fbox{$E[C] \vdash \alpha \isA \layout{X}$}
      \\\\\\
      \labinfer{E[C] \vdash \Int \isA \layout{\Int}}{}
      ~~~
      \labinfer{E[C] \vdash \Bool \isA \layout{\Bool}}{}
      \\\\
      \labinfer{E[C] \vdash L \isA \layout{X}}{(L \isA \layout{X}) \in E}
      \\\\
      \labinfer{E[C, \alpha \isA \layout{X}] \vdash \alpha \isA \layout{X}}{}
      % TODO: Should the rule below be included?
      % \\\\
      % \labinfer{E[\Delta] \vdash (\ell_1 \ra \ell_2) \isA \layout{A \ra B}}{E[\Delta] \vdash \ell_1 \isA \layout{A} & E[\Delta] \vdash \ell_2 \isA \layout{B}}
    \end{array}
  \]
  \caption{Layout constraint relation $\isA$}
  \label{fig:layout-constraints}
\end{figure}

\begin{figure}
  \[
    \begin{array}{c}
      \fbox{$E[\Delta;C;\Gamma] \vdash e : \tau$}
      \\\\\\
      \labinfer[T-Int]{E[\Delta;C;\Gamma] \vdash i : \Int}{i \in \mathbb{Z}}
      ~~~
      \labinfer[T-Bool]{E[\Delta;C;\Gamma] \vdash b : \Bool}{b \in \mathbb{B}}
      \\\\
      \labinfer[T-Var]{E[\Delta;C;\Gamma,v : \tau] \vdash v : \tau}{}
      \\\\
      % % \labinfer[T-Int-Layout]{\Delta;\Gamma \vdash \Int : \layout{\Int}}{}
      % % ~~~
      % % \labinfer[T-Bool-Layout]{\Delta;\Gamma \vdash \Bool : \layout{\Bool}}{}
      % % \\\\
      % % \labinfer{\Delta;\Gamma \vdash \alpha \ra \beta : \layout{A \ra B}}{\Delta;\Gamma \vdash \alpha : \layout{A} & \Delta;\Gamma \vdash \beta : \layout{B}}
      \highlight{
        \labinfer[T-Layout-Lambda]{E[\Delta;C;\Gamma] \vdash \Lambda (\alpha \isA \layout{X}).\; e : (\alpha \isA \layout{X}) \Ra \tau}
        {E[\Delta] \vdash \alpha ~\type & E[\Delta;C,\alpha \isA \layout{X};\Gamma] \vdash e : \tau}
      }
      % % \\\\
      % % \labinfer[T-Lower]{\Delta;\Gamma \vdash \lowerExpr_\alpha(e) : \forall (\alpha : \layout{A}).\; \alpha}
      % %   {\Delta, \alpha ~\type;\Gamma, \alpha : \layout{A} \vdash e : A}
      \\\\
      \highlight{
      \labinfer[T-Layout-App]{E[\Delta;C;\Gamma] \vdash \apply_{\alpha'}(e) : \tau[\alpha\mapsto\alpha']}
        {E[C] \vdash \alpha' \isA \layout{X} & E[\Delta;C;\Gamma] \vdash e : (\alpha \isA \layout{X}) \Ra \tau}
      }
      \\\\
      \highlight{
      % \textnormal{{\color{red} TODO:} Does this rule make sense?}
      % \\
        \labinfer[T-Layout-Lift]{E[\Delta;C;\Gamma] \vdash \liftExpr(e) : A}{E[C] \vdash \alpha \isA \layout{X} & E[\Delta;C;\Gamma] \vdash e : \alpha}
      }
      % \labinfer[T-Instantiate]{\Delta;\Gamma \vdash \instantiate_{\alpha,\beta}(e) : A \ra B}
      %   {\Delta;\Gamma \vdash e : \alpha \ra \beta & \Delta;\Gamma \vdash \alpha : \layout{A} & \Delta;\Gamma \vdash \beta : \layout{B}}
      % \\\\
      % \fbox{{\color{red} TODO:} Finish this rule}
      % \\
      % \labinfer[T-Match]{\matchWith{e_0} C_1\; \overline{x}_1 \Ra e_1 \mid \cdots \mid C_n\; \overline{x}_n \Ra e_n}{}
      \\\\
      % \labinfer[T-Layout-Def]{E[] \vdash f : \layout{A}}{}
      % \\\\
      \labinfer[T-Lambda]{E[\Delta;C;\Gamma] \vdash \lambda x.\; e : \tau_1 \ra \tau_2}
        {E[\Delta;C;\Gamma, x : \tau_1] \vdash e : \tau_2}
      \\\\
      \labinfer[T-App]{E[\Delta;C;\Gamma] \vdash e_1\; e_2 : \tau_2}
        {E[\Delta;C;\Gamma] \vdash e_1 : \tau_1 \ra \tau_2
        &E[\Delta;C;\Gamma] \vdash e_2 : \tau_1}
      \\\\
      \labinfer[T-InL]{E[\Delta;C;\Gamma] \vdash \inL(e) : \tau_1 \Either \tau_2}{E[\Delta;C;\Gamma] \vdash e : \tau_1}
      ~~~
      \labinfer[T-InR]{E[\Delta;C;\Gamma] \vdash \inR(e) : \tau_1 \Either \tau_2}{E[\Delta;C;\Gamma] \vdash e : \tau_2}
      \\\\
      \labinfer[T-Pair]{E[\Delta;C;\Gamma] \vdash (e_1, e_2) : \tau_1 \Pair \tau_2}{E[\Delta;C;\Gamma] \vdash e_1 : \tau_1 & E[\Delta;C;\Gamma] \vdash e_2 : \tau_2}
      \\\\
      \labinfer[T-Fst]{E[\Delta;C;\Gamma] \vdash \fst(e) : \tau_1}{E[\Delta;C;\Gamma] \vdash e : \tau_1 \Pair \tau_2}
      ~~~
      \labinfer[T-Snd]{E[\Delta;C;\Gamma] \vdash \snd(e) : \tau_2}{E[\Delta;C;\Gamma] \vdash e : \tau_1 \Pair \tau_2}
      \\\\
      \labinfer[T-Match]{E[\Delta;C;\Gamma] \vdash \matchWith{e} \inL(x) \Ra e_1 \mid \inR(y) \Ra e_2 : \tau}
        {E[\Delta;C;\Gamma] \vdash e : \tau_1 \Either \tau_2
        &E[\Delta;C;\Gamma, x : \tau_1] \vdash e_1 : \tau
        &E[\Delta;C;\Gamma, y : \tau_2] \vdash e_2 : \tau}
      \\\\
      \labinfer[T-Fold]{E[\Delta;C;\Gamma] \vdash \fold_\tau(e) : \tau}
        {\tau = \mu\alpha.\tau_1 & E[\Delta;C;\Gamma] \vdash e : \tau_1[\alpha\mapsto\tau]}
      \\\\
      \labinfer[T-Unfold]{E[\Delta;C;\Gamma] \vdash \unfold_\tau(e) : \tau_1[\alpha\mapsto\tau]}
        {\tau = \mu\alpha.\tau_1 & E[\Delta;C;\Gamma] \vdash e : \tau}

      \\\\
      \highlight{
      \labinfer[T-SSL]{E[\Delta;C;\Gamma] \vdash \ssl{\overline{x}}{\overline{x \mapsto e}} : \SSL({\overline{\langle x, \tau \rangle}})}
        {E[\Delta;C;\Gamma] \vdash e_i : \tau_i \ensuremath{\textnormal{ for each $e_i$}}}
      }
    \end{array}
  \]
  \caption{Typing judgment}
  \label{fig:typing-judgment}
\end{figure}

\begin{figure}
  \[
    \begin{array}{c}
      \labinfer[T-Def-Eq]{E[\Delta;C;\Gamma] \rhd e_1 \defeq e_2 : \tau}
        {E[\Delta;C;\Gamma] \vdash e_1 : \tau
        &E[\Delta;C;\Gamma] \vdash e_2 : \tau}
    \end{array}
  \]
  \caption{Definitional equation typing}
  \label{fig:def-eq}
\end{figure}

% \begin{figure}
%   \[
%     \begin{array}{c}
%       % \labinfer[T-Lower]{\Delta;\Gamma \vdash \lowerExpr_\ell'(e) : \tau[\ell\mapsto\ell']}
%       %   {\vdash \ell' \isA \layout{A} & \Delta;\Gamma \vdash e : \forall (\ell \isA \layout{A}).\; \tau}
%       % \\\\
%       \labinfer[T-Instantiate]{\Delta;\Gamma \vdash \instantiate_{\ell_1',\ell_2'}(e) : (\alpha \ra \beta)[\ell_1\mapsto\ell_1'][\ell_2\mapsto\ell_2']}
%         {\begin{gathered}
%           \Delta \vdash \ell_1' \isA \layout{A}
%           ~~~ \Delta \vdash \ell_2' \isA \layout{B}
%           \\ \Delta;\Gamma \vdash e : \forall (\ell_1 \isA \layout{A}).\; \forall (\ell_2 \isA \layout{B}).\; \alpha \ra \beta
%         \end{gathered}}
%     \end{array}
%   \]
%   \caption{Derived types for layout constructs}
% \end{figure}

\subsection{ANF Translation}

All nested function applications are turned into \verb|with-in| expressions.

\subsection{Desugaring}

Well-typed functions written in the surface language
are desugared into a simplified language with no layout polymorphism. Instead of using layout polymorphism, this language explicitly manipulates layouts.

% \begin{figure}
%   \caption{Additional typing rules for desugared language}
% \end{figure}

\begin{figure}
  \[
    \labinfer{E \vdash \apply_\alpha(e) \desugars \cdots}{(\alpha = e'') \in E & \textrm{{\color{red}TODO:} finish}}
  \]
  \caption{Desugaring judgment}
  \label{fig:desugaring}
\end{figure}

% \section{Elaboration}

The body of a layout polymorphic \Pika{} function will often make use of the 
layout parameters that it is given. For example, consider the function \texttt{singleton}.

\begin{align*}
  &\texttt{singleton} : (a \isA \layout{\textrm{List}}) \Ra \Int \ra a\\
  &\texttt{singleton}\; i := \textrm{Cons}\; i\; \textrm{Nil}
\end{align*}

\noindent
We know just by looking at the type that we want to give back a \textrm{List} using
whatever layout is given as $a$. However, without elaboration, this is not enough. We must
explicitly lower the constructors to use the layout $a$ like this:

\begin{align*}
  &\texttt{singleton} : (a \isA \layout{\textrm{List}}) \Ra \Int \ra a\\
  &\texttt{singleton}\; [a]\; i := \apply_a(\textrm{Cons}\; i\; (\apply_a(\textrm{Nil})))
\end{align*}

\noindent
Elaboration will turn the first definition of \texttt{singleton} into the second definition.


% \section{Denotational Semantics}

We give a denotational semantics for \PikaCore. Since \Pika{} is translated to \PikaCore, this can
also be seen as a denotational semantics for \Pika.

The function $\argCount{\tau}$ gives the number of fields in abstract
heap associated to the type $\tau$. In the case of function types, this
is the number of fields associated to the \textit{result} of the function.

\begin{align*}
  % &\typesem{\Int} = \Var \ra \lifted{\mathbb{Z}}
  % \\
  % &\typesem{\Bool} = \Var \ra \lifted{\mathbb{B}}
  % \\
  % &\typesem{\SSL(\langle x_1, \tau_1 \rangle, \cdots, \langle x_n, \tau_n \rangle)} = \Var^n \ra \lifted{\Heap}
  % \\
  % &\typesem{\tau_1 \ra \tau_2} = \typesem{\tau_1} \ra \typesem{\tau_2}
  &\typesem{\Int} = \namesem{\Int} \ra \Heap
  \\
  &\typesem{\Bool} = \namesem{\Bool} \ra \Heap
  \\
  &\typesem{\SSL(n)} = \namesem{\SSL(n)} \ra \Heap
  \\
  &\typesem{\tau_1 \ra \tau_2} = \namesem{\tau_1 \ra \tau_2} \ra \typesem{\tau_1} \ra \typesem{\tau_2}
\end{align*}

\begin{align*}
  &\argCount{\Int} = 1
  \\
  &\argCount{\Bool} = 1
  \\
  &\argCount{\SSL(n)} = n
  \\
  &\argCount{\tau_1 \ra \tau_2} = \argCount{\tau_2}
\end{align*}
\\

\begin{figure}
\begin{center}
  \fbox{
    $\sem{e} \in \typesem{\tau}\textrm{ where $E[\bullet] \vdash e : \tau$}$
  }
\end{center}
\begin{align*}
  &\sem{i}_r = \{ r \mapsto i \}\tag{where $i \in \mathbb{Z}$}
  \\
  &\sem{b}_r = \{ r \mapsto b \}\tag{where $i \in \mathbb{B}$}
  \\
  &\sem{\withIn{a := e_1} e_2}_r = \sem{e_1}_a \osep \sem{e_2}_r
  \\
  &\sem{\ssl{\overline{x}}\{\overline{x_i \mapsto e_i}\}}_r =
    \sem{e_1}_{r_1} \osep \cdots \osep \sem{e_n}_{r_n}
  \\
  &\sem{e\; v}_r =
      \sem{e}_r(\sem{v}_{-})
  \\
  &\sem{\lambda x.\; e}_r(f)
      = \sem{e}_r \osep f(x)
\end{align*}
  \caption{Denotation function for \PikaCore ({\color{red}TODO}: Finish)}
\end{figure}

% \subsection{Equational soundness}
%
% \begin{theorem}[Type denotation]
%   If $\tau$ is a quantifier-free type without layout constraints then
%   \[
%     \sem{E[\Delta;C;\Gamma] \vdash e : \tau} \in \typesem{\tau}
%   \]
% \end{theorem}
%
% \begin{theorem}[Denotation continuity]
%   $\sem{E[\Delta;C;\Gamma] \vdash e : \tau_1 \ra \tau_2}$ is a continuous function from $\typesem{\tau_1}$ to $\typesem{\tau_2}$.
% \end{theorem}
%
% \begin{theorem}[Equational soundness]
%   \[
%       \boxmath{\sem{E[\Delta;C;\Gamma] \rhd e_1 \defeq e_2 : \tau}}
%     \models
%       \boxmath{\sem{E[\Delta;C;\Gamma] \vdash e_1 : \tau}}
%         =
%       \boxmath{\sem{E[\Delta;C;\Gamma] \vdash e_2 : \tau}}
%   \]
% \end{theorem}
%

% \section{C Code Generation}

The translation of \PikaCore{} function into C is divided into the stages.

\begin{enumerate}
  \item \label{stage:in-out} For each branch, determine the abstract heap for the input and a collection of
    heaplets for the output.
  \item Generate the appropriate assignment statements, \verb|malloc| calls, etc from the
    previous stage
\end{enumerate}

\subsection{Examples}

\subsubsection{leftList}
First, lets look at a \Pika{} function that takes a tree and gives back a list by traversing the leftmost branches.

\begin{lstlisting}
%generate leftList[TreeLayout, Sll]

data List := Nil | Cons Int List;
data Tree := Tip | Node Int Tree Tree;

Sll : layout(List);
Sll Nil := emp;
Sll (Cons h t) := x :-> h, (x+1) :-> t, Sll t;

TreeLayout : layout(Tree);
TreeLayout Tip := emp;
TreeLayout (Node v left right) :=
  y :-> v, (y+1) :-> left, (y+2) :-> right,
  TreeLayout left, TreeLayout right;

leftList : (a ~ layout(Tree), b ~ layout(List)) => a -> b
leftList Tip := Nil;
leftList [a b] (Node v left right) :=
  Cons v ($\instExpr_{\verb|a|,\verb|b|}$(leftList) left);
\end{lstlisting}

\noindent
This is first translated into \PikaCore:

\begin{lstlisting}
leftList' : SSL(1) -> SSL(1)
leftList' {} = ssl {}
leftList' {x :-> h, (x+1) :-> left, (x+2) :-> right} =
  with nxt := leftList' {left} : SSL(1)
  in
  ssl {y :-> h, (y+1) :-> nxt}
\end{lstlisting}

\noindent
In Stage~\ref{stage:in-out}


% \section{Related Work}
\label{sec:related-work}

\tool is built upon the \suslik synthesis framework. \suslik provides
a synthesis mechanism for heap-manipulating programs using a variant
of separation logic.~\cite{polikarpova:2019:suslik} However, it does
not have any high-level abstractions. In particular, writing \suslik
specifications directly involves a significant amount of pointer
manipulation. Further, it does not provide abstraction over specific
memory layouts. As described in \autoref{sec:language}, \tool
addresses these limitations.

The \tname{Dargent} language~\cite{chen:2023:dargent} also includes a
notion of layouts and layout polymorphism for a class of algebraic
data types, which differs from our treatment of layouts in two primary
ways:

\begin{enumerate}

\item In \tool, abstract memory locations (with offsets) are used. In
  contrast, \tname{Dargent} uses offsets that are all
  relative to a single ``beginning'' memory location. The \tool
  approach is more amenable to heap allocation, though
  this requires a separate memory manager of some kind. This is
  exposed in the generated language with \verb|malloc| and
  \verb|free|.
  On the other hand, the technique taken by \tname{Dargent} allows
  for greater control over memory management. This makes
  dealing with memory more complex for the programmer, but it is no
  longer necessary to have a separate memory manager.

  \item Algebraic data types in the present language include
    \emph{recursive} types and,
    as a result, \tool has recursive layouts for these ADTs. This
    feature is not currently available in \tname{Dargent}.
  \end{enumerate}

Furthermore, layout polymorphism also works differently. While
\tname{Dargent} tracks layout instantiations at a type-level with type
variables, in the present work we simply only check to see if a layout
is valid for a given type when type-checking. In particular, we cannot
write type signatures that \textit{require} the same layout in
multiple parts of the type (for instance, in a function type
\verb|List -> List| we have no way at the type-level of requiring that
the argument \verb|List| layout and the result \verb|List| layout are
the same). \Pika{} 2 makes progress towards eliminating this restriction but, as of now, it is layout monomorphic. This more rudimentary approach that \tool currently takes
could be extended in future work.
%
Overall, the examples in the \tname{Dargent} paper tend to focus on
the manipulation of integer values. In contrast, we have focused
largely on data structure manipulation, which follow the primary
motivation of \suslik. \autoref{fig:dargent-comparison} has a comparison of some of the major features of \Pika{} and \tname{Dargent}.

\begin{figure}
  \begin{tabular}{|c|c|c|}
    \hline
    Feature & \Pika & \tname{Dargent}\\
    \hline
    Bit-level layouts & \xmark & \cmark\\
    Recursive layouts & \cmark & \xmark\\
    Synthesis from type signatures & \cmark & \xmark\\
    Custom record accessors & \xmark & \cmark\\
    \hline
  \end{tabular}
  \caption{Comparison to \tname{Dargent}}
  \label{fig:dargent-comparison}
\end{figure}

\tname{Synquid} is another synthesis framework with a functional
surface language. While \tname{Synquid} allows an even higher-level
program specification than \tool through its liquid types, it does not
provide any access to low-level data structure
representation.~\cite{polikarpova:2016:synquid} In contrast, \tool's
level of abstraction is similar to that of a traditional functional
language but, similar to \tname{Dargent}, it also allows control over
the data structure representation in memory. The \synth{} feature
has similarities to \tname{Synquid}'s approach to program synthesis, but
\Pika{} 2's \synth{} allows lower level constraints to be expressed.




\bibliographystyle{splncs04}
\bibliography{refs}

\end{document}

