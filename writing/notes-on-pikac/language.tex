\section{Syntax}

\Pika{} code is translated into \PikaCore{} before being translated into the target language. \PikaCore{}
is a layout monomorphic version of \Pika{}. While it retains most of the functional syntax of \Pika,
all data structures are manipulated as collections of SSL heaplets. This includes both construction
of data structure values and pattern matching. Also, the argument of every function application in
\PikaCore{} is a literal or a variable. This is because it is entirely given in a kind of ANF, using
the \verb|with-in| construct to introduce an ANF binding.

The syntactic form that introduces an SSL assertion in \PikaCore{} can be seen as a kind of binding
form, the values of which can be ``applied''. For example, $\sslmath{x, y}{T}{x \mapsto y, (x+1) \mapsto z}$
can be seen as an SSL assertion that is parametrized over two locations: $x$ and $y$. Also, the name
$z$ occurs free. When this layout value is applied to two locations, $z$ must remain a fresh variable in
the use-site.

There is only one situation in \PikaCore{} in which layouts can be applied. This is when a function
is called. A function, generally, takes at least one layout value as input and produces a layout value as
output. The \verb|with-in| construct allows both of the input and output layouts to be applied with
appropriate variables, representing the input and output variables that the function will use when
it's called.

For example, given a function

\begin{align*}
  &f\; \{ x \mapsto a \} := \sslmath{y, z}{T}{(y+1) \mapsto a, z \mapsto w}
\end{align*}

\noindent
we can apply it like this

\begin{align*}
  \withIn{\{p, q\} := f\; \{s\}} e
\end{align*}

\noindent
If $s$ is nonzero (meaning $s$ points to something and this branch of $f$ is actually used), this will result in an SSL assertion
for the input $\{s \mapsto a\}$ and an SSL assertion for the output $\{p \mapsto a, q \mapsto w\}$ where $w$ is fresh. These
SSL assertions are combined with the SSL assertion given by the expression $e$.

\subsection{Shared Syntax}

The following syntactic forms are shared between the \Pika{} surface language and \PikaCore.

\begin{align*}
  &i ::= \cdots \mid -2 \mid -1 \mid 0 \mid 1 \mid 2 \mid \cdots
  \\
  &b ::= \True{} \mid \False
  \\
  &e ::= x \mid i \mid b \mid \lambda x.\; e
  \\
  &\tau ::= \Int \mid \Bool \mid \alpha \mid \tau \ra \tau
\end{align*}

\subsection{\Pika{} Syntax}
\begin{align*}
  &e ::= \cdots \mid e\; e \mid \apply_{\alpha}(e) \mid \Lambda (\alpha \isA \Layout{X}).\; e
  \\
  &\tau ::= \cdots \mid X \mid (\alpha \isA \Layout{X}) \Ra \tau
  \\
  &D ::= \overline{B}
  \\
  &B ::= f\; (\textrm{Ctr}\; \overline{x}) := e
\end{align*}

\subsection{\PikaCore{} Syntax}
\begin{align*}
  &e ::= \cdots \mid e\; \overline{a} \mid \sslmath{\overline{x}}{T}{\overline{h}} \mid \withIn{\{ \overline{x} \} := e} e
  \\
  &a ::= \{ \overline{x} \} \mid \{ i \} \mid \{ b \}
  \\
  &\tau ::= \cdots \mid A
  \\
  &h ::= \ell \mapsto e
  \\
  &T ::= \overline{\ell : \tau}
  % \\
  % &d ::= \recspec{\overline{n : \tau}}
  \\
  &\ell ::= x \mid (x + n)
  \\
  &n ::= 0 \mid 1 \mid 2 \mid \cdots
  \\
  &D ::= \overline{B}
  \\
  &B ::= f\; \{ \overline{h} \} := e
\end{align*}

The construct $\sslmath{\overline{x}}{T}{\overline{h}}$ is an SSL assertion with heaplets $\overline{h}$ and
bound variables $\overline{x}$. Note that it can have free variables as well.

$\mu\alpha.\; \SSL(\overline{x : \tau})$ is an equi-recursive type, with $\mu$ being the recursive binder,
introducing the type variable $\alpha$. When
the variable bound by $\mu$ does not appear in any of the $\tau$, we leave out the binder and write $\SSL(\overline{x : \tau})$
as a shorthand.


% $d$ is a record specification, given by index. Consider the following SSL assertion.
% \[
%   \sslmath{x,y}{x \mapsto 1, (x+2) \mapsto \True, y \mapsto 22, (y+1) \mapsto 0}
% \]
% This could have the type $\SSL(x : \recspec{0 : \Int, 2 : \Bool}, y : \recspec{0 : \Int, 1 : \Int})$.
% It could have other types as well, however. For example, $y$ could also fit the
% layout of a singly-linked list. We might not know at a type-level how long a list is. So, the
% assertion also has the type
% $\SSL(x : \recspec{0 : \Int, 2 : \Bool}, y : \recspec{0 : \Layout})$. We use $\Layout$ to indicate
% that we only know 

% $d$ is a record specification, given by index. Consider the following SSL assertion.
% \[
%   \sslmath{x,y}{x \mapsto 1, (x+2) \mapsto \True, y \mapsto 22, (y+1) \mapsto 0}
% \]
% This could have the type $\SSL(x : \recspec{0 : \Int, 2 : \Bool}, y : \recspec{0 : \Int, 1 : \Int})$.
% It could have other types as well, however. For example, $y$ could also fit the
% layout of a singly-linked list. If we have such a singly-linked list layout
% named \verb|Sll|, then the assertion would also have the type
% $\SSL(x : \recspec{0 : \Int, 2 : \Bool}, y : \recspec{0 : \texttt{Sll}})$. Regarding $y$, this
% type says that, at offset $0$, $y$ satisfies the \texttt{Sll} layout.

% The general form of a one-parameter \PikaCore{} function definition is
% \begin{align*}
%   &f : \SSL(m) \ra \SSL(n)\\
%   &f\; \{ \overline{\ell \mapsto y} \} \outs \{ \overline{r} \} := e\\
%   &f\; \{ \overline{\ell' \mapsto y} \} \outs \{ \overline{r} \} := e'\\
%   &\cdots
% \end{align*}
%
% \noindent
% where $\overline{r}$ are the names of the output parameters. These output parameters
% must be the same in each branch and it is also required that the number of output parameters matches the type signature: $|\overline{r}| = n$.
%
% Output parameters are given in \PikaCore{} as part of the \verb|with-in| construct:
%
% \begin{align*}
%   \withIn{\{ a, b \} := f\; \{ x \}} e
% \end{align*}
%
% This applies the function $f$ to the input parameter $x$ and gives $a$ and $b$ as output parameters. These two
% output parameters of the application are brought into scope for the \PikaCore{} expression $e$.
%
