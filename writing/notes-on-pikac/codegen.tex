\section{C Code Generation}

The translation of \PikaCore{} function into C is divided into the stages.

\begin{enumerate}
  \item \label{stage:in-out} For each branch, determine the abstract heap for the input and a collection of
    heaplets for the output.
  \item Generate the appropriate assignment statements, \verb|malloc| calls, etc from the
    previous stage
\end{enumerate}

\subsection{Example: leftList}
First, lets look at a \Pika{} function that takes a tree and gives back a list by traversing the leftmost branches.

$\begin{aligned}
  &\generate\; \textrm{leftList} : \textrm{TreeLayout} \ra \textrm{Sll}\\
  \\%
  &\data\; \textrm{List} := \textrm{Nil} \mid \textrm{Cons}\; \Int\; \textrm{List}\\
  &\data\; \textrm{Tree} := \textrm{Tip} \mid \textrm{Node}\; \Int\; \textrm{Tree}\; \textrm{Tree}\\
  \\%
  &\textrm{Sll} : \Layout{\textrm{List}}\\
  &\textrm{Sll}\; \textrm{Nil} := \ssl\{x\}\{\}\\
  &\textrm{Sll}\; (\textrm{Cons}\; h\; t) := \ssl\{x\}\{ x \mapsto h, (x+1) \mapsto n, \textrm{Sll}\; t\; \{n\} \}\\
  \\%
  &\textrm{TreeLayout} : \Layout{\textrm{Tree}}\\
  &\textrm{TreeLayout}\; \textrm{Tip} := \ssl\{x\}\{\}\\
  &\begin{aligned}
    \textrm{Tr}&\textrm{eeLayout}\; (\textrm{Node}\; v\; \textrm{left}\; \textrm{right}) :=\\
        &\ssl\{x\}\{ x \mapsto v, (x+1) \mapsto p, (x+2) \mapsto q, \textrm{TreeLayout}\; \textrm{left}\; \{p\}, \textrm{TreeLayout}\; \textrm{right}\; \{q\}\}
   \end{aligned}\\
  \\%
  &\textrm{leftList} : (a \isA \Layout{\textrm{Tree}}, b \isA \Layout{\textrm{List}}) \Ra a \ra b\\
  &\textrm{leftList}\; \textrm{Tip} := \textrm{Nil}\\
  &\textrm{leftList}\; [a, b]\; (\textrm{Node}\; v\; \textrm{left}\; \textrm{right}) :=
      \textrm{Cons}\; v\; (\textrm{leftList}_{a,b}\; \textrm{left})
\end{aligned}$
\\

% \begin{lstlisting}
% %generate leftList[TreeLayout, Sll]
%
% data List := Nil | Cons Int List;
% data Tree := Tip | Node Int Tree Tree;
%
% Sll : List >-> SSL(1);
% Sll Nil -> {x} := emp;
% Sll (Cons h t) -> {x} := x :-> h, (x+1) :-> nxt, Sll t {nxt};
%
% TreeLayout : Tree >-> SSL(1);
% TreeLayout Tip -> {x} := emp;
% TreeLayout (Node v left right) -> {x} :=
%   y :-> v, (y+1) :-> p, (y+2) :-> q,
%   TreeLayout left {p}, TreeLayout right {q} ;
%
% leftList : (a ~ layout(Tree), b ~ layout(List)) =>
%   a -> b
% leftList Tip := Nil;
% leftList [a b] (Node v left right) :=
%   Cons v (leftList$_{\verb|a|,\verb|b|}$ left);
% \end{lstlisting}

\noindent
This is first translated into \PikaCore:
\\

$\begin{aligned}
  &\textrm{leftList'} : \SSL(x : \Int, (x+1) : \textrm{TreeLayout}, (x+2) : \textrm{TreeLayout}) \ra \SSL(r : \Int, (r+1) : \textrm{Sll})\\
  &\textrm{leftList'}\; \{\} := \sslmath{r}{r : \Int, (r+1) : \textrm{Sll}}{}\\
  &\begin{aligned}
    \textrm{le}&\textrm{ftList'}\; \{x \mapsto h, (x+1) \mapsto p, (x+2) \mapsto q\} :=\\
      &\withIn{\{t\} := \textrm{leftList'}\; \{p\}}\\
      &\sslmath{r}{r : \Int, (r+1) : \textrm{Sll}}{r \mapsto h, (r+1) \mapsto t}
   \end{aligned}
\end{aligned}$
\\

% \begin{lstlisting}
% leftList' : SSL(1) -> SSL(1);
% leftList' {} -> {r} := {};
% leftList' {x :-> h, (x+1) :-> p, (x+2) :-> q} -> {r} :=
%   with {t} := leftList' {p}
%   in
%   {r :-> h, (r+1) :-> t};
% \end{lstlisting}

\noindent
Generated C code:

\begin{lstlisting}
typedef struct Sll {
  int x_0;
  struct Sll* x_1;
} Sll;

typedef struct TreeLayout {
  int y_0;
  struct TreeLayout* y_1;
  struct TreeLayout* y_2;
} TreeLayout;

void leftList(TreeLayout* arg, Sll** r) {
  if (arg == 0) {
      // pattern match {}

      // {}
    *r = 0;
  } else {
      // pattern match {x :-> h, (x+1) :-> p
      //               , (x+2) :-> q}
    int v = arg->y_0;
    TreeLayout* p = arg->y_1;
    TreeLayout* q = arg->y_2;

      // with {t} := leftList' {p} in ...
    Sll* t = 0;
    leftList(p, &t);

      // {y :-> h, (y+1) :-> t}
    *r = malloc(sizeof(Sll));
    *r->x_0 = v;
    *r->x_1 = t;
  }
}
\end{lstlisting}

% \noindent
% In Stage~\ref{stage:in-out}

\subsection{Example: convertList}

\begin{lstlisting}
%generate convertList[Dll, Sll]
%generate convertList[Sll, Dll]

Dll : List >-> SSL(2)
Dll Nil := $\ssl${x, z} {};
Dll (Cons h t) :=
  $\ssl${x, z} {x :-> h, (x+1) :-> w, (x+2) :-> z,
              Dll t {w x}};

convertList : (a ~ Layout(List), b ~ Layout(List)) =>
  a -> b;
convertList Nil := Nil;
convertList [a b] (Cons h t) := Cons h (convertList$_{\verb|a|,\verb|b|}$ t);
\end{lstlisting}

Generated \PikaCore:

\begin{lstlisting}
convertList1 : SSL(2) -> SSL(1);
convertList1 {} := [r]{};
convertList1 {x :-> h, (x+1) :-> w, (x+2) :-> z} :=
  with {nxt} := convertList1 {w x}
  in
  [r]{r :-> h, (r+1) :-> nxt}

convertList2 : SSL(1) -> SSL(2);
convertList2 {} := $\ssl${r, z}{};
convertList2 {x :-> h, (x+1) :-> nxt} :=
  with {w, r} := convertList2 {nxt}
  in
  $\ssl${r, z} {r :-> h, (r+1) :-> w, (r+2) :-> z}
\end{lstlisting}

Generated C code:

\begin{lstlisting}
typedef struct Dll {
  int x_0;
  struct Dll* x_1;
  struct Dll* x_2;
} Dll;

void convertList1(Dll* arg, Sll** r) {
  if (arg == 0) {
      // pattern match {}

      // {}
    *r = 0;
  } else {
      // pattern match {x :-> h, (x+1) :-> w, (x+2) :-> z}
    int h = arg->x_0;
    Dll* w = arg->x_1;
    Dll* z = arg->x_2;

      // allocations for result
    *r = malloc(sizeof(Sll));

      // with {nxt} := convertList1 {w x} in ...
    Sll* nxt = 0;
    convertList1(arg, &nxt);

      // {r :-> h, (r+1) :-> nxt}
    (*r)->x_0 = h;
    (*r)->x_1 = nxt;
  }
}

void convertList2(Sll* arg, Dll** r, Dll** z) {
  if (arg == 0) {
      // pattern match {}

      // {}
    *r = 0;
  } else {
      // pattern match {x :-> h, (x+1) :-> nxt}
    int h = arg->x_0;
    Sll* nxt = arg->x_1;

      // allocations for result
    *r = malloc(sizeof(Dll));

      // with {w r} := convertList2 {nxt} in ...
    Dll* w = 0;
    convertList2(nxt, &w, r);

      // {r :-> h, (r+1) :-> w, (r+2) :-> z}
    (*r)->x_0 = h;
    (*r)->x_1 = w;
    (*r)->x_2 = *z;
  }
}
\end{lstlisting}

