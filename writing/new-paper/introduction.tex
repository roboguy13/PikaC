\section{Introduction}

Synthetic separation logic (SSL) is the specification language for the \SuSLik{} program synthesiser. \SuSLik{} will take a precondition and postcondition pair together
with a function prototype and use these to synthesise a function that satisfies that specification in a C-like language called \SuSLang.

We have developed a  functional programming language, \Pika, which will compile to SSL specifications. \SuSLik{} can then take these SSL specifications and generate \SuSLang. Finally, the resulting \SuSLang{} code can be converted to C and then run. In addition to this translational semantics, we also provide a standard abstract machine semantics for \Pika{} and use this to show correctness of the translation into SSL.

Two versions of \Pika{} have been built. \Pika{} 1 has a focus on using \SuSLik{} to generate all of the low-level code, while \Pika{} 2 adds an additional synthesis feature inside the \Pika{} language itself and has a builtin testing framework.

%In the present work, we will start by talking about \Pika{} 1, describe the fundamental limitations of using \SuSLik{} to generate everything and then use this to motivate the creation of \Pika{} 2. The two versions are then compared.

\paragraph{Contributions} In this paper, we make the following contributions:
\begin{itemize}
  \item We explore the connection between SSL and functional programming using a pair of formal semantics for \Pika: A standard abstract machine semantics and a
    semantics that translates \Pika{} into SSL specifications. The second semantics is shown to be sound relative to the first semantics. (Section~\ref{sec:semantics})

  \item In order to provide seamless interoperability between SSL and \Pika, we provide a mechanism to related algebraic data types to their representation in memory. In particular, Pika has a language construct that allows the programmer to define a kind of function from values of an algebraic data type to SSL assertions. This is called a \textit{layout}. (Section~\ref{sec:layouts})

  \item We develop an extension of \Pika{} that allows \SuSLik's synthesis capabilities to be exposed more naturally inside the \Pika{} language. (Section~\ref{sec:synth})

  \item A comparison of examples that work on the first version of \Pika{} is compared with examples that are enabled by the new feature in the second version of \Pika. This new version of \Pika{} is not backwards compatible with the first version, so we also examine which examples in the old version no longer work in the new version. (Section~\ref{sec:examples})
\end{itemize}


